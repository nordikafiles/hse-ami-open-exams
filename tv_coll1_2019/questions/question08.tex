\section{Марковские цепи. Существование стационарного распределения и сходимость к стационарному распределению.}

\subsection{Марковские цепи.}
Пусть $X$ - конечное множество, $|X| = N$. Это называют множеством состояний цепи. На декартовом произведении $X \times X$ задана функция $P(x,y)$, про которую известно, что $P(x,y) \geq 0, \forall x \in X, \sum\limits_{y \in X}P(x, y) = 1$ и называется это стохастическая матрица. Смысл этих чисел таков: $P(x,y)$ это вероятность из точки(состояния) $x$ перейти в точку(состояние) $y$. Считается, что вы обязаны что-то сделать: либо остаться в этом же состоянии, либо перейти в новое.

\subsection{Существование стационарного распределения и сходимость к стационарному распределению.}
Возьмем произвольное вероятностное распределение $\mu$. $\sum \limits_{x} \mu(x) = 1$.
Рассмотрим новое вероятностное распределение $\sigma^m = \frac{\mu + \mu P + \mu P^2 + ... + \mu P^m}{m+1}$.
То есть как мы будем искать стационарное распределение?
Возьмем любое вероятностное распределение и начнем его гонять под действием этого преобразование, и будем брать среднее арифметическое(называется среднее по времени).
Сейчас мы с вами докажем, что $\exists \sigma^{m_k}$, которая сходится к некоторому $\mu$ и $\mu$ - стационарное распределением. Но за этим должны стоять какие-то слова, иначе непонятно, в каком смысле подпоследовательность сходится в терминах вероятностного распределения. Вот тут полезно смотреть на $\mu$ как на вектор в конечно-мерном пространстве.
То есть всякое вероятностное распределение $\mu$ на $X$ - вектор на $\mathbb{R}^N$, принадлежащий множеству $\{t_1,...,t_n | \forall i  t_i \geq 0, \sum\limits_{i=1}^N t_i = 1\}$,
рассмотрим множество на размерность меньше $
    \{
        t_1,...,t_{n-1} |
        \forall i  t_i \geq 0,
        \sum\limits_{i=1}^{N-1}
        t_i \leq 1
    \}
$.
Это множество вам ничего не напоминает? В двухмерном пространстве на графике это будет треугольник. В трехмерном - пирамидка. Можно воспринимать тогда как $R^n$, а можно как пирамидку, но в любом случае это множество ограниченное и замкнутое, а то есть компакт. Что такое компакт? Каким он свойством обладает? В компакте можно выделить сходящуюся подпоследовательность, а значит во всякой последовательности существует сходящаяся подпоследовательность при чем она сходится к элементу этого множества. У нас как раз есть последовательность $\sigma^m$, и есть сходящаяся подпоследовательность $\sigma^{m_k} \to \mu$, которое является вероятностным распределением на $X$. Отображение $\mu \longmapsto \mu P$ - непрерывное отображение $\xrightarrow{} \sigma^{m_k} P \xrightarrow{k \to \infty} \mu P$.\newline
С другой стороны: $\sigma^{m_k} P = \frac{\nu P + \nu P^2 + ... + \nu P^{m_k + 1}}{m_k+1} = \sigma^{m_k} - \frac{\nu}{m_k + 1}^{\to 0} + \frac{\nu P^{m_k + 1}}{m_k + 1}^{\to 0} \xrightarrow{k \to \infty} \mu$ $\xrightarrow{} \mu = \mu P$