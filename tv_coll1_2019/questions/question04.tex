\section{Формула полной вероятности. Формула Байеса. Задача о сумасшедшей старушке. Парадокс Байеса.}

\subsection{Формула полной вероятности.}
Пусть
$$\Omega = \bigcup\limits_{i \in \{1,...,n\}} A_i \quad \text{и} \quad \forall i,j, i \neq j, A_i \cap A_j = \varnothing, \; P(A_i) > 0,$$
тогда для всякого события $B$ имеет место равенство
$$P(B) = \sum\limits_{i \in \{1,...,n\}}P(B|A_i) \cdot P(A_i).$$
\begin{proof}
    В качестве доказательства это раскладывается из
    $$\sum\limits_{i \in \{1,...,n\}} P(B|A_i)$$
\end{proof}

\subsection{Формула Байеса.}
Пусть $P(A) > 0$ и $P(B) > 0$. Тогда
$$P(B|A) = \frac{P(A|B)P(B)}{P(A)}$$
\begin{proof}
    Следует из формулы условной вероятности.
\end{proof}

\subsection{Задача о сумасшедшей старушке.}
На посадку в самолет стоят $N \geqslant 2$ пассажиров, среди которых сумасшедшая старушка. Старушка расталкивает всех пассажиров и садится в самолет на произвольнольное место. Затем пассажиры, когда заходят в самолет, садятся на свое место, если оно свободно, и на произвольное свободное место в противном случае. Какова вероятность того, что последний пассажир сядет на свое место?
\newline
Пусть эта вероятность равна $P_N$. Если $N = 2$, то $P_N = 1/2$. Предположим, что уже для всех $k \leqslant N$ доказано, что $P_k = 1/2$. Докажем равенство $P_{N+1} = 1/2$. Событие $B$ состоит из тех исходов, когда последний пассажир садится на свое место. Событие $A_m$ состоит из тех исходов, когда старушка села на место $m$-го пассажира. По формуле полной вероятности
$$P_{N+1} = P(B) = \sum_m P(B|A_m)P(A_m).$$
Заметим, что $P(A_m) = 1/(N+1)$ и все кроме двух (когда старушка села на свое место или на место последнего пассажира) вероятности $P(B|A_m) = 1/2$. Следовательно, имеем
$$P_{N+1} = \frac{N-1}{2(N+1)} + \frac{1}{N+1} = \frac{1}{2}.$$

\subsection{Парадокс Байеса.}
Пусть иеется тест, используемый для диагностики некоторого заболевания. Известно, что доля больных этим заболеванием равна 0,001. Если человек болен, то тест дает положительный результат с вероятностью 0,9. Если человек здоров, то тест дает положительный результат с вероятностью 0,01. Известно, что тест оказался положительным. Какова вероятность того, что человек на самом деле здоров?
Пусть $T_+$ и $T_-$ -- события состоящие в том, что тест дал положительный результат и тест дал отрицательный результат. Пусть также $Z_+$ и $Z_-$ -- события состоящие в том, что человек здоров и человек болен соответственно. По формуле полной вероятности
$$P(T_+) = P(T_+|Z_+)P(Z_+) + P(T_+|Z_-)P(Z_-) = 0,01 \cdot 0,999 + 0,9 \cdot 0,001 = 0,01089.$$
По формуле Байеса
$$P(Z_+|T_+) = \frac{P(T_+|Z_+) \cdot P(Z_+)}{P(T_+)} = \frac{0,01 \cdot 0,999}{0,01089} \geq 0,91$$