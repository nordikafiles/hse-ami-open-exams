\section{Дискретное вероятностное пространство. Задача о разделе ставки. Вероятностный алгоритм проверки на простоту. Универсальная хеш-функция.}

\subsection{Дискретное вероятностное пространство.}
Пусть $\Omega$ -- непустое конечное \textit{множество элементарных исходов}.
\begin{definition}
    Всякое подмножество $A \subseteq \Omega$ называют \textbf{событием}.
\end{definition}
\begin{definition}
    Функцию $P : 2^{\Omega} \to [0, 1]$, удовлетворяющую следующим свойствам:
    \begin{itemize}
        \item $P(\Omega) = 1$
        \item $A \cap B = \varnothing \Rightarrow P(A \cup B) = P(A) + P(B)$
    \end{itemize}
    называют \textbf{вероятностной мерой}, а значение $P(A)$ \textbf{вероятностью события $A$}.
    Вероятностная мера полностью определяется значениями $P(\{\omega\}) = p_{\omega}$, т.е.
    \[
        P(A) = \sum_{\omega \in A} p_{\omega}
    \]
\end{definition}
Если все элементарные исходы равновозможны, то полагаем $p_{\omega_1} = ... = p_{\omega_n} = 1/n$.

\subsection{Задача о разделе ставки.}
Два человека играют в некоторую игру, причем у обоих шансы победить одинаковые. Они договорились, что тот, кто первым выиграет 6 партий, получит весь приз. Однако игра остановилась раньше, когда первый выиграл пять партий, а второй выиграл три партии. Как справедливо разделить приз?
\newline
Предлагается разделить приз в отношении, в котором относятся вероятности выиграть для каждого из игроков в случае продолжения игры. Ясно, что еще надо сыграть не более трех партий. Пространство исходов этих трех партий состоит из восьми элементов, причем только один из этих исходов означает выигрыш второго игрока. Значит приз надо разделить в отношении 7 к 1.

\subsection{Вероятностный алгоритм проверки на простоту.}
Пусть дано некоторое натуральное число $N > 1$. Если $N$ простое число, то по малой теореме Ферма для всякого натурального числа, такого, что $(b, N) = 1$, число $b^{N -1} - 1$ делится на $N$. Следовательно, если для некоторого $b$, удовлетворяющего условию $(b, N) = 1$, число $b^{N-1} - 1$ не делится на $N$, то $N$ не является простым. Это наблюдение используют для построения простейшего теста на простоту. Если $b^{N-1} - 1$  не делится на $N$, то говорим, что $N$ не проходит тест для основания $b$.
\newline
Пусть основание мы выбираем случайно из множества $\mathbb{Z}_N^*$. Предположим, что существует такое основание, для которого $N$ не проходит тест. Какова вероятность выбрать такое основание?
\newline
Предположим, что для $a \in \mathbb{Z}_N^*$ число $N$ не проходит тест. Если $N$ проходит тест для основания $b$, то для основания $ab$ число $N$ уже тест не проходит. В противном случае $(ab)^{N-1} \equiv_N 1$ и $(b^{-1})^{N-1} \equiv_N 1$. Следовательно, $a^{N-1} \equiv (b^{-1})^{N-1}(ab)^{N-1} \equiv 1$, что противоречит предположению. Таким образом, каждому основанию $b$, для которого $N$ проходит тест, можно сопоставить основания $ab$, для котрого результат теста отрицательный. Значит, оснований, для которых $N$ не проходит тест, не м еньше оснований, для которых $N$ проходит тест на простоту. Искомая вероятность не меньше $1/2$. Если независимым образом повторять набор основания $k$ раз, то вероятность выбрать основание, для которого данное число не проходит тест, меньше $1/2^k$.
\newpage

\subsection{Универсальная хеш-функция.}
Пусть $K = \{0,1,2,...,n-1\}$ -- множество <<ключей>>. Отображение
\[
h: K \to \{0,1,2,...,m-1\}
\]
называется хеш-функцией. Предполагается, что $m < n$. Одним из важнейших свойств функции $h$ является равномерность, когда доля ключей $k$ с фиксированным значением $h(k)$ должно быть примерно $n/m$.  Это означает, что вероятность коллизии $h(k_1) = h(k_2)$ при $k_1 \neq k_2$ не больше $1/m$. Ясно, что не всегда можно предполагать, что ключи равномерно распределены по таблице. Предположим, что $h(k) = k \mod{m}$ и на вход сначала подаются ключи вида $m, 2m, 3m, ...$ Ясно, что всем таким ключам присваивается хеш-код 0 и реально никакого равномерного распределения значений не происходит. Оказывается, с этой проблемой можно справиться, если перед началом хеширования случайным образом выбирать функцию $h$ из некоторого набора таких функций.
\newline
Зафиксируем простое число $p > n$. Пусть $a \in \{1,2,...,o-1\}$ и $b \in \{0,1,2,...,o-1\}$. Положим $$h_{a,b} = ak + b \mod{p} \mod{m}.$$
Пара параметров $(a,b)$ выбирается случайным образом из множества $$\{1,2,...,p-1\} \times \{0,1,2,...,p-1\},$$ причем все элементы этого множества считаем равновероятными. Докажем, что для любых $k_1, k_2 \in \{0,2,...,n-1\}, k_1 \neq k_2$, вероятность коллизии $h_{a,b}(k_1) = h_{a,b}(k_2)$ не превосходит $1/m$.
\newline
Заметим, что $ak_1 + b = ak_2 + b \mod{p}$ тогда и только тогда, когда $k_1 = k2$. Кроме того, для различных $k_1$ и $k_2$ по значениям $ak_1 + b \mod{p}$ и $ak_2 + b \mod{p}$ однозначно находятся числа $a$ и $b$. Пусть $k_1 \neq k_2$. Отображение
$$(a,b) \to (ak_1 + b \mod{p}, ak_2 + b \mod{p})$$ является биекцией множества
$$\{1,2,...,p-1\} \times \{0,1,2,...,p-1\}$$ на множество 
$$(\{0, 1,2,...,p-1\} \times \{0,1,2,...,p-1\}) \setminus \{(i, i) \> | \> 0 \leqslant i \leqslant p - 1 \}.$$ Остается отметить, что для всякого $t \in \{0,1,2,...,p-1\}$ количетсво чисел $s \in \{0,1,2,...,p-1\}$ таких, что $s \neq t$ и $s = t \mod{m}$, не превосходит $(p-1)/m$, т.е. вероятность выбора такой пары $(t,s)$ или, что эквивалентно, выбора пары $(a,b)$, у которой $h_{a,b}(k_1) = h_{a,b}(k_2)$, не превосходит $1/m$.