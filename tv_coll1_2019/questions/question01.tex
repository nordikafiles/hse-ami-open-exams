\section{Дискретное вероятностное пространство. Задача о разделе ставки. Вероятностный алгоритм проверки на простоту. Универсальная хеш-функция.}

\subsection{Дискретное вероятностное пространство.}
Пусть $\Omega$ -- непустое конечное \textit{множество элементарных исходов}.
\begin{definition}
    Всякое подмножество $A \subseteq \Omega$ называют \textbf{событием}.
\end{definition}
\begin{definition}
    Функцию $P : 2^{\Omega} \to [0, 1]$, удовлетворяющую следующим свойствам:
    \begin{itemize}
        \item $P(\Omega) = 1$
        \item $A \cap B = \varnothing \Rightarrow P(A \cup B) = P(A) + P(B)$
    \end{itemize}
    называют \textbf{вероятностной мерой}, а значение $P(A)$ \textbf{вероятностью события $A$}.
    Вероятностная мера полностью определяется значениями $P(\{\omega\}) = p_{\omega}$, т.е.
    \[
        P(A) = \sum_{\omega \in A} p_{\omega}
    \]
\end{definition}
Если все элементарные исходы равновозможны, то полагаем $p_{\omega_1} = ... = p_{\omega_n} = 1/n$.

\subsection{Задача о разделе ставки.}
Два человека играют в некоторую игру, причем у обоих шансы победить одинаковые. Они договорились, что тот, кто первым выиграет 6 партий, получит весь приз. Однако игра остановилась раньше, когда первый выиграл пять партий, а второй выиграл три партии. Как справедливо разделить приз?
\newline
Предлагается разделить приз в отношении, в котором относятся вероятности выиграть для каждого из игроков в случае продолжения игры. Ясно, что еще надо сыграть не более трех партий. Пространство исходов этих трех партий состоит из восьми элементов, причем только один из этих исходов означает выигрыш второго игрока. Значит приз надо разделить в отношении 7 к 1.

\subsection{Вероятностный алгоритм проверки на простоту.}
Пусть дано некоторое натуральное число $N > 1$. Если $N$ простое число, то по малой теореме Ферма для всякого натурального числа, такого, что $(b, N) = 1$, число $b^{N -1} - 1$ делится на $N$. Следовательно, если для некоторого $b$, удовлетворяющего условию $(b, N) = 1$, число $b^{N-1} - 1$ не делится на $N$, то $N$ не является простым. Это наблюдение используют для построения простейшего теста на простоту. Если $b^{N-1} - 1$  не делится на $N$, то говорим, что $N$ не проходит тест для основания $b$.
\newline
Пусть основание мы выбираем случайно из множества $\mathbb{Z}_N^*$. Предположим, что существует такое основание, для которого $N$ не проходит тест. Какова вероятность выбрать такое основание?
\newline
Предположим, что для $a \in \mathbb{Z}_N^*$ число $N$ не проходит тест. Если $N$ проходит тест для основания $b$, то для основания $ab$ число $N$ уже тест не проходит. В противном случае $(ab)^{N-1} \equiv_N 1$ и $(b^{-1})^{N-1} \equiv_N 1$. Следовательно, $a^{N-1} \equiv (b^{-1})^{N-1}(ab)^{N-1} \equiv 1$, что противоречит предположению. Таким образом, каждому основанию $b$, для которого $N$ проходит тест, можно сопоставить основания $ab$, для котрого результат теста отрицательный. Значит, оснований, для которых $N$ не проходит тест, не м еньше оснований, для которых $N$ проходит тест на простоту. Искомая вероятность не меньше $1/2$. Если независимым образом повторять набор основания $k$ раз, то вероятность выбрать основание, для которого данное число не проходит тест, меньше $1/2^k$.

\subsection{Универсальная хеш-функция.}