\section{Схема Бернулли. Моделирования бросания правильной монеты. Теорема Муавра-Лапласа. Закон больших чисел для схемы Бернулли.}

\subsection{Схема Бернулли.}
Рассмотрим следующий эксперимент: $N$ раз бросается монета с вероятностью выпадения орла (успеха) $p$, причем результат одного бросания не влияет на результаты других бросаний. Нас интересует число выпадений орла или число успехов. Пусть $q = 1 - p$.
\newline
Можно считать, что множество элементарных исходов состоит из последовательностей 0 и 1 длины $N$, где 1 соответствует успеху. Каждому исходу с $k$ единицами сопоставляем вероятность $p^k q^{N-k}$. Построенное вероятностное пространство называют \textit{схемой Бернулли}.
\newline
Так как всего исходов с $k$ единицами $C_N^K$, то вероятность события $A_k,N$, что в последовательности длины $N$ ровно $k$ единиц, равна
$$P(A_k, N) = C_N^k p^k q^{N-k}.$$


\subsection{Моделирования бросания правильной монеты.}
Пусть монета бросается $N$ раз. Будем смотреть на четность количества выпавших орлов. Рассмотрим следующие вероятности:
\[
    P(\text{<<количество орлов четно>>}) = \textit{Ч}, \quad P(\text{<<количество орлов нечетно>>}) = \textit{Н}
\]
Очевидно, что $\textit{Ч} + \textit{Н} = (p+q)^N = 1$ (бином Ньютона). Теперь рассмотрим их разность:
\[
    \textit{Ч} - \textit{Н} = \sum_{k = 2m} C_N^k p^k q^{N-k} - \sum_{k = 2m + 1} C_N^k p^k q^{N-k} = (-p + q)^N
\]
Таким образом,
\[
    \textit{Ч} = \frac{1 + (1 - 2p)^N}{2} \longrightarrow \frac{1}{2}
\]

\subsection{Теорема Муавра-Лапласа. (TODO)}
Будем бросать правильную монетку. Тогда вероятность выпадения ровно $n$ орлов при совершении $2n$ бросков будет следующей:
\[
    P_{2n, n} = C_{2n}^n \cdot \frac{1}{2^{2n}} = \frac{(2n)!}{n! \cdot n! \cdot 2^{2n}}
\]
Приблизим все это дело с помощью формулы Стирлинга:
\[
    P_{2n, n} = \frac{(2n)!}{n! \cdot n! \cdot 2^{2n}} =
    \frac{
        \sqrt{2\pi} \cdot (2n)^{2n + 1/2} \cdot e^{2n} \cdot e^{O(1/n)}
    }{
        \sqrt{2\pi} \cdot n^{n + 1/2} \cdot e^{n} \cdot
        \sqrt{2\pi} \cdot n^{n + 1/2} \cdot e^{n} \cdot
        2^{2n}  \cdot e^{O(1/n)}
    } =
    \frac{
        2^{2n + 1/2} \cdot n^{2n + 1/2}  \cdot e^{O(1/n)}
    }{
        n^{n + 1/2} \cdot
        \sqrt{2\pi} \cdot n^{n + 1/2} \cdot
        2^{2n}
    } =
    \frac{
        e^{O(1/n)}
    }{
        \sqrt{\pi n}
    }
\]
% Мы уже знает, что $P($число единиц равно $k) = C_{N}^{k}p^k q^{N-k}$. Обозначим это число за $P_{N,k}$. Для простоты будем считать, что $p=q=\frac{1}{2}$, хоть мы и сформируем с такими вероятностями, но теорема будет доказана целиком. Давайте посмотрим, как будет выглядеть график наших вероятностей при $N=5$. $P_{5,0} = \frac{1}{2^5}, P_{5,1} = \frac{5}{2^5}, P_{5,2} = \frac{10}{2^5}, P_{5,3} = \frac{10}{2^5}, P_{5,4} = \frac{5}{2^5}, P_{5,5} = \frac{1}{2^5}$. получилась "зеркальная гистограмма". Мы видим, что эта линия формируется $C$-шками. А как описать эту линию? Она в точности до масштабирования равна функции $\phi(x) = \frac{1}{\sqrt{2\pi}} \cdot e^{-\frac{x^2}{2}}$. Давайте теперь посмотрим на число $P_{2n,n}$, но перед этим вспомним формулу Стирлинга, которая говорит, что $n!=\sqrt{2\pi} n^{n + \frac{1}{2}} \cdot e^{-n + O(\frac{1}{n})}$.  Теперь заметим, что если прологарифмировать $ln(n!) = ln(1) + ln(2) + .... + ln(n) = \int\limits_{1}^{n} ln(x) dx = x ln(x) - x \big{|}_{1}^{n} = n ln(n) - n + 1$. Теперь мы что говорим? Что $P_{2n, n} = C_{2n}^{n} = \frac{1}{2^{2n}} \cdot \frac{(2n)!}{n! \cdot n!} = \frac{(2n)!}{n! \cdot n! \cdot 2^{2n}} = \frac{\sqrt{2\pi}(2n)^{2n + \frac{1}{2}} \cdot e^{-2n + O(\frac{1}{n})}}{(\sqrt{2\pi} n^{n + \frac{1}{n}} \cdot e^{-n + O(\frac{1}{n})})\cdot (\sqrt{2\pi}n^{n + \frac{1}{2}} \cdot e^{-n + O(\frac{1}{n})}) \cdot 2^{2n}} = \frac{\sqrt{2\pi} \cdot 2^{2n} \cdot \sqrt{2} \cdot n^{2n} \cdot \sqrt{n} \cdot e^{-2n + O(\frac{1}{n})}}{(\sqrt{2\pi} n^{n}\sqrt{n} \cdot e^{-n + O(\frac{1}{n})})\cdot (\sqrt{2\pi}n^{n}\sqrt{n} \cdot e^{-n + O(\frac{1}{n})}) \cdot 2^{2n}} = \frac{1}{\sqrt{\pi n} \cdot e^{(\frac{1}{n})}} \to P_{2n,n} \sim \frac{1}{\sqrt{\pi n}}$. Таким образом мы нашли центр. Теперь же узнаем, как обстоят дела на $k$ от центра, а именно $P_{2n, n+k}$ ($0 < k < n$, для отрицательных будет аналогично) и хотим узнать, как это относится к $\frac{P_{2n, n+k}}{P_{2n,n}} = \frac{C_{2n}^{n+k}}{C_{2n}^{n}} = \frac{\frac{(2n)!}{(n+k)! (n-k)! \cdot 2^{2n}}}{\frac{(2n)!}{n! n! \cdot 2^{2n}}} = \frac{n!n!}{(n-k)!(n+k)!} = \frac{n(n-1)...(n-(k-1))}{(n+1)(n+2)...(n+k)}  = \frac{1(1 - \frac{1}{n})(1-\frac{2}{n})...(1-\frac{k-1}{n})}{(1+\frac{1}{n})...(1+\frac{k}{n})}$. Теперь скажем, что $ln(1+x) = x - \frac{x^2}{2} + O(x^3)$, тогда мы нашу дробь пролагарифмируем, а логарифм нам будет давать сумму: $\frac{e^{\sum\limits_{m=1}^{k-1} ln(1 - \frac{m}{n})}}{e^{\sum\limits_{n=1}^k ln(1+\frac{m}{n})}} = e^{\sum\limits_{m=1}^{k-1}(-\frac{m}{n} - \frac{m^2}{2n^2} + O(\frac{m^3}{n^3})) - \sum\limits_{m=1}^{k} (\frac{m}{n} - \frac{m^2}{2n^2} + O(\frac{m^3}{n^3}))} = e^{-\frac{2}{n}\sum\limits_{m=1}^{k}(m) - \frac{k}{n} + \frac{k^2}{2n^2} - O(\frac{k^3}{n^3})} = e^{-\frac{2}{n} \cdot \frac{(k-1)k}{2} - \frac{k}{n} + \frac{k^2}{2n^2} - O(\frac{k^3}{n^3})} = e^{\frac{k^2}{n} + \frac{k^2}{2n^2} - O(\frac{k^3}{n^3})}$. Теперь мы можем сказать, что $\frac{P_{2n, n+k}}{P_{2n, n}} \sim e^{\frac{k^2}{n} + \frac{k^2}{2n^2}}$ или же $\frac{P_{2n, n+k}}{P_{2n, n}} = e^{\frac{k^2}{n} + \frac{k^2}{2n^2} - O(\frac{k^3}{n^3})}$. Теперь, если вспомнить, что $\phi(x) = \frac{1}{\sqrt{2\pi}}e^{-\frac{x^2}{2}}$, то $P_{N,n} \sim \frac{1}{\sqrt{\frac{N}{3}}}\phi(\sqrt{\frac{N}{4}})$. Теперь будем менять $k$, сначала $k=\frac{1}{\sqrt{\frac{N}{4}}}$, потом $\frac{2}{\sqrt{\frac{N}{4}}}$ итд. Напоминает интеграл: $\sum\limits_{0 < k < 2\sqrt{\frac{N}{4}}} \frac{1}{\sqrt{\frac{N}{4}}} \phi(\sqrt{\frac{N}{4}}) \to \int\limits_{0}^{2} \phi(x) dx$, $\xrightarrow{} \sum\limits_{0 < k < \sqrt{N}} P_{N, \frac{N}{2} + k} \to \int\limits_{0}^{2} \phi(x) dx$, когда $N \to \infty$. За что же у нас отвечает $k$? Как отходит от середины наша гистограмма. То есть $P(\frac{N}{2} < $ число успехов $< \frac{N}{2} + \sqrt{N}) \to \int\limits_{0}^{2} \phi(x) dx$ . Поделим в нашем неравенстве на $N$: $P(\frac{1}{2}$ < число успехов$/N < \frac{1}{2} + \frac{1}{\sqrt{N}}) \to \int\limits_{0}^{2} \phi(x) dx$. Это значит, что вероятность того, что доля успеха в эксперименте отличается от той доли, которую мы ожидаем на величину не более $\frac{1}{\sqrt{N}}$. А теперь сформируем в общем виде: $X_{N,k} = \frac{k - Np}{\sqrt{Npq}}$, существует константа $C$, которая не зависит от $N$, такая, что $X_{N,k} \leq C$, тогда $P_{N,k} \sim \frac{1}{Npq} \cdot \phi(X_{N,k})$ при $N \to \infty$. Для любых чисел $a < b$ имеем при $N \to \infty$ $P(a \leq \frac{k - Np}{Npq} \leq b) \to \int\limits_{a}^{b} \phi(x) dx$. В качестве док-во последнего в неравенстве домножим на $\sqrt{\frac{pq}{N}}$, и получим при $N \to \infty$ $P(a\sqrt{\frac{pq}{N}} \leq \frac{k}{N} - p \leq b\sqrt{\frac{pq}{N}}) \to \int\limits_{a}^{b} \phi(x) dx$.

\subsection{Закон больших чисел для схемы Бернулли.  (TODO)}