\section{Теорема Пуассона. Распределение Пуассона. Задача про изюм. Пуассоновский процесс.}

\subsection{Задача про изюм.}
Сколько изюма должны содержать в среднем булочки, для того чтобы вероятность иметь хотя бы одну изюминку в булочки была не меньше 0,99?
\newline
Предположим, что уже изготовлено тесто на некоторое количество булочек. В это тесто добавлено $N$ изюминок так, что соотношение числа изюминок к количеству булочек равно $\lambda$. Значит количество булочек равно $N/\lambda$.
\newline
Выделим в тесте кусок, из которого будет изготовлена данная булочка. Вероятность попадания одной изюминку в эту булочку равна $\lambda / N$, а вероятность того, что хотя бы одна изюминка попала в булку, равна
\[
    1 - \left(
        1 - \frac{\lambda}{N}
    \right)^N.
\]
Поскольку мы рассматриваем серийное производство булочек, то можно предполагать, что $N \to +\infty$, т.е. растет объем теста и количество изюма, но не меняется плотность $\lambda$. Получаем
\[
    \left(
        1  - \frac{\lambda}{N}
    \right)^{N} \to e^{-\lambda}.
\]
Для решения задачи надо найти $\lambda$ такое, что $e^{-\lambda} , 0,01$. Подходит $\lambda = 5$, т.е. плотность изюма должна быть не менее пяти изюминок на булочку.
\newline
Мы рассматривали серии событий, причем $N$-ая состоит из $N$ событий. Например, в задаче про горшочек каши такими событиями являются попадание $i$-й ягоды в половник. В каждой серии все события независимы в совокупности и в $N$-й серии вероятность каждого события равна $p_N$, причем число $N \cdot p_N = \lambda$ не зависит от $N$. Нас интересует вероятность $P(A_{k, N})$ наступления ровно $k$ событий в данной серии из $N$ событий. Поскольку рассматриваемая ситуация представляет собой схему Бернулли, то вероятность $P(A_{k, N})$ вычисляется по формуле $$C_N^j p_N^k (1 - p_n)^{N-k}.$$

\subsection{Теорема Пуассона.}
\begin{theorem}
    Пусть $N \cdot p_N = \lambda$ -- не зависит от $N$. Тогда
    \[
        P(A_{k, N}) = C_N^k p_N^k (1 - p_n)^{N - k} \to \frac{\lambda^k}{k!} \cdot e^{-\lambda},
        \quad
        N \to +\infty
    \]
    \begin{proof}
        Распишем вероятность $P(A_{k, N})$ в следующем виде:
        \[
            P(A_{k, N}) = \frac{\lambda^k}{k!}
            \left( 1 - \frac{1}{N} \right)
            ...
            \left( 1 - \frac{k-1}{N} \right)
            \left( 1 - \frac{\lambda}{N} \right)^{-k}
            \left( 1 - \frac{\lambda}{N} \right)^{N}.
        \]
        Учитывая, что $\lambda$ и $k$ не меняются, устремляем $N \to \infty$ и получаем искомое выражение.
    \end{proof}
\end{theorem}

\subsection{Распределение Пуассона.}
\begin{definition}
    Набор вероятностей $\{ \frac{\lambda^k}{k!} e^{-\lambda} \}$ называется \textbf{распределением Пуассона}.
\end{definition}

\subsection{Пуассоновский процесс.}