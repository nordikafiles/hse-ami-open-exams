\section{Теорема Пуассона. Распределение Пуассона. Задача про изюм. Пуассоновский процесс.}

\subsection{Задача про изюм.}
Сколько изюма должны содержать в среднем булочки, для того чтобы вероятность иметь хотя бы одну изюминку в булочки была не меньше 0,99?
\newline
Предположим, что уже изготовлено тесто на некоторое количество булочек. В это тесто добавлено $N$ изюминок так, что соотношение числа изюминок к количеству булочек равно $\lambda$. Значит количество булочек равно $N/\lambda$.
\newline
Выделим в тесте кусок, из которого будет изготовлена данная булочка. Вероятность попадания одной изюминку в эту булочку равна $\lambda / N$, а вероятность того, что хотя бы одна изюминка попала в булку, равна
\[
    1 - \left(
        1 - \frac{\lambda}{N}
    \right)^N.
\]
Поскольку мы рассматриваем серийное производство булочек, то можно предполагать, что $N \to +\infty$, т.е. растет объем теста и количество изюма, но не меняется плотность $\lambda$. Получаем
\[
    \left(
        1  - \frac{\lambda}{N}
    \right)^{N} \to e^{-\lambda}.
\]
Для решения задачи надо найти $\lambda$ такое, что $e^{-\lambda} , 0,01$. Подходит $\lambda = 5$, т.е. плотность изюма должна быть не менее пяти изюминок на булочку.
\newline
Мы рассматривали серии событий, причем $N$-ая состоит из $N$ событий. Например, в задаче про горшочек каши такими событиями являются попадание $i$-й ягоды в половник. В каждой серии все события независимы в совокупности и в $N$-й серии вероятность каждого события равна $p_N$, причем число $N \cdot p_N = \lambda$ не зависит от $N$. Нас интересует вероятность $P(A_{k, N})$ наступления ровно $k$ событий в данной серии из $N$ событий. Поскольку рассматриваемая ситуация представляет собой схему Бернулли, то вероятность $P(A_{k, N})$ вычисляется по формуле $$C_N^j p_N^k (1 - p_n)^{N-k}.$$

\subsection{Теорема Пуассона.}
\begin{theorem}
    Пусть $N \cdot p_N = \lambda$ -- не зависит от $N$. Тогда
    \[
        P(A_{k, N}) = C_N^k p_N^k (1 - p_n)^{N - k} \to \frac{\lambda^k}{k!} \cdot e^{-\lambda},
        \quad
        N \to +\infty
    \]
    \begin{proof}
        Распишем вероятность $P(A_{k, N})$ в следующем виде:
        \[
            P(A_{k, N}) = \frac{\lambda^k}{k!}
            \left( 1 - \frac{1}{N} \right)
            ...
            \left( 1 - \frac{k-1}{N} \right)
            \left( 1 - \frac{\lambda}{N} \right)^{-k}
            \left( 1 - \frac{\lambda}{N} \right)^{N}.
        \]
        Учитывая, что $\lambda$ и $k$ не меняются, устремляем $N \to \infty$ и получаем искомое выражение.
    \end{proof}
\end{theorem}

\subsection{Распределение Пуассона.}
\begin{definition}
    Набор вероятностей $\{ \frac{\lambda^k}{k!} e^{-\lambda} \}$ называется \textbf{распределением Пуассона}.
\end{definition}
\newpage

\subsection{Пуассоновский процесс.}
В случайные моменты времени регистрируются некоторые события. Будем отмечать эти моменты времени точками на луче $[0, +\infty)$. Обозначим через $X(t)$ число точек на временном промежутке $(0, t)$. Нас интересует вероятность $P_k(t)$ того, что $X(t) = k$. Будем предполагать, что
\begin{enumerate}
    \item[1)] вероятность попадания $k$ точек в данный промежуток зависит только от длины этого промежутка, но не зависит от его расположения
    \item[2)] для любой конечной системы промежутков, которые могут попарно пересекаться лишь концами, попадания точек в каждый из них являются независимыми в совокупности событиями
    \item[3)] вероятность попадания по крайней мере двух точек в интервал длины $\delta$ равна $o$-малым от $\delta$
\end{enumerate}
\begin{definition}
    Величина $X(t)$ называется \textbf{пуассоновским процессом}.
\end{definition}
Рассмотрим сначала $P_0(t)$. Разделим промежуток $[0,1]$ на $N$ промежутков. По свойству (1) вероятность отсутствия событий на каждом промежутке разбиения равна $P_0(t/N)$. По свойству (2) регистрация события на одном промежутке разбиения не зависит от регистрации событий на других промежутках. Следовательно, мы имеем дело со схемой Бернулли и вероятность отсутствия событий на $[0, t]$ равна $P_0(t) = P_0(t/N)^N$. Положим $P_0(1) = q$. Тогда $P_0(1/N) = q^{\frac{1}{N}}$ и $P_0(m/N) = q^{\frac{m}{N}}$. Заметим, что $P_0(t + h) = P_0(t)P_0(h) \leqslant P_0(t)$, т.е. $P_0(t)$ не возрастает. Следовательно, для $\frac{m-1}{N} \leqslant t \leqslant \frac{m}{N}$ выполняются неравенства $q^{\frac{m-1}{N}} \leqslant P_0(t) \leqslant q^{\frac{m}{N}}$. Приближая $t$ последовательностью дробей $m/N$, приходим к равенству $P_0(t) = q^t$. Положим $\lambda = - \ln{q} > 0$. Тогда $P_0(t) = e^{- \lambda t}$.
\newline
Теперь вычислим $P_k(t)$ при $k > 0$. Опять разобьем промежуток $[0, t]$ на $N$ промежутков. Пусть $B$ -- событие, состоящее в том, что хотя бы на одном из промежутков зарегистрированы по крайней мере два события. Тогда противоположное событие $\overline{B}$ состоит в том, что в каждом промежутке регистрируется не более одного события. Заметим, что по свойству (3) вероятность $B$ не превосходит $N \cdot o(t/N) = o(t)$, что стремится к нулю при $N \to \infty$. Рассмотрим теперь событие $\overline{B}_k$, состоящее в том, что на промежутке $[0, t]$ зарегистрировано ровно $k$ событий и на каждом промежутке разбиения не более одного события. Вероятность отсутствия на одном промежутке разбиения равна $P_0(t/N) = e^{-\lambda t / N}$ и мы опять находимся в ситуации схемы Бернулли. Следовательно, имеем
\[
    P(\overline{B}_k) = C_n^k (e^{-\lambda t / N})^{N-k} (1 - e^{- \lambda t / N})^k.
\]
Устремляем здесь $N \to \infty$ и в качестве предела получаем $\frac{(\lambda t)^k}{k!} e^{-\lambda t}$. С учетом сказанного про стремление к нулю $P(B)$ приходим к равенству 
\[
    P_k(t) = \frac{(\lambda t)^k}{k!} e^{-\lambda t},
\]
т.е. получаем распределение Пуассона. Число $\lambda$ называется \textbf{интенсивностью или параметром процесса $X(t)$}.