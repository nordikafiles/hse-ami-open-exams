\section{Условная вероятность. Независимые события. Отличие попарной независимости и независимости в совокупности.}

\subsection{Условная вероятность.}
Условной вероятностью $A$ при событии $B$ называется число $$P(A|B) = \frac{P(A \cap B)}{P(B)}.$$
Если зафиксировать $B$, то $P'(x) = P(x | B)$ является вероятностной мерой.
Равенство часто записывают как $P(A \cap B) = P(B) \cdot P(A | B)$ и называют правилом произведения.

\subsection{Независимые события.}
События $A$ и $B$ называются независимыми, если $P(A \cap B) = P(A) \cdot P(B)$.

\subsection{Отличие попарной независимости и независимости в совокупности.}
