\section{Свойства вероятностной меры. Формула включений и исключений. Парадокс распределения подарков. Задача про конференцию.}

\subsection{Свойства вероятностной меры.}
\begin{itemize}
    \item $P(A \cup B) = P(A) + P(B) - P(A \cap B) \leq P(A) + P(B)$
    \item $P(\bigcup\limits_{k} A_k) \leq \sum\limits_{k} P(A_k)$
    \item $P(\bigcup\limits_{k} A_k) = \sum\limits_{k=1}^n (-1)^{k+1} \sum\limits_{i_1<...<i_k} P(A_{i_1} \cap ... \cap A_{i_k})$
\end{itemize}

\subsection{Формула включений и исключений.}
$$P(\bigcup\limits_{k} A_k) = \sum\limits_{k=1}^n (-1)^{k+1} \sum\limits_{i_1<...<i_k} P(A_{i_1} \cap ... \cap A_{i_k})$$
Докажем индуктивно. Для $k=1$ очевидно, что верно. Для $k=2$ проверено перывм свойством. Допустим для $n$ верно, докажем для $n+1$: $P(A_1 \cup ... \cup A_n \cup A_{n+1}) = P(A_1 \cup ... \cup A_n) + P(A_{n+1}) - P((A_1 \cap A_{n+1}) \cup ... \cup (A_n \cap A_{n+1})) = P(A_1) + ... + P(A_n) - \sum\limits_{j < k} P(A_j \cap A_k) + P(A_{n+1}) + P(A_1 \cap A_{n+1}) + ... + P(A_n \cap A_{n+1}) - \sum\limits_{j < k}P(A_j \cap A_k \cap A_{n+1})$ и так далее. Таким образом, переместив все, где есть $A_{n+1}$ под знаки суммирования, мы получим исходную формулу.

\subsection{Парадокс распределения подарков.}
Несколько человек решили сделать друг другу подарки следующим образом. Каждый приносит подарок. Подарки складываются вместе, перемешиваются и случайно распределяются среди участников. Этот справедливый способ раздачи подарков применяется часто, так как считают, что для больших групп людей вероятность совпадения, т. е. получения кем-то собственного подарка, очень мала. Парадоксально, но вероятность по крайней мере одного совпадения намного больше вероятности того, что совпадений нет (кроме случая, когда группа состоит из двух человек, тогда вероятность отсутствия совпадений равна $1/2$).
\begin{proof}
    Рассмотрим компанию из $n$ человек, тогда число подарков также равно $n$. Подарки могут быть распределены $n!$ различными способами. (Это общее число исходов.) Число исходов, в которых никто не получит свой собственный подарок, равно
    $$\binom{n}{0}n! - \binom{n}{1}(n-1)! + ... + (-1)^n 0!,$$
    так что отношение числа благоприятных исходов к общему числу исходов вычисляется по формуле
    $$P = \frac{1}{2!} - \frac{1}{3!} + ... + \frac{(-1)^n}{n!}$$
    и $P$ действительно меньше $1/2$ при $n > 2$.
    Таким образом, при $n = 6$ имеем
    $$\frac{1}{2!} - \frac{1}{3!} + \frac{1}{4!} - \frac{1}{5!} + \frac{1}{6!} = \frac{53}{144} \approx 0.36$$
\end{proof}
\newpage

\subsection{Задача про конференцию.}
В научном центре работают специалисты по 60 различным разделам компьютерных наук. Известно, что по каждому разделу в центре работает ровно 7 ученых, причем вполне может быть, что один ученый является специалистом сразу по нескольким направлениям. Все ученые должны принять участие в одной (и только одной) из двух конференций, одна из которых проходит в Канаде, а другая в Австралии. Оказывается, что всегда можно так распределить ученых по этим конференциям, что на каждой конференции будут присутствовать специалисты по всем 60 направлениям компьютерных наук.
\begin{proof}
    Будем для каждого ученого выбирать конференцию простым подбрасыванием правильной монеты. Для каждого направления компьютерной мысли рассмотрим событие, состоящее в том, что среди ученых этого направления окажутся и те, которые поехали в Канаду, и те, которые поехали в Австралию. Вероятность этого события равна $1 - 2^{-6}$ (нас устроят все исходы кроме двух, когда все отправились на конференцию в одну страну). Остается заметить, что число событий равно 60 и вероятность каждого события больше $1 - 60^{-1}$.
\end{proof}