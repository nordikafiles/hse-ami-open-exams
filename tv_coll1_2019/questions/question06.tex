\section{Случайное блуждание: принцип отражения, задача о баллотировке и задача о возвращении в начало координат. Броуновское движение.}

\subsection{Случайное блуждание}
Схема Бернулли имеет красивую геометрическую интерпретацию.
По числовой прямой двигается частица, которая каждую секунду перемешается на единицу вправо или на единицу влево, причем выбор обоих направлений равновозможен и не зависит от соответствующего выбора на других шагах. Мы считаем, что в начальный момент времени частица находится в точке $x = 0$. Ясно, что траекторию движения частицы за $N$ перемещений можно закодировать последовательностью из 1 или -1 длины $N$. Набор таких последовательностей -- пространство элементарных исходов. Вероятность каждой траектории равна $2^{-N}$. Таким образом, с точностью до обозначений мы получили схему Бернулли, описывающую бросание правильной монеты.
\newline
При исследовании случайного блуждания, нас интересует вероятность того, что траектория обладает некоторым свойством. Какова вероятность того, что частица не возвращается в начало координат? Какова вероятность, что $N$-м шаге частица первый раз вернулась в начало координат?
\newline
Траектории частицы изображаем на координатной плоскости переменных $(t, x)$ в виде ломанных, соединяющих точки с целочисленными координатами $t$ и $x$. Здесь $x$ -- положение частицы, а $t$ -- время.
\subsection{Принцип отражения}
\begin{theorem}
    Пусть $x_0 > 0, x_1 > 0$ и $t_0 < t_1$. Число путей из $(t_0, x_0)$ в $(t_1, x_1)$, которые касаются или пересекают ось времени, равно числу путей из $(t_0, -x_0)$ в $(t_1, x_1)$.
    \begin{proof}
        Очевидно, что существует биекция.
    \end{proof}
\end{theorem}

\subsection{Задача о баллотировке}
Какова вероятность того, что частица, которая вышла из нуля и пришла в точку $k > 0$ за $N$ шагов, все время пути находилась в точках с положительными координатами? Обратим внимание, что требуется вычислить условную вероятность, где условием является то, что частица за $N$ шагов пришла в точку $k$. Следовательно, надо среди таких путей найти долю тех, которые проходят только через точки с положительными координатами.
\newline
Рассматриваемая задача иеет интересную интерпретацию и называется <<теоремой о баллотировке>>. Если на выборах один кандидат набрал $q$ голосов, а другой $r$ голосов и $r > q$, то какова вероятность того, что победивший кандидат все время выборов был впереди? Предполагается, что голосовавшие не имели предпочтений и отдавали свой голос случайно, а подсчет голосов происходил последовательно.
\newline
В первый момент времени частица с вероятностью $1/2$ перемешается в точку $x = 1$ или $x = -1$. Нас устраивает только первый вариант. Затем, нужная нам траектория частицы соединяет точки $(1,1)$ и $(N, k)$ и не касается и не пересекает ось времени. По принципу отражения мы умеем считать число остальных траекторий, соединяющих $(1,1)$ и $(N, k)$. Таких траекторий ровно столько, сколько всего траекторий из $(1, -1)$ в $(N, k)$. Несложно посчитать, что таких траекторий $C_{N-1}^{\frac{N+k}{2}}$. Всего траекторий из $(1,1)$ в $(N, k)$ равно $C_{N-1}^{\frac{N+k}{2} - 1}$. Следовательно, число нужных нам траекторий частицы
\[
    C_{N-1}^{\frac{N+k}{2} - 1} - C_{N-1}^{\frac{N+k}{2}} = \frac{k}{N} \cdot C_{N}^{\frac{N+k}{2}}.
\]
Здесь $C_{N}^{\frac{N+k}{2}}$ -- количество путей, соединяющих начало координат и точку $(N, k)$. Значит вероятность искомого события равна $\frac{k}{N}$. В условиях задачи о баллотировке соответствующая вероятность равна $\frac{r-q}{r+q}$.
\newpage

\subsection{Задача о возвращении в начало координат}
Пусть чатсица вышла из начала координат. Обозначим через $u_{2n}$ вероятность того, что в момент времени $t = 2n$ частица вернулась в точку $x = 0$, а через $f_{2n}$ вероятность того, чтро это произошло первый раз.
\newline
Легко посчитать, что $u_{2n} = C_{2n}^n \cdot 2^{-2n}$. Сложнее найти $f_{2n}$. Частица приходит в точку $x = 0$ в момент времени $2n$ из точек $x = 1$ или $x = -1$ в момент времени $t = 2n - 1$. Число путей в точку $(2n-1, 1)$ из начала координат таких, что все координаты точек, через которые проходит путь, положительные, равно $\frac{1}{2n -1} \cdot C_{2n-1}^n$. Столько же путей в точку $(2n-1, -1)$ из начала координат таких, что все координаты точек, через которые проходит путь, отрицательные. Следовательно, всего нужных нам путей $\frac{2}{2n-1} \cdot C_{2n-1}^n$ и
\[
    f_{2n} = \frac{2}{2n - 1} \cdot C_{2n-1}^n \cdot 2^{-2n}.
\]
Несложно проверить, что $f_{2n} = u_{2n-2} - u_{2n}$ и $f_{2n} = (2n)^{-1} \cdot u_{2n-2}$. Формула Стирлинга позволяет найти асимптотику этих вероятостей:
\[
    u_{2n} \sim \frac{1}{\sqrt{\pi n}},
    \quad
    f_{2n} \sim \frac{1}{2 \sqrt{\pi} \cdot n^{3/2}}.
\]

\subsection{Броуновское движение.}
Будем считать, что частица за время $\Delta t$ перемещается вправо или влево на $\Delta x$, где уже не предполагается, что $\Delta t$ и $\Delta x$ равны единице. Пусть в момент времени $t$ частица находится в точке $X(t)$. Хотим узнать распределение значений $X(t)$. Предположим, что $t = N \Delta t$. Если за эти $N$ перемещений частица $k$ раз перемещалась вправо, то
\[
    X(t) = k \Delta x + (N - k)(- \Delta x) = (2k - N) \Delta x.
\]
Из наблюдений известно, что $|\Delta x|^2 = \sigma \Delta t$ для некоторого числа $\sigma > 0$. Тогда
\[
    X(t) = \frac{k - \frac{N}{2}}{\sqrt{\frac{N}{4}}} \sqrt{t \sigma}.
\]
Для моделирования непрерывного движения частицы устремим $\Delta t$ к нулю. Это равносильно тому, что $N \to \infty$. По теореме Муавра-Лаплаа
\[
    \lim_{N \to \infty} P \left(
        a \leqslant X(t) \leqslant b
    \right) =
    \frac{1}{\sqrt{2\pi}} \int_{\frac{a}{\sqrt{t \sigma}}}^{\frac{b}{\sqrt{t \sigma}}} e^{-x^2 / 2} dx = 
    \int_a^b \frac{1}{\sqrt{2 \pi \sigma t}} e^{-\frac{x^2}{2 \sigma t}} dx.
\]
Таким образом, вероятность того, что частица в момент времени $t$ находится в $[a,b]$ вычисляется с помощью плотности $$\frac{1}{\sqrt{2\pi \sigma t}} e^{-\frac{x^2}{2 \sigma t}}.$$