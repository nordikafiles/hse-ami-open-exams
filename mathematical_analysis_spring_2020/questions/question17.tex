\section{Билет 17.}

\subsection{Дайте определение касательного вектора к подмножеству $x \subseteq \mathbb{R}^n$ в точке $A \in X$.}
\begin{definition}
    Пусть
    \[
        X \subseteq \mathbb{R}^n,
        A \in X,
        \begin{cases}
            x_1 = \psi_1(s) \\
            \vdots \\
            x_n = \psi_n(s)
        \end{cases},
    \]
    \[
        s \in (-\epsilon, \epsilon)
        \text{ -- какая-нибудь кривая, такая что }
        (\psi_1(s), ..., \psi_n(s)) \in X
        \> \forall s \in (-\epsilon, \epsilon)
        \text{ и }
        (\psi_1(0), ..., \psi_n(0)) = A.
    \]
    Тогда вектор $\left(
        \frac{d \psi_1}{ds}(0),
        ...,
        \frac{d \psi_n}{ds}(0)
    \right)$ называется касательным вектором к $X$ в точке $A$.
\end{definition}

\subsection{Как устроено множество всех касательных векторов к гладкому подмногообразию в фиксированной точке?}
\begin{statement}
    Пусть $X$ -- гладкое $k$-мерное многообразие и $x_i = \varphi_i(t_1, ..., t_k)$ -- гладкие координаты в окрестности $x^{(0)} = \Phi(t^{(0)})$. Тогда множество касательных векторов в точке $x^{(0)}$ образует $k$-мерное векторное пространство, линейно порожденное следующими векторами:
    \[
        \left(
            \frac{\partial \varphi_1}{\partial t_1} (t^{(0)}),
            ...,
            \frac{\partial \varphi_1n}{\partial t_1} (t^{(0)}),
        \right),
        \> ..., \>
        \left(
            \frac{\partial \varphi_1}{\partial t_k} (t^{(0)}),
            ...,
            \frac{\partial \varphi_1n}{\partial t_k} (t^{(0)}),
        \right)
    \]
\end{statement}