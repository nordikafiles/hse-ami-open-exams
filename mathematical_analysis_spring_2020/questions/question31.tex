\section{Билет 31.}

\subsection{Объясните, что такое дифференциальная форма ранга $k$, и как вычисляется интеграл (2-го рода) от $k$-формы $\omega$ по $k$-мерному многообразию $\Omega \subset \mathbb{R}^n$.}
\begin{definition}
    \textit{Дифференциальной формой} ранга $k$ (или дифференциальной $k$-формой) на $M \subseteq \mathbb{R}^n$ называется выражение вида $\sum\limits_{\{i_1, \ldots, i_k\} \subseteq \{1, \ldots, n\}} f_{i_1 \ldots i_k}(x)\diff x_{i_1} \land \ldots \land \diff x_{i_k}$, где $f_{i_1\ldots i_k}$ ---~ некоторые дифференцируемые\footnote{Часто ограничиваются гладкими функциями.} функции $f_{i_1\ldots i_k}:M \to \mathbb{R}$.
\end{definition}

Если вам очень понравился предыдущий билет, можно сказать, что это сумма грассмановых мономов степени $k$ от переменных $\diff x_1, \ldots \diff x_n$ с дифференцируемыми функциями в качестве коэффициентов. Можно считать, что среди чисел $i_1, \ldots, i_k$ нет повторений, так как мономы с повторениями всё равно зануляются.

Пусть имеются $k$-мерное многообразие $\Omega \subseteq \mathbb{R}^n$ с параметризацией $\varphi: M \to \Omega, M \subseteq \mathbb{R}^k$ и дифференциальная $k$-форма $\omega = \sum\limits_{\{i_1, \ldots, i_k\} \subseteq \{1, \ldots, n\}} f_{i_1 \ldots i_k}(x)\diff x_{i_1} \land \ldots \land \diff x_{i_k}$ на $\Omega$. Определим интеграл (2-го рода) $\omega$ по $\Omega$.

\[ \int\limits_\Omega \omega := \int\limits_M \sum_{\{i_1, \ldots, i_k\} \subseteq \{1, \ldots, n\}} f_{i_1 \ldots i_k} (\varphi(t)) \diff \varphi_{i_1} \land \ldots \land \diff \varphi_{i_k}; \]

Поясним, что творится в этой формуле. Во-первых, $\varphi_i : M \to \mathbb{R}$ ---~ это функция, соответствующая $i$-й координате $\varphi$. Во-вторых, $\diff \varphi_i$ ---~ это привычный дифференциал функции нескольких переменных, но теперь мы говорим, что это линейная комбинация грассмановых переменных $\diff t_1, \ldots, \diff t_k$. Когда мы грассманово перемножим эти дифференциалы, у нас останется выражение вида $f(t) \diff t_1 \land \ldots  \land \diff t_k$. Это так, ведь в любом слагаемом результата будут перемножаться $k$ переменных, одинаковые занулятся, останутся только слагаемые с различными, возможно, не в том порядке. Но мы можем привести порядок к правильному. После этих преобразований мы считаем интеграл как обычный кратный интеграл.
\[\int\limits_M f \diff t_1 \land \ldots \land \diff t_k = \int\limits_M f \diff t_1 \ldots \diff t_k;  \]  

\subsection{Запишите вычислительную формулу для поверхностного интеграла 2-го рода.}
Для случая $k = 2$ это всё можно записать в следующую формулу.

\[\iint\limits_\Omega P\diff y \land \diff z + Q \diff z \land \diff x + R \diff x \land \diff y = \iint\limits_M \begin{vmatrix} P & Q & R \\ \frac{\partial \varphi_1}{\partial u} & \frac{\partial \varphi_2}{\partial u} & \frac{\partial \varphi_3}{\partial u} \\ \frac{\partial \varphi_1}{\partial v} & \frac{\partial \varphi_2}{\partial v} & \frac{\partial \varphi_3}{\partial v} \end{vmatrix} \diff u \diff v;\]

Обратите внимание, при $Q$ стоит $\diff z \land \diff x$, а не $\diff x \land \diff z$. 