\section{Билет 9.}

\subsection{Дайте определение преобразования Фурье для функции $f : \mathbb{R} \to \mathbb{C}$, приведите пример вычисления преобразования Фурье.}
\begin{definition}
    Пусть $f : \mathbb{R} \to \mathbb{C}$. Если выражение
    \[
        \hat{f}(y) = \int \limits_{\mathbb{R}} f(x) e^{ixy} dx = \lim_{A \to \infty} \int \limits_{-A}^{A} f(x) e^{ixy} dx
    \]
    определено для любого $y \in \mathbb{R}$, то $\hat{f}(y) : \mathbb{R} \to \mathbb{C}$ называется преобразованием Фурье от $f$.
\end{definition}
Вычислим преобразование Фурье для функции $f(x) = I_{[-1, 1]}(x)$:
\[
    \hat{f}(y) = \int \limits_{-\infty}^{\infty} I_{[-1, 1]}(x) e^{ixy} dx = 
    \int \limits_{-1}^{1} e^{ixy} dx =
    \int \limits_{-1}^{1} \cos(xy) dx + i \int \limits_{-1}^{1} \sin(xy) dx =
    \frac{2 \sin y}{y}
\]

\subsection{Пусть $f \in L_1(\mathbb{R})$. Сформулируйте основную теорему об образе преобразования Фурье.}
\begin{theorem}
    Пусть $L_1(\mathbb{R})$ -- пространство функций, абсолютно интегрируемых на $\mathbb{R}$.
    Если $f \in L_1(\mathbb{R})$, то
    \begin{enumerate}
        \item $\hat{f}$ существует
        \item $\hat{f}$ непрерывна
        \item $\lim_{y \to \infty} \hat{f} = 0$
    \end{enumerate}
\end{theorem}