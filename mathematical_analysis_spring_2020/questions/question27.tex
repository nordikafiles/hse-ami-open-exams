\section{Билет 27.}

\subsection{Сформулируйте формулу Грина и докажите ее для области $\Omega \subset \mathbb{R}^2$ вида $\Omega = [a,b] \times [c, d]$ (прямоугольник).}
\begin{statement}
    Пусть $U \subset \mathbb{R}^2$ -- связное подмножество, ограниченное кусочно гладкой кривой $\partial U$; $P(x,y), Q(x,y)$ -- функции, дифференцируемые в некоторой окрестности $U$. Тогда выполнены равенства:
    \begin{enumerate}
        \item $\int_{\partial U} P(x,y) dx = \iint_U - \frac{\partial P}{\partial y} dx dy$
        \item $\int_{\partial U} Q(x,y) dx = \iint_U \frac{\partial Q}{\partial x} dx dy$
        \item $\int_{\partial U} P dx + Q dy = \iint_U \left(
            \frac{\partial Q}{\partial x} - \frac{\partial P}{\partial y}
        \right) dx dy$ (формула Грина)
    \end{enumerate}
    \begin{proof}
        (todo)
    \end{proof}
\end{statement}