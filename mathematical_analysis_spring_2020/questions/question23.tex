\section{Билет 23.}

\subsection{Сформулируйте теорему Каруша-Куна-Таккера.}
\begin{theorem}
    Допустим, $x^{(0)}$ -- решение задачи $\begin{cases}
        f(x) \to extr \\
        g_i(x) \leqslant 0, \> i = 1,...,m
    \end{cases}$. Пусть $L(x, \lambda) = f(x) - \sum_{i=1}^{m} \lambda_i g_i(x)$ -- функция Лагранжа. Тогда в $x^{(0)}$ выполнены условия дополняющей нежесткости:
    \[
        \begin{cases}
            L'_{x_j} = 0, & j = 1, ..., n \\
            g_i(x) \leqslant 0, & i = 1, ..., m \\
            \lambda_i g_i(x) = 0, & i = 1, ..., m \\
        \end{cases}
    \]
\end{theorem}

\subsection{Объясните, в чем смысл условий дополняющей нежесткости?}
(todo)