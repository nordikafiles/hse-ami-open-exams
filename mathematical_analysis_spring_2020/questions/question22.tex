\section{Билет 22.}

\subsection{Сформулируйте достаточное условие в методе множителей Лагранжа.}
\begin{theorem}
    Пусть $X = \{
        g_i(x) = 0
        \> | \>
        i = 1, ..., m
    \}$ в точке $x^{(0)} \in X$ является гладким многообразием,
    $f : \mathbb{R}^n \to \mathbb{R}$ -- целевая функция. Допустим, в $x^{(0)}$ выполняются условия:
    \begin{enumerate}
        \item Условия Лагранжа:
        \[
            \begin{cases}
                \operatorname{grad} f(x^{(0)}) = \sum_{i = 1}^m \lambda_i^{(0)} \operatorname{grad} g_i(x^{(0)}) \\
                g_i(x^{(0)}) = 0
            \end{cases}
        \]
        \item $d^2 \left. L(x^{(0)}, \lambda^{(0)}) \right|_{T_{x^{(0)}} X} > 0 \> (< 0)$
    \end{enumerate}
    Тогда $x^{(0)}$ -- точка условного локального минимума (максимума) для $f$.
\end{theorem}

\subsection{Объясните, как на практике проверять это условие (скажем, с помощью примера).}
\[
    \begin{cases}
        2x^2 - y^2 \to extx \\
        e^{y} - \sin x - 1 = 0
    \end{cases}
    \Rightarrow
    L = 2x^2 - y^2 - \lambda (e^y - \sin x - 1)
    \Rightarrow
    \text{ условие Лагранжа: }
    \begin{cases}
        4x + \lambda \cos x = 0 \\
        -2y - \lambda e^3 = 0 \\
        e^y - \sin x - 1 = 0 \\
    \end{cases}
    \> (*)
\]
Рассмотрим решение $(x_0, y_0, \lambda_0) = (0,0,0)$
\[
    d^2 L = L''_{xx} dx^2 + 2 L''_{xy} dx dy + L''_{yy} dy^2 = (4 - \lambda \sin x) dx^2 + (-2 - \lambda e^y) dy^2
\]
\[
    \left.
        d^2 L
    \right|_{(0,0,0)} = 4dx^2 - 2dy^2
    \text{ -- знаконеопределенная квадратичная форма}
\]
\[
    dg = d(e^y - \sin x - 1) = \begin{cases}
        d0 = 0 \\
        e^y dy - \cos x dx \\
    \end{cases}
    \Rightarrow
    e^y dy - \cos x dx = 0
    \Rightarrow
    dg(0, 0) = e^0dy - \cos 0 dx = dy - dx = 0
    \Rightarrow
\]
\[
    \Rightarrow
    dy = dx
    \Rightarrow
    \left.
        d^2 L
    \right|_{T_{(x_0, y_0)} X} = 2dx^2 > 0
    \Rightarrow
    (x_0, y_0) = (0, 0)
    \text{ -- точка условного локального минимума.}
\]