\section{Билет 15.}

\subsection{Сформулируйте теорему о неявной функции.}
\begin{theorem}
    Пусть $F : \mathbb{R}^3 \to \mathbb{R}$ удовлетворяет условиям:
    \begin{enumerate}
        \item $F$ определена и непрерывна в окрестности $(x_0, y_0)$
        \item $F'_y(x_0, y_0) \neq 0$ и  $F'_y$ непрерывна в точке $(x_0, y_0)$
        \item $F(x_0, y_0) = 0$
    \end{enumerate}
    Тогда существует окрестность
    \[
        U_{\delta, \epsilon}(x_0, y_0) = \left\{
            (x,y) \> \left|
                \>
                \begin{matrix}[l]
                    x \in (x_0 - \delta, x_0 + \delta) \\
                    y \in (y_0 - \epsilon, y_0 + \epsilon) \\
                \end{matrix}
            \right.
        \right\}
    \]
    и существует непрерывная функция $f : \mathbb{R} \to \mathbb{R}$ такая, что в окрестности $U_{\delta, \epsilon}(x_0, y_0)$
    \[
        F(x,y) = 0 \Leftrightarrow y = f(x).
    \]
    Если вдобавок выполнено $F \in C^1(U_{\delta, \epsilon}(x_0, y_0))$, то
    \[
        \begin{cases}
            f \in C^1(U_{\delta}(x_0)) \\
            f'(x_0) = - \frac{F'_x(x_0, y_0)}{F'_y(x_0, y_0)}
        \end{cases}
    \]
\end{theorem}

\subsection{Допустим, кривая $X \subset \mathbb{R}^3$ задана уравнениями $f(x,y,z) = 0, g(x,y,z) = 0$, и известно, что $\operatorname{grad} f(x_0, y_0, z_0) = (2;0;0), \operatorname{grad} g(x_0, y_0, z_0) = (0;1;3)$. Какие из координат $x,y,z$ можно использовать в качестве локальных координат на $X$ в окрестности точки $(x_0, y_0, z_0)$?}
(todo) 