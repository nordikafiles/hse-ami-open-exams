\section{Билет 14.}

\subsection{Дайте определение гладкого $k$-мерного подмногообразия в $\mathbb{R}^n$ и сопутствующее определение гладких координат.}
\begin{definition}
    Пусть $\Phi : V \subseteq \mathbb{R}^k \to M$, где $\Phi$ -- совокупность всех $\varphi_i$. Координаты
    \[
        \begin{cases}
            x_1 = \varphi_1(t_1, ..., t_k) \\
            \vdots \\
            x_n = \varphi_n(t_1, ..., t_k) \\
        \end{cases}
    \]
    называются гладкими, если
    \begin{enumerate}
        \item $\varphi_1, ..., \varphi_n \in C^{1}(V)$, то есть дифференцируются, как функции многих переменных
        \item Ранг матрицы частных производных в любой точке $t = (t_1, ..., t_k) \in V$ равен $k$, то есть
        \[
            \forall t = (t_1, ..., t_k) \in V
            \quad
            \operatorname{rank} \begin{pmatrix}
                \frac{\partial \varphi_1}{\partial t_1}
                & ... &
                \frac{\partial \varphi_n}{\partial t_1} \\
                \vdots
                & \ddots &
                \vdots \\
                \frac{\partial \varphi_1}{\partial t_k}
                & ... &
                \frac{\partial \varphi_n}{\partial t_k} \\
            \end{pmatrix}
            = k
        \]
    \end{enumerate}
\end{definition}
\begin{definition}
    Подмножество $M \subseteq \mathbb{R}^n$ называется гладким $k$-мерным подмногообразием, если $\forall x \in M$ существует окрестность $U x \in U$ такая, что на $M \cap U$ можно задать гладкие координаты:
    \[
        \begin{cases}
            x_1 = \varphi_1(t_1, ..., t_k) \\
            \vdots \\
            x_n = \varphi_n(t_1, ..., t_k) \\
        \end{cases}
        \Leftrightarrow
        x \in U \cap M
    \]
\end{definition}

\subsection{Приведите пример параметрической кривой, которая параметрически задана дифференцируемыми функциями, но не является гладким 1-мерным многообразием в какой-нибудь точке.}
Пусть $t \in \mathbb{R}$ и
\[
    \begin{cases}
        x = t^3 \\
        y = t^6 \\
    \end{cases}
\]
Многообразие задано дифференцируемыми функциями, но
\[
    \left.
        \begin{pmatrix}
            \frac{\partial x}{\partial t}
            & \frac{\partial y}{\partial t}\\
        \end{pmatrix}
    \right|_{t=0} =
    \left.
        \begin{pmatrix}
            3t^2
            &
            6t^5
            \\
        \end{pmatrix}
    \right|_{t=0} =
    \begin{pmatrix}
        0 & 0\\
    \end{pmatrix}.
\]
Следовательно, данное многообразие не является гладким.