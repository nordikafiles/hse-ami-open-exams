\section{Билет 3.}

\subsection{Неравенство Бесселя (идея доказательства).}
\begin{theorem} 
    (неравенство Бесселя) $\sum_{i=1}^{\infty} \langle f, \psi_i \rangle^2 \le \| f \|^2$, где $f$ - элемент векторного пространства $V$ с ортонормированной системой $\{{\psi_i}\}$
    \begin{proof}
        $0 \le \| f- \sum_{i=1}^{n} \langle f \psi_i \rangle \|^2=\|f\|^2-\sum_{i=1}^{n} \|\langle f, \psi_i \rangle \|^2=\|f\|^2-\sum_{i=1}^{n} f_i ^2$ 
        
        $\sum_{i=1}^{n} f_i^2$ ограничена сверху $\|f\|^2$ $\Rightarrow$  сходится. Переходим к пределу, получаем требуемое. 
    \end{proof}
\end{theorem}

\subsection{Определения замкнутой и полной ортонормированных систем (ОНС).}
\begin{definition}
    Ортонормированная система $\{\psi_i\} \subset V$ называется замкнутой, если
    \[
        \forall f \in V \>
        \forall \epsilon > 0 \>
        \exists \> n \in \mathbb{N}, c_i \in \mathbb{R} \>
        \left\lVert
            f - \sum_{k=1}^n c_k \psi_k
        \right\rVert < \epsilon
    \]
\end{definition}
\begin{definition}
    Ортонормированная система $\{\psi_i\} \subset V$  называется полной, если для любой $f \in V$ из равенств  $(f, \psi_k) = 0$ следует, что $f$ -- нулевой элемент в $V$.
\end{definition}

\subsection{Тождество Парсеваля для замкнутой ОНС.}
\begin{theorem}
    Неравенство Бесселя для замкнутой ОНС обращяется в равенство Парсеваля:
    \[
        \sum_{k} f_k^2 = ||f||^2
    \]
    \begin{proof}
        Так как ортогональная проекция дает наилучшее приближение, имеем
        \[
            ||f||^2 - \sum_{k=1}^n f_k^2 = ||f - \sum_{k=1}^n f_k \psi_k||^2 \leqslant ||f - \sum_{k=1}^n c_k \psi_k|| < \epsilon \text{ (в силу замкнутости) }
            \Rightarrow
            \sum_{k=1}^n f_k^2 \to ||f||^2
        \]
    \end{proof}
\end{theorem}