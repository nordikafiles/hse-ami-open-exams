\section{Билет 32.}

\subsection{Что такое ориентация $k$-мерного многообразия? Как изменится интеграл 2-го рода от дифференциальной формы при смене ориентации многообразия (б. д.)?}

Пусть $\Omega \subseteq \mathbb{R}^n$ ---~ $k$-мерное связное\footnote{Напомним, многообразие называется гладким, если любые две его точки можно соединить проходяще по нему непрерывной кривой.} многообразие, и у него имеются две параметризации $\varphi: M \to \Omega, \psi: N \to \Omega; M, N \subseteq \mathbb{R}^k$. Предположим, что функция замены координат $c = \varphi^{-1} \circ \psi$ биективна и непрерывно дифференцируема.

\[ \begin{tikzcd}
    M \arrow[swap]{rd}{\varphi} &  &\arrow[swap]{ll}{c} N \arrow{ld}{\psi} \\
    & \Omega
\end{tikzcd} \]

Посмотрим на якобиан $J(c)$. Если бы где-то он был равен нулю, в окрестности этой точки $c$ была бы необратима. Значит он не равен нулю нигде. Поскольку $J(c)$ непрерывен и $\Omega$ связно, из этого следует, что он имеет постоянный знак. Тогда если он положителен, будем говорить, что $\varphi$ и $\psi$ задают одну и ту же ориентацию, а если отрицателен ---~ то разные. Таким образом мы определяем ориентацию как отношение эквивалентности с двумя классами на параметризациях многообразия.

Ориентация задаёт ориентацию на любом касательном пространстве $T_{x}\Omega$ как на векторном пространстве. Если мы назвали ориентацию некоторой параметризации положительной, то назовём положительным базис $T_x \Omega$, полученный из её производных.

При смене параметризации на имеющую противоположную ориентацию интеграл 2-го рода меняет знак. 