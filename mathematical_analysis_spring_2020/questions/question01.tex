\section{Билет 1.}

\subsection{Пространство кусочно непрерывных функций на отрезке как пример евклидова пространства.}
\begin{definition}
    Функция $f : \mathbb{R} \to \mathbb{R}$ называется \textbf{кусочно непрерывной} на отрезке, если она непрерывна во всех точках этого отрезка, за исключением конечного числа точек, где она имеет разрывы 1-го рода.
\end{definition}
\begin{definition}
    Множество $V = \hat{C}([a,b])$ называется пространством кусочно непрерыных функций на $[a,b]$, если $\forall f \in V : f$ является кусочно непрерывной функцией и для операции скалярного произведения
    $(f, g) = \int \limits_a^b f(x) g(x) dx$ выполняются следующие свойства:
    \begin{enumerate}
        \item $(f, g) = (g, f)$.
        \item $(f, f) \geqslant 0$ и $(f, f) \Rightarrow f(x) = 0$ на $[a,b]$, исключая, быть может, конечное число точек $x$.
        \item $(\alpha f + \beta g, \psi) = \alpha (f, \psi) + \beta (g, \psi)$, где $\alpha, \beta \in \mathbb{R}$.
    \end{enumerate}
\end{definition}

\subsection{Неравенство Коши-Буняковского в этом пространстве (б.д.).}
\[
    (f,g)^2 \leqslant (x,x) \cdot (y,y)
\]

\subsection{Ортогональные и ортонормированные системы в евклидовом пространстве.}
\begin{definition}
    Множество ${x_i} \subset L$ называется ортогональной системой, если элементы этого множества попарно ортогональны, то есть $\forall i, j \> (x_i, x_j) = 0$.
\end{definition}

\begin{definition}
    Ортогональная система ${x_i} \subset L$ называется ортонормированной системой, если норма каждого элемента равна 1, то есть $\forall i \> (x_i, x_i) = 1$.
\end{definition}


\subsection{Главный пример: тригонометрическая система функций в $\hat{C}([-\pi, \pi])$.}
\begin{definition}
    Множество $L = \{\frac{1}{\sqrt{2\pi}}, \frac{1}{\sqrt{\pi}} \cos x, \frac{1}{\sqrt{\pi}} \sin x, \frac{1}{\sqrt{\pi}} \cos 2x, \frac{1}{\sqrt{\pi}} \sin 2x, ..., \frac{1}{\sqrt{\pi}} \cos nx, \frac{1}{\sqrt{\pi}} \sin nx \} \subset \hat{C}([-\pi,\pi])$ называется основной тригонометрической системой функций.
\end{definition}

\begin{statement}
    L -- ортогональная система.
    \begin{proof}
        Пусть $k, l$ -- произвольные натуральные числа. Тогда
        \[
            \begin{matrix}
                \int \limits_{-\pi}^{\pi} \frac{1}{\sqrt{\pi}} sin kx \frac{1}{\sqrt{\pi}} \cos lx dx
                & = &
                \frac{1}{2\pi} \int \limits_{-\pi}^{\pi} (\sin(x(k - l)) + \sin(x(k + l))) dx
                & = &
                0 \\
                \int \limits_{-\pi}^{\pi} \frac{1}{\sqrt{\pi}} sin kx \frac{1}{\sqrt{\pi}} \sin lx dx
                & = &
                \frac{1}{2\pi} \int \limits_{-\pi}^{\pi} (\cos(x(k - l)) - \cos(x(k + l))) dx
                & = &
                0 \\
                \int \limits_{-\pi}^{\pi} \frac{1}{\sqrt{\pi}} cos kx \frac{1}{\sqrt{\pi}} \cos lx dx
                & = &
                \frac{1}{2\pi} \int \limits_{-\pi}^{\pi} (\cos(x(k - l)) + \cos(x(k + l))) dx
                & = &
                0 \\  
                & & \int \limits_{-\pi}^{\pi} \frac{1}{\sqrt{\pi}} \cos kx \frac{1}{\sqrt{2\pi}} dx
                & = & 0 \\
                & & \int \limits_{-\pi}^{\pi} \frac{1}{\sqrt{\pi}} \sin kx \frac{1}{\sqrt{2\pi}} dx
                & = & 0 \\
            \end{matrix}
        \]
    \end{proof}
\end{statement}
\begin{statement}
    L -- ортогональная система.
    \begin{proof}
        Пусть $k, l$ -- произвольные натуральные числа. Найдем норму каждого элемента множества:
        \[
            \begin{matrix}
                (\frac{1}{\sqrt{2\pi}}, \frac{1}{\sqrt{2\pi}})
                & = &
                \frac{1}{2\pi} \int \limits_{-\pi}^{\pi} dx
                & = &
                1 \\

                (\frac{1}{\sqrt{\pi}} \sin kx, \frac{1}{\sqrt{\pi}} \sin kx)
                & = &
                \frac{1}{\pi} \int \limits_{-\pi}^{\pi} \sin^2 kx dx
                & = &
                1 \\

                (\frac{1}{\sqrt{\pi}} \cos kx, \frac{1}{\sqrt{\pi}} \cos kx)
                & = &
                \frac{1}{\pi} \int \limits_{-\pi}^{\pi} \cos^2 kx dx
                & = &
                1 \\
            \end{matrix}
        \]
    \end{proof}
\end{statement}