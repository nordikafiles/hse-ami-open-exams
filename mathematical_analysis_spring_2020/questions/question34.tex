\section{Билет 34.}

\subsection{Выведите из общей формулы Стокса частные случаи: формулу Ньютона-Лейбница, формулу Грина, формулу Гаусса-Остроградского.}

\begin{enumerate}

	\item \textbf{Формула Ньютона-Лейбница, $n = k = 1$} \newline

	Пусть наше многообразие это отрезок на прямой $[a; b]$. Его границей будет множество из двух точек $\{a, b\}$. Заметим, что точка ---~ это нульмерное многообразие и по нему можно интегрировать 0-формы (то есть просто функции). При чём этот интеграл будет с точностью до знака (знак как всегда определяется ориентацией) равен значению функции в точке. Если на отрезке мы берём стандартную ориентацию <<слева направо>>, то для границы это будет означать взятие $b$ с плюсом и $a$ с минусом. Итак, формула Стокса принимает следующий вид:
	
	\[F(b) - F(a) = \int\limits_a^b \diff F; \]
	
	Перепишем в более привычную запись.

	\[\int\limits_a^b F'(x) \diff x = F(b) - F(a); \]

	\item \textbf{Формула Грина, $n = k = 2$} \newline

	Дифференциальная 1-форма в $\mathbb{R}^2$ имеет вид $P\diff x + Q \diff y$. Посчитаем её дифференциал

	\[\diff (P \diff x + Q \diff y) = \left(\frac{\partial P}{\partial x}\diff x + \frac{\partial P}{\partial y}\diff y \right) \land \diff x +\left(\frac{\partial Q}{\partial x}\diff x + \frac{\partial Q}{\partial y}\diff y \right) \land \diff y = \frac{\partial P}{\partial y} \diff y \land \diff x + \frac{\partial Q}{\partial x} \diff x \land \diff y = \] \[ = \left(\frac{\partial Q}{\partial x} - \frac{\partial P}{\partial y}\right) \diff x \land \diff y;  \]

	Формула Стокса принимает следующий вид:

	\[\int\limits_{\partial U} P\diff x + Q\diff y = \iint\limits_{U} \left(\frac{\partial Q}{\partial x} - \frac{\partial P}{\partial y}\right)\diff x \diff y; \]

	Здесь условие согласованности ориентации можно сформулировать как <<при обходе $\partial U$ по заданной параметризации $U$ всегда находится слева>>.

	\item \textbf{Формула Гаусса-Остроградского, $n = k = 3$} \newline

	Дифференциальная 2-форма в $\mathbb{R}^3$ имеет вид $P\diff y \land \diff x + Q \diff z \land \diff x + R \diff x \land \diff y$. Посчитаем её дифференциал.

	\[\diff (P\diff y \land \diff z + Q \diff z \land \diff x + R \diff x \land \diff y) = \frac{\partial P}{\partial x} \diff x \land \diff y \land \diff z + \frac{\partial Q}{\partial y}\diff y \land \diff z \land \diff x + \frac{\partial R}{\partial z} \diff z \land \diff x \land \diff y  = \] \[ =\left( \frac{\partial P}{\partial x} + \frac{\partial Q}{\partial y} + \frac{\partial R}{\partial z} \right) \diff x \land \diff y \land \diff z; \]

	Формула Стокса принимает следующий вид:

	\[\iint\limits_{\partial V} P\diff y \land \diff x + Q \diff z \land \diff x + R \diff x \land \diff y = \iiint\limits_{V}\left( \frac{\partial P}{\partial x} + \frac{\partial Q}{\partial y} + \frac{\partial R}{\partial z} \right) \diff x\diff y\diff z; \]
\end{enumerate}
