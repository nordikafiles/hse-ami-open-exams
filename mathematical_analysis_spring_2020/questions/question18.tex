\section{Билет 18.}

\subsection{Допустим, что все точки множества $X \in \mathbb{R}^n$ удовлетворяют уравнению $f(x) = 0$. Докажите, что в любой точке $x^{(0)} \in X$ любой касательный вектор к $X$ перпендикулярен градиенту $\operatorname{grad} f(x^{(0)})$.}
\begin{proof}
    Пусть
    \[
        \forall x^{(0)} \subset \mathbb{R}^n
        \>
        f(x^{(0)}) = 0
        \Rightarrow
        \forall \text{ кривой }
        \{
            x_i = \varphi_i(s)
        \} \subset X
        \>
        f(\varphi_1(s), .., \varphi_n(s)) = 0
    \]
    Продифференцируем по $S$:
    \[
        \frac{\partial f}{ \partial x_1} \cdot \frac{\partial \varphi_1}{\partial S} +
        ... +
        \frac{\partial f}{ \partial x_n} \cdot \frac{\partial \varphi_n}{\partial S}
        = 0,
        \text{ то есть }
        \left<
            \operatorname{grad} f(x^{(0)}),
            \text{ касательный вектор к $X$ в точке $x^{(0)}$ }
        \right> = 0
        \Leftrightarrow
    \]
    \[
        \Leftrightarrow
        \text{ касательный вектор к $X$ в точке $x^{(0)}$ перпендикулярен }
        \operatorname{grad} f(x^{(0)})
    \]
\end{proof}

\subsection{Опишите касательное пространство к $k$-мерному подмногообразию в $\mathbb{R}^n$, заданному системой неявных уранвнений (б.д.).}
Пусть $m \leqslant n$ и $X$ задано
\[
    \begin{cases}
        f_1(x) = 0 \\
        \vdots \\
        f_m(x) = 0 \\
    \end{cases},
\]
и в $x^{(0)} \in X$
\[
    \operatorname{grad} f_1,
    ...,
    \operatorname{grad} f_m,
    \text{ линейно независимы.}
\]
Тогда
\[
    T_{x^{(0)}} X = \left<
        \operatorname{grad} f_1,
        ...,
        \operatorname{grad} f_m,
    \right>^{\perp}
    \text{ (ортогональное дополнение)}.
\]