\section{Билет 19.}

\subsection{Необходимое и достаточное условия локального экстремума для функции нескольких переменных (б.д.).}
\begin{theorem}[необходимое условие локального экстремума для функции нескольких переменных]
    Если $f$ дифференцируема в точке $x^{(0)}$, и $x^{(0)}$ -- точка локального экстремума, то $\forall i = 1, ..., n \> \frac{\partial f}{\partial x_i}(x^{(0)}) = 0$.
\end{theorem}
\begin{theorem}[достаточное условие локального экстремума для функции нескольких переменных]
    Пусть $f$ дважды дифференцируема в окрестности точки $x^{(0)}$. Допустим $d f(x^{(0)}) = 0, d^2 f(x^{(0)}) > 0 \> (< 0).$ Тогда $x^{(0)}$ -- точка локального минимума (максимума).
\end{theorem}