\section{Билет 2.}

\subsection{Задача о наилучшем приближении элемента евклидова пространства элементом конечномерного подпространства (б.д.).}
Пусть $X$ -- евклидово пространство.
\begin{definition}
    Для любого $x \in X$ величина $E_Y(x) = \inf_{y \in Y} ||x - y||$ называется наилучшим приближением точки $x$ элементами линейного множества $Y$. Если при этом существует $y^* \in Y$ такой, что $E_Y(x) = ||x - y^*||$, то этот $y^*$ называется элементом наилучшего приближения точки $x$.
\end{definition}
\begin{theorem}
    Для любого $x \in X$ существует элемент наилучшего приближения.
\end{theorem}

\subsection{Ряд Фурье по произвольной ортонормированной системе.}
\begin{definition}
    Назовем рядом Фурье элемента $f$ по ортонормированной системе $\{\psi_k\}$ ряд вида
    \[
        \sum_{k=1}^{\infty} f_k \psi_k,
    \]
    в котором через $f_k$ обозначены постоянные числа, называемые коэффициентами Фурье элемента $f$ и определяемые равенствами $f_k = (f, \psi_k)$.
\end{definition}
\subsection{Ряд Фурье по тригонометрической системе.}
Ряд Фурье функции $f$ по тригонометрической системе $$V = \{\frac{1}{\sqrt{2\pi}}, \frac{1}{\sqrt{\pi}} \cos x, \frac{1}{\sqrt{\pi}} \sin x, \frac{1}{\sqrt{\pi}} \cos 2x, \frac{1}{\sqrt{\pi}} \sin 2x, ..., \frac{1}{\sqrt{\pi}} \cos nx, \frac{1}{\sqrt{\pi}} \sin nx \}$$ имеет вид:
\[
    f = a_0\frac{1}{\sqrt{2\pi}} +  \sum_{n=1}^{\infty} \left(
        a_n \frac{\cos nx}{\sqrt{\pi}} + b_n \frac{\sin nx}{\sqrt{\pi}}
    \right),
\]
где
\[
    a_0 = \left(
        f, \frac{1}{\sqrt{2 \pi}}
    \right),
    \quad
    a_n = \left(
        f, \frac{\cos nx}{\sqrt{\pi}}
    \right),
    \quad
    b_n = \left(
        f, \frac{\sin nx}{\sqrt{\pi}}
    \right).
\]