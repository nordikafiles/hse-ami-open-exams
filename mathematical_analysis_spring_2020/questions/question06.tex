\section{Билет 6.}

\subsection{Дайте определение ядра Дирихле и ядра Фейера.}
\begin{definition}
    Ядро Дирихле -- $2\pi$-периодическая функция, задаваемая формулой
    \[
        D_n(x) = \frac{\sin \left((n + \frac{1}{2})x\right) }{2\pi \sin (x/2)}
    \]
\end{definition}

\subsection{Какой смысл у свертки произвольной периодической функции с этими ядрами? (б.д.)}
\begin{statement}
    Пусть $f(x)$ -- интегрируемая на $[-\pi, \pi]$ $2\pi$-периодическая функция, тогда для любых $x \in \mathbb{R}, n \in \mathbb{N}$ выполняется следующее равенство
    \[
        S_n(f; x) = \frac{a_0}{2} + \sum (a_n \cos nx + b_n \sin nx) = \int \limits_{-\pi}^{\pi} f(x + u) D_n(u) du = (f * D_n)(x)
    \]
\end{statement}