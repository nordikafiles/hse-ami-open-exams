\section{Билет 7.}

\subsection{Сформулируйте (б.д.) теоремы о приближении $2\pi$-периодической функции тригонометрическими многочленами: о сходимости ряда Фурье в точке, и о приближении функции тригонометрическими многочленами в различных функциональных метриках.}

\begin{theorem} 
    Если $2\pi$-периодическая функция имеет в точке производную слева и справа, то ряд Фурье в ней сходится к среднему арифметическому этих производных.
\end{theorem}

Заметим, что в условиях данной теоремы функция может быть разрывной.\\

Еще раз напомним, что $\sigma_n(x,f)=\frac{S_0(x, f)+\cdots+S_{n-1}(x, f)}{n}$

\begin{theorem} 
    Пусть $f$ - непрерывная $2\pi$-периодическая функция. Тогда $\sigma_n \rightarrow f$ на $[-\pi, \pi]$ в смысле метрики $d_{\infty}$
\end{theorem}

\begin{theorem} 
    Пусть $f$ - кусочно-непрерывная. Тогда ее можно приблизить тригонометрическим многочленом в смысле метрики $d_2$.
\end{theorem}