\section{Билет 16.}

\subsection{Сформулируйте общую теорему о неявном отображении.}
\begin{theorem}
    Пусть
    \begin{enumerate}
        \item $F_1(x,y), ..., F_m(x,y)$ непрерывно дифференцируемы в окрестности точки $(x^{(0)}, y^{(0)})$
        \item $F_j(x^{(0)}, y^{(0)})=0 \> \forall j = 1, ..., m$
        \item $J = \left.
            \det \frac{D(F_1, ..., F_m)}{D(y1, ..., y_m)} 
        \right|_{(x_0, y_0)} = \left.
            \det \begin{pmatrix}
                \operatorname{grad} F_1 \\
                \vdots \\
                \operatorname{grad} F_m \\
            \end{pmatrix}
        \right|_{(x_0, y_0)} \neq 0$
    \end{enumerate}
    Тогда существует прямоугольная окрестность $U_\delta(x^{(0)}) \times U_\epsilon(y^{(0)})$ и набор дифференцируемых функций $f_1(x_1, ..., x_n), ..., f_m(x_1, ..., x_n)$ таких, что 
    \[
        \{
            F_j(x,y) = 0
        \}_{j=1}^m
        \Leftrightarrow
        \{
            y_j = f_j(x)
        \}_{j=1}^m,
    \]
    причем $f_j(x^{(0)}) = y_j^{(0)}$.
\end{theorem}

\subsection{Допустим, что все точки множества $x \subset \mathbb{R}^n$ удовлетворяют уравнению $f(x) = 0$. Докажите, что в любой точке $x^{(0)} \in X$ любой касательный вектор к $X$ перпендикулярен градиенту $\operatorname{grad} f(x^{(0)})$. Опишите касательное пространство к $k$-мерному подмногообразию в $\mathbb{R}^n$, заданному системой неявных уравнений (б.д.).}
Запишем матрицу
\begin{gather*}
\begin{pmatrix}
    2 & 0 & 0\\
    0 & 1 & 3
\end{pmatrix}
\end{gather*}
Видим, что линейно независимы первый и второй столбец, и первый и третьей. Значит координатой может быть $z$ или $y$. Обратите внимание, если матрица производных по $x$ и $y$ невырождена, то подходит как координата $z$.