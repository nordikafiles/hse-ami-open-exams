\section{Дизъюнкты, универсальные дизъюнкты. Исчисление резолюций (ИР) для доказательства несовместности множеств универсальных дизъюнктов. Теорема корректности ИР.}

\subsection{Дизъюнкты, универсальные дизъюнкты.}

\begin{definition}
  Дизъюнктом называется дизъюнкция атомарных формул и их отрицаний.
\end{definition}

\begin{definition}
  Универсальным дизъюнктом называется формлуа, полученная из дизъюнкта приписыванием кванторов всеобщности.
\end{definition}

\subsection{Исчисление резолюций (ИР) для доказательства несовместности множеств универсальных дизъюнктов.}

Для доказательства несовместности множества универсальных дизъюнктов используется метод резолюций, состоящий из двух правил:
\begin{enumerate}
  \item Если $A,B$ -- дизъюнкты, $p$ -- атомарная формула. $\frac{A \vee p \quad B \vee \neg p}{A \vee B}$;
  \item $\frac{\forall x \> A(x)}{A(t)}$, где $t$ -- терм.
\end{enumerate}

\begin{definition}
  $\frac{p \quad \neg p}{\bot}$. $\bot$ -- пустой дизъюнкт.
\end{definition}

\subsection{Теорема корректности ИР.}

\begin{theorem}[Теорема корректности ИР]
  Если из набора универсальных дизъюнктов $\{D_1, \dots, D_n\}$ можно вывести пустой дизъюнкт в модели $M$, то этот набор несовместен (т.е. он не имеет модели).
  \begin{proof}
    Заметим, что оба правила исчисления резолюций сохраняют истинность. \newline
    Действительно, если в правиле резолюций $A \vee p$ и $B \vee p$ были истинны, то $A \vee B$ истинно. \newline
    Если $D(x)$ было истинно для любого $x$, то $D(t)$ истинно для какого терма $t$. \newline
    Если мы вывели пустой дизъюнкт, то по истинности правил исчисления резолюций получаем, что пустой дизъюнкт является истинным. Противоречие.
  \end{proof}
\end{theorem}