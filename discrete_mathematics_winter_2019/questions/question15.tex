\section{Исчисление резолюций для опровержения пропозициональных формул в КНФ: дизъюнкты, правило резолюций, опровержение КНФ в исчислении резолюций. Теорема корректности исчисления резолюций (для пропозициональных формул в КНФ).}

\subsection{Исчисление резолюций для опровержения пропозициональных формул в КНФ: дизъюнкты, правило резолюций, опровержение КНФ в исчислении резолюций.}

\begin{definition}
  Литерал -- переменная или отрицание переменной.
\end{definition}

\begin{definition}
  Дизъюнкт -- это дизъюнкция по некоторому конечному множеству литералов. 
\end{definition}

Обратите внимание, что в этом определении речь про множество. Хотя мы записываем дизъюнкты как формулы
$\lambda_1, \dots, \lambda_n$, мы считаем, что, к примеру, $\lambda_1 \vee \lambda_2$, $\lambda_2 \vee \lambda_1$, $\lambda_1 \vee \lambda_2 \vee \lambda_1$ — это всё один и тот же дизъюнкт. \newline

\begin{definition}
  У исчисления резолюций нет аксиом и есть одно правило — правило резолюции:
\[
  \frac{A \vee p, \quad B \vee \neg p}{A \vee B}
\]
\end{definition}

\begin{definition}
  При применении правила к $p$ и $\neg p$ результатом будет пустой дизъюнкт. Обозначение: $\bot$.
\end{definition}

\begin{definition}
  На записанные в КНФ пропозиональные формулы можно смотреть как на множества дизъюнктов в исчислении резолюций. Будем говорить, что множество дизъюнктов совместно, если есть набор значений переменных, при котором каждый дизъюнкт возвращает истину.
\end{definition}

\subsection{Теорема корректности исчисления резолюций.}

\begin{theorem}[Теорема корректности исчисления резолюций]
  Если из множества дизъюнктов можно вывести пустой дизъюнкт, то оно несовместно.
  \begin{proof}
    Пусть $S = (A \vee p) \wedge (B \vee \neg p)$, $S' = S \wedge (A \vee B)$.
    Рассмотрим таблицу истинности: \newline
    \begin{center}
      \begin{tabular}{|ccc|c|c|}
        \hline
        $A$ & $B$ & $p$ & $(A \vee p) \wedge (B \vee \neg p)$ & $(A \vee p) \wedge (B \vee \neg p) \wedge (A \vee B)$ \\
        \hline
        $0$ & $0$ & $0$ & $0$ & $0$ \\
        $0$ & $0$ & $1$ & $0$ & $0$ \\
        $0$ & $1$ & $0$ & $0$ & $0$ \\
        $0$ & $1$ & $1$ & $1$ & $1$ \\
        $1$ & $0$ & $0$ & $1$ & $1$ \\
        $1$ & $0$ & $1$ & $0$ & $0$ \\
        $1$ & $1$ & $0$ & $1$ & $1$ \\
        $1$ & $1$ & $1$ & $1$ & $1$ \\
        \hline
      \end{tabular}
    \end{center}
    Откуда понятно, $S = S'$. Очевидно, что $p \wedge \neg p$ несовместно.
  \end{proof}
\end{theorem}