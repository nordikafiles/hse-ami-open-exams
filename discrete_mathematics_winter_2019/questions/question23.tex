\section{Гомоморфизмы, эпиморфизмы (сюръективные гомоморфизмы), изоморфизмы. Теорема о сохранении истинности при эпиморфизме. Изоморфные модели. Элементарно эквивалентные модели, элементарная эквивалентность изоморфных моделей.}

\subsection{Гомоморфизмы, эпиморфизмы (сюръективные гомоморфизмы), изоморфизмы.}
Пусть $\Gamma$ -- сигнатура, $M_1, M_2$ -- интерпретации $\Gamma$.

\begin{definition}
  Гомоморфизмом $h : M_1 \to M_2$ называется отображение, сохраняющее все предикаты и формулы в сигнатуре:
  \begin{itemize}
    \item для предиката $P^n \in \Gamma$ его интерпретации $P_1 \in M_1$, $P_2 \in M_2$ $\forall a_1, \dots, a_n (P_1(a_1, \dots, a_n) = P_2(h(a_1), \dots, h(a_n)))$;
    \item для формулы $f^n \in \Gamma$ его интерпретации $f_1 \in M_1$, $f_2 \in M_2$ $\forall a_1, \dots, a_n (h(f_1(a_1, \dots, a_n)) = f_2(h(a_1), \dots, h(a_n))$.
   \end{itemize}
\end{definition}

\begin{definition}
  Эпиморфизмом называется сюръективный гомоморфизм.
\end{definition}

\begin{definition}
  Изоморфизмом называется биективный гомоморфизм.
\end{definition}

\subsection{Теорема о сохранении истинности при эпиморфизме.}

\begin{statement}
  Пусть $h : M_1 \to M_2$ -- эпиморфизм. Тогда $\forall a_1, \dots, a_n \in M_1$, $M_1 \vDash A(a_1, \dots, a_n)$ $\Leftrightarrow$ $M_2 \vDash A(h(a_1), \dots, h(a_n))$.

\end{statement}

\subsection{Изоморфные модели.}

\begin{definition}
  Модели $M_1$ и $M_2$ называются 
\end{definition}