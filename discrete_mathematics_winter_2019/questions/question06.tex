\section{Теорема Поста. Существование перечислимого множества, дополнение которого неперечислимо. Перечислимые неотделимые множества.}

\subsection{Теорема Поста}

\begin{theorem}[Теорема Поста]
  Множества $A$ и $\overline{A}$ перечислимы тогда и только тогда, когда $A$ разрешимо.
  \begin{proof}
    $\Rightarrow$ \newline
    Построим алгоритм, вычисляющий $\bigchi_A(x)$: будем по очереди делать по одному шагу для $w_A(x)$ и $w_{\overline{A}}(x)$, т.к. $x \in A \vee x \in \overline{A}$, то какой-то один из алгоритмов вернет $1$ на каком-то шаге. Если это будет $w_A$, то вернем $1$, если же $w_{\overline{A}}$, то вернем $0$. \newline
    $\Leftarrow$ \newline
    Очевидно, из разрешимости следует перечислимость. Если $A$ разрешимо, то и $\overline{A}$ разрешимо.
  \end{proof}
\end{theorem}

\subsection{Существование перечислимого множества, дополнение которого неперечислимо.}

\begin{statement}
  Существует перечислимое множество, дополнение которого перечислимо.
  \begin{proof}
    Множество $K = \{x \> | \> U_x(x) \> \textit{определено}\}$ перечислимо, но не разрешимо. Тогда $\overline{K}$ неперечислимо. Если $\overline{K}$ было бы перечислимо, то по теореме Поста, оно было бы разрешимо.
  \end{proof}
\end{statement}


\subsection{Перечислимые неотделимые множества.}

\begin{definition}
  Множества $A, B$ называются отделимыми, если существует множество $C$, т.ч. $A \subseteq C$ и $B \cap C = \emptyset$.
\end{definition}

\begin{statement}
  Существуют непересекающиеся перечислимые множества, которые нельзя отделить разрешимом множеством.
  \begin{proof}
    Рассмотрим множества $A = \{x \> | \> U_x(x) = 42\}$, $B = \{x \> | \> U_x(x) = 0\}$. Они перечислимы и не пересекаются. 
    Допустим, существует разрешимое $C$, отделяющее $A$ и $B$. Пусть оно содержит в себе
        множество $A$. Тогда:

        $$
            \chi_C(x) =
            \begin{cases}
                1, & \text{если $U_x(x) = 42$,}\\
                0, & \text{если $U_x(x) = 0$,}\\
                \text{и что-то ещё на других числах.}
            \end{cases}
        $$

        $\chi_C$ вычислима. Значит, $\exists n \ \chi_C(x) = U_n(x)$ Но пусть тогда $x = n$.
        Получаем:

        $$
            U_n(n) =
            \begin{cases}
                1, & \text{если $U_n(n) = 42$,}\\
                0, & \text{если $U_n(n) = 0$,}\\
                \text{и что-то ещё на других числах.}
            \end{cases}
        $$

        Получили противоречие.
  \end{proof}
\end{statement}