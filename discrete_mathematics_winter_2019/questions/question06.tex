\section{Теорема Поста. Существование перечислимого множества, дополнение которого перечислимо. Перечислимые неотделимые множества.}

\subsection{Теорема Поста}

\begin{theorem}[Теорема Поста]
  Множества $A$ и $\overline{A}$ перечислимы тогда и только тогда, когда $A$ разрешимо.
  \begin{proof}
    $\Rightarrow$ \newline
    Построим алгоритм, вычисляющий $\bigchi_A(x)$: будем по очереди делать по одному шагу для $w_A(x)$ и $w_{\overline{A}}(x)$, т.к. $x \in A \vee x \in \overline{A}$, то какой-то один из алгоритмов вернет $1$ на каком-то шаге. Если это будет $w_A$, то вернем $1$, если же $w_{\overline{A}}$, то вернем $0$. \newline
    $\Leftarrow$ \newline
    Очевидно, из разрешимости следует перечислимость. Если $A$ разрешимо, то и $\overline{A}$ разрешимо.
  \end{proof}
\end{theorem}

\subsection{Существование перечислимого множества, дополнение которого перечислимо.}


\subsection{Перечислимые неотделимые множества.}