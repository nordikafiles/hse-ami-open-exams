\section{Теорема Клини о неподвижной точке.}

\begin{theorem}
    Для всякой тотальной вычислимой функции $f$ и главной нумерации $U_n$ найдётся $n$
    такое, что $U_n = U_{f(n)}$.
    \begin{proof}
        Педагогический трюк для лучшего запоминания: сначала попытаемся доказать ложное утверждение
        о том, что у всякой тотальной вычислимой функции есть неподвижная точка, то есть число $n$
        такое, что $n = f(n)$.

        Функция $f$ вычислима, функция $U_x(x)$ вычислима, значит, вычислима их композиция. То
        есть, существует $m$ такое, что
        $$
            f(U_x(x)) = U_m(x).
        $$

        Подставляя $x = m$, получаем
        $$
            f(U_m(m)) = U_m(m),
        $$
        то есть $n = U_m(m)$.

        И всё бы было хорошо, если бы $U_m(m)$ было всегда определено. Но вернёмся в суровую
        реальность и докажем истинное утверждение теоремы Клини.

        Вместо равенства чисел нужно рассматривать эквивалентность следующего вида: $n \sim m$,
        если $U_n = U_m$.

        Рассмотрим вычислимую функцию $$V(m, x) = U_{f(U_m(m))}(x).$$ По свойству главности
        нумерации $U$ найдётся тотальная вычислимая функция $s$ такая, что $$V(m, x)
        = U_{s(m)}(x).$$ Значит, можем записать, что
        $$
            f(U_m(m)) \sim s(m).
        $$

        Важно отметить, что $s(m)$ тотальна. Какое бы значение $m$ мы не подставили, наше
        утверждение сохранит истинность в силу того, что правая часть уравнения определена.

        Теперь, зная, что $s(m)$ вычислима, запишем её как $U_k(m)$
        $$
            f(U_m(m)) \sim U_k(m),
        $$
        и подставим $m = k$
        $$
            f(U_k(k)) \sim U_k(k).
        $$

        Чудесным образом получили слева и справа вполне определённые значения, так как и $f$,
        и $U_k$ являются тотальными функциями. То есть, неподвижной точкой (в смысле
        определённой нами эквивалентности, а не обычного равенства) функции $f$ является
        $U_k(k)$, совершенно точно определённое значение.
    \end{proof}
\end{theorem}