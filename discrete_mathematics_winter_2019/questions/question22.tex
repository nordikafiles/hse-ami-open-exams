\section{Исчисление резолюций для теорий, состоящих из формул общего вида (приведение к предваренной нормальной форме и сколемизация). Доказательства общезначимости с помощью ИР. Выводимость формулы в теории с помощью ИР. Теорема компактности.}

\subsection{Исчисление резолюций для теорий, состоящих из формул общего вида (приведение к предваренной нормальной форме и сколемизация).}

% Для того, чтобы применить правила исчисления резолюций для теории, состоящих из формул общего вида, необходимо для нача
\begin{definition}
  Предваренное нормальной формулой называется такая формула, что все кванторы стоят в начале.
\end{definition}

\begin{definition}
  Сколемовской нормальной формой называется предваренная нормальная форма, где все кванторы -- кванторы всеобщности.
\end{definition}

\begin{statement}
  Если привести формулу к сколемовской нормальной форме, то метод ИР можно использовать как для множеств универсальных дизъюнктов.
  \begin{proof}
    Так как, формула в общем виде эквивалента формуле в предваренном нормальной виде. А при переходе к сколемовской нормальной формуле сохраняется совместность (если формула имела модель, то и новая формула будет иметь модель, и наоборот). А для сколемовской нормальной формы уже можно использовать как для множеств универсальных дизъюнктов.
  \end{proof}
\end{statement}

\subsection{Доказательства общезначимости с помощью ИР.}

Для доказательства общезначимости формулы, достаточно доказать несовместность отрицания с помощью ИР.

\subsection{Выводимость формулы в теории с помощью ИР.}

\subsection{Теорема компактности.}

Пусть $T$ -- множество формул.

\begin{theorem}[Теорема компактности]
  Если $T$ несовместно, то у него существует несовместное конечное подмножество $T'$.
  \begin{proof}
    
  \end{proof}
\end{theorem}