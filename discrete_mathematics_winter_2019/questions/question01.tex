\section{Определение вычислимой частичной функции из $\mathbb{N}$ в $\mathbb{N}$. Счетность семейства частичных вычислимых функций, и существование невычислимых функций. Разрешимость подмножества $\mathbb{N}$. Перечислимые подмножества $\mathbb{N}$. Счетсность семейства перечислимых множеств, и существование неперечислимых множеств.}



\subsection{Определение вычислимой частичной функции из $\mathbb{N}$ в $\mathbb{N}$}
\begin{definition}
  Пусть $A$ и $B$ некоторые множества. Частичной функцией из $A$ в $B$ называется произвольное подмножество $f \subseteq A \times B$, удовлетворяющая свойству
  \[
    \forall a \in A, b_1,b_2 \in B \> (a,b_1) \in f \wedge (a,b_2) \in f \Rightarrow b_1 = b_2
  \]
  Обозначение: $f:A \overset{p}{\to} B$
\end{definition}

\begin{definition}
  Функция $f: A \overset{p}{\to} B$ вычислима, если существует программа (на C, на ассемблере, машина Тьюринга и т.п.), которая на любом входе $x \in \operatorname{dom} f$ выписывает $f(x)$ и завершается, а на любом входе $x \in A \setminus \operatorname{dom} f$ не завершается ни за какое конечное число шагов.
\end{definition}

\subsection{Счетность семейств частичных вычислимых функций, и существование невычислимых функций.}

\begin{statement}
  Множество частичных вычислимых функций не более, чем счетно.
  \begin{proof}
    Действительно, всякой вычислимой функции можно поставить в соответствие алгоритм, причем различные функции вычисляются различными алгоритмами. Алгоритм -- это конечная строка. То есть множество алгоритмов счетно. Существует инъекция из множества вычислимых функций в множество алгоритмов, следовательно, множество вычислимых функций не более, чем счетно.
  \end{proof}
\end{statement}

\begin{statement}
  Существуют невычислимые функции $f : \mathbb{N} \overset{p}{\to} \mathbb{N}$.
  \begin{proof}
    Множество всех функций из $\mathbb{N}$ в $\mathbb{N}$ имеет мощность континуум, а множество вычислимых функций не более, чем счетно. Следовательно, множество невычислимых функций не пусто.
  \end{proof}
\end{statement}

\subsection{Разрешимость подмножества $\mathbb{N}$.}

\begin{definition}
  Множество $A$ разрешимо, если его характеристическая функция
  \[
    \bigchi_A(x) = \begin{cases}
      1, & \textit{если } x \in A \\
      0, & \textit{если } x \notin A
    \end{cases}
  \]
  вычислима.
\end{definition}

\subsection{Перечислимые подмножества $\mathbb{N}$.}

\begin{definition}
  Множество $A$ пречислимо, если есть программа, на пустом входе последовательно выписывающая все элементы $A$ и только их.
\end{definition}

\subsection{Счетность семейства перечислимых множеств, и существование неперечислимых множеств.}

\begin{statement}
  Множество перечислимых множеств $\mathbb{N}$ не более, чем счетно.
  \begin{proof}
    Всякому перечислимому множеству соответствует алгоритм, который его перечисляет, причем разные множества перечисляются разными алгоритмами. Отсюда следует, что мощность множества перечислимых множеств $\mathbb{N}$ не превосходит мощность множества алгоритмов, которое является счетным.
  \end{proof}
\end{statement}

\begin{statement}
  Существуют неперечислимые множества $A \subseteq \mathbb{N}$.
  \begin{proof}
    Множество перечислимых множеств не более, чем счетно. А множество всех подмножеств $\mathbb{N}$ имеет мощность континуум. Следовательно, множество неперечислимых подмножеств $\mathbb{N}$ не пусто. 
  \end{proof}
\end{statement}
