\section{Универсальные вычислимые функции (нумерации) для семейства частичных вычислимых функций натурального аргумента. Несуществование универсальной вычислимой функции для семейства тотальных вычислимых функций натурального аргумента (диагональное рассуждение). Главные универсальные функции.}

\subsection{Универсальные вычислимые функции (нумерации) для семейства частичных вычислимых функций натурального элемента.}

\begin{definition}
  Функций $U : \mathbb{N}^2 \overset{p}{\to} \mathbb{N}$ называется универсальной вычислимой, если она вычислима и для всякой вычислимой функции $f : \mathbb{N} \overset{p}{\to} \mathbb{N}$ найдется такое число $n \in \mathbb{N}$, называемое индексом функции $f$ относительно $U$, т.ч. $U_n = f$, т.е.
  \[
    \forall x (f(x) \simeq U(n, x))
  \] 
\end{definition}

\subsection{Несуществование универсальной вычислимой функции для семейства тотальных вычислимых функций натурального аргумента (диагональное рассуждение).}

\subsection{Главные универсальные функции.}

\begin{definition}
  Вычислимая функция $U : \mathbb{N}^2 \overset{p}{\to} \mathbb{N}$ называется главной универсальной вычислимой функцией, если для любой вычислимой функции $V : \mathbb{N}^2 \overset{p}{\to} \mathbb{N}$ найдется вычислимая тотальная функция $s : \mathbb{N} \to \mathbb{N}$, т.ч. $V_n = U_{s(n)}$ для всех $n \in \mathbb{N}$, т.е.
  \[
    \forall x \forall n V_n(x) \simeq U(s(n), x)
  \]
\end{definition}