\section{Сводимости: m-сводимость и Тьюрингова сводимость. Их свойства. Полные перечислимые множества.}

\subsection{m-сводимость.}

\begin{definition}
    Множество $A$ \textbf{m-сводится} к множеству $B$, если существует тотальная вычислимая
    функция $f$ такая, что $\forall x \ x \in A \iff f(x) \in B$. Обозначается как $A
    \leqslant_m B$.
\end{definition}

Свойства m-сводимости:
\begin{itemize}
    \item $A \leqslant_m A$ (рефлексивность),
    \item $A \leqslant_m B \wedge B \leqslant_m C \implies A \leqslant_m C$ (транзитивность),
    \item
    $
    \left.\begin{aligned}
            &B \text{ разрешимо}\\
            &A \leqslant_m B
    \end{aligned}\right\}
    \implies A \text{ разрешимо},
    $
    \item
    $
    \left.\begin{aligned}
            &A \text{ неразрешимо}\\
            &A \leqslant_m B
    \end{aligned}\right\}
    \implies B \text{ неразрешимо}.
    $
\end{itemize}

\subsection{Тьюрингова сводимость.}

\begin{definition}
    Множество $A$ \textbf{T-сводится} (сводится по Тьюрингу) к множеству $B$, если при помощи
    алгоритма вычисления $\chi_B$ (не обязательно существующего) можно вычислить $\chi_A$.
    Обозначается как $A \leqslant_T B$.
\end{definition}

Свойства Тьюринговой сводимости:
\begin{itemize}
    \item $A \leqslant_T A$ (рефлексивность),
    \item $A \leqslant_T B \wedge B \leqslant_T C \implies A \leqslant_T C$ (транзитивность),
    \item
    $
    \left.\begin{aligned}
            &B \text{ разрешимо}\\
            &A \leqslant_T B
    \end{aligned}\right\}
    \implies A \text{ разрешимо},
    $
    \item
    $
    \left.\begin{aligned}
            &A \text{ неразрешимо}\\
            &A \leqslant_T B
    \end{aligned}\right\}
    \implies B \text{ неразрешимо}.
    $
    \item $A \leqslant_m B \Rightarrow A \leqslant_T B$
    \item $A \leqslant_T \overline{A}$
\end{itemize}

\begin{definition}
    Перечислимое множество, к которому m-сводится любое другое перечислимое множество, называется
    \textbf{полным перечислимым множеством}.
\end{definition}

\begin{theorem}
    Существует полное перечислимое множество.
    \begin{proof}
        Рассмотрим множество $A = \{(n, x) \mid U_n(x) \text{ определено}\}$. Заметим, что оно
        перечислимо.
    
        Пусть мы хотим свести некоторое перечислимое множество $B$ к множеству $A$. Знаем, что
        в силу перечислимости существует вычислимая частичная функция $f$ такая, что $B
        = \operatorname{Dom} f$. Эта функция должна присутствовать в универсальной нумерации. Пусть
        это $U_n$. Тогда для того, чтобы проверить принадлежность $x \in B$, достаточно проверить
        принадлежность $(n, x) \in A$. Сводящая функция в данном случае выглядит как $m(x) = (n,
        x)$.
    \end{proof}
\end{theorem}