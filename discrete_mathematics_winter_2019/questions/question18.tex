\section{Определение формулы первого порядка в данной сигнатуре. Свободные и связанные вхождения переменных. Интерпретации данной сигнатуры. Общезначимые и выполнимые формулы. Равносильные формулы.}

\subsection{Определение формулы первого порядка в данной сигнатуре.}

\begin{definition}
  Сигнатура языка -- набор функциональных и предикатных символов определенных валентностей.
\end{definition}

\begin{definition}
  Термом являются
  \begin{itemize}
    \item любая индивидуальная переменная;
    \item $f(t_1, \dots, t_n)$, если $f$ -- функциональный символ валентности $n$, а $t_1, \dots, t_n$ -- термы.
  \end{itemize}
\end{definition}

\begin{definition}
  Атомарной формулой называется выражение вида $P(t_1, \dots, t_n)$, где $P$ -- предикатный символ валентности $n$, а $t_1, \dots, t_n$ -- термы.
\end{definition}

\begin{definition}
  Формулой первого порядка являются
  \begin{itemize}
    \item любая атомарная формула;
    \item  $a \wedge b$, $a \vee b$, $a \to b$, $\neg a$, если $a,b$ -- фомрулы
    \item $\exists x a$ и $\forall x a$, если $a$ -- формула.
  \end{itemize}
\end{definition}

\subsection{Свободные и связанные вхождения переменных.}

\begin{definition}
  Если перед каждым появлением переменной в формуле в составе терма эта переменная появляется под квантором, говорят что она связана в этой формуле. Иначе говорят, что она входит в формулу свободно. 
\end{definition}

\begin{definition}
  Формула без свободных вхождений называется замкнутой.
\end{definition}

\subsection{Интерпретации данной сигнатуры.}

\begin{definition}
  Интерпретацией сигнатуры называется набор, который состоит из непустого множества $M$ (носителя) и функции, сопоставленных каждому символу сигнатуры:
  \begin{itemize}
    \item функциональному символу $f$ валентности $n$ сопоставляется функция $\hat{f} : M^n \to M$;
    \item предикатному символу $p$ валентности $m$ -- функция $\hat{p} : M^m \to \{0,1\}$.
  \end{itemize}
\end{definition}

\subsection{Общезначимые и выполнимые формулы.}

\begin{definition}
  Формула называется общезначимой, если она верна в любой интерпретации и выполнимой, если верна хотя бы в одной. При этом, если формула содержит свободные переменные, то считается, что она верна, если она верна при любых их значениях.
\end{definition}

\subsection{Равносильные формулы.}

\begin{definition}
  Формулы $A$ и $B$ называются равносильными, если формула $(A \to B) \wedge (B \to A)$ общезначима.
\end{definition}