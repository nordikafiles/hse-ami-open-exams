\section{Теорема полноты исчисления резолюций (для пропозициональных формул в КНФ). Доказательство нужно знать только для конечных и счетных множеств формул.}

\subsection{Теорема полноты исчисления резолюций (для пропозициональных формул в КНФ).}

\begin{theorem}[Теорема полноты исчисления резолюций]
  Если множество дизъюнктов $S$ несовместно, то из него можно вывести пустой дизъюнкт.
  \begin{proof}
    Докажем для случая, когда $S$ не более, чем счетно. \newline
    Применим закон контрапозиции и докажем, что если из $S$ нельзя вывести пустой дизъюнкт, то оно совместно. \newline
    Пусть $S'$ -- множество всех формул, которое можно вывести из $S$ в исчислении резолюций. Понятно, что множество используемых переменных счетно, занумеруем их $x_1, \dots, x_n$. Для любого $k \in \mathbb{N}$ положим $S'_k$ все дизъюнкты из $S'$, содержащие переменные только с номерами не больше $k$ и докажем, что можно выбрать значения переменных так $x_1, \dots, x_k$, что $S'_k$ совместно. Докажем индукцией по $k$: \newline
    \begin{itemize}
      \item \textit{База.} $k=1$, тогда в $S'_k$ содержатся только $x_1$ и/или $x_1'$. Причем система совместна тогда и только тогда, когда $S'_k = x_1$ или $S'_k = \neg x_1$. Следовательно, можно выбрать такое значение.
      \item \textit{Переход.} Пусть верно для всех $k < n$, покажем, что верно и для $n$. По предположению индукции мы можем как-то выбрать $x_1, \dots, x_{n-1}$, чтобы $S_{n-1}$ было совместно. Предположим, что нельзя выбрать $x_n$. То есть и при $x_n=0$, и при $x_n=1$ система ложна.
      \begin{itemize}
        \item $x_n = 0$ делает систему ложной $\Rightarrow$ $\exists A \in S'_{n-1} \> A \vee x_n \in S'_n$ и $A$ ложно при выбранном для них значениях.
        \item $x_n = 1$ делает систему ложной $\Rightarrow$ $\exists B \in S'_{n-1} \> B \vee \neg x_n \in S'_n$ и $B$ ложно при выбранном для них значениях.
      \end{itemize}
      Тогда $A \vee B \in S'_{n-1}$. Следовательно, $A$ и $B$ не могут быть истинными. Противоречие.
    \end{itemize}
  \end{proof}
\end{theorem}