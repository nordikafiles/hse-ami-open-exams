\section{Вычислимая функция, не имеющая тотального вычислимого продолжения. Перечислимое неразрешимое множество. Неразрешимость проблемы применимости.}

\subsection{Вычислимая функция, не имеющая тотального вычислимого продолжения.}

\begin{definition}
  Функция $g$ является \textit{продолжением} функции $f$, если $\operatorname{Dom} f \subset
  \operatorname{Dom} g$ и $\forall x \in \operatorname{Dom} f \; g(x) = f(x)$.
\end{definition}

\begin{statement}
  Существует вычислимая функция, не имеющая вычислимого тотального продолжения.
  \begin{proof}
    Пусть $d(x) \simeq U(x, x)$, $d$ вычислима. Предположим, что вычислимая тотальная функция $g$ продолжает $d$. Тогда функция $h$, т.ч. $h(x) = g(x) + 1$ для всех $x \in \mathbb{N}$, также будет вычислимой тотальной. Пусть $h = U_n$, $h$ определена всюду, значит $n \in \operatorname{Dom} d$. Тогда $U_n(n) = h(n) = g(n) + 1 = d(n)+1 = U_n(n)+1$, что неверно. Следовательно, вычислимого тотального продолжения функции $d$ не существует.
  \end{proof}
\end{statement}

\subsection{Перечислимое неразрешимое множество.}

\begin{statement}
  Множество $K := \{n \> | \> U_n(n) \> \textit{определено} \}$ перечислимо, но не разрешимо.
  \begin{proof}
    Перечислимость очевидна, поскольку $K = \operatorname{Dom} d$, где вычислимая функция $d : \mathbb{N} \overset{p}{\to} \mathbb{N}$ такова, что $d(x) \simeq U(x,x)$ для всех $x \in \mathbb{N}$. \newline
    Установим неразрешимость $K$. Предположим противное. Тогда функция
    \[
      g(x) = \begin{cases}
        d(x), & x \in \operatorname{Dom} d \\
        0, & x \notin \operatorname{Dom} d
      \end{cases}
    \]
    является тотальной, более того вычислимой (поскольку $d(x)$ вычислима и $\operatorname{Dom} d$ разрешимо). Но мы знаем, что $d$ не имеет тотального вычислимого продолжения. Противоречие. Следовательно, $K$ неразрешимо.
  \end{proof}
\end{statement}

\subsection{Неразрешимость проблемы применимости (остановки).}

\begin{definition}
  Задача разрешения множества $S := \{ (n,x) \> | \> U(n,x) \> \textit{определено} \}$ называется проблемой применимости (остановки).
\end{definition}

\begin{theorem}
  Проблема применимости (остановки) неразрешима.
  \begin{proof}
    Пусть $\bigchi_S$ вычисляется алгоритмом $S$. Тогда, запустив $S(x,x)$, можно разрешить множество $\{x \> | \> U_x(x) \> \textit{определено} \}$, для которого доказана неразрешимость.
  \end{proof}
\end{theorem}