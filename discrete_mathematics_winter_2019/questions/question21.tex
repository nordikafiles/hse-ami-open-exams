\section{Непротиворечивые теории. Теорема полноты ИР (для множеств универсальных дизъюнктов).}

\subsection{Непротиворечивые теории.}

\begin{definition}
  Непротиворечивой теорией называется теория такая, что в ней утверждение не быть одновременно доказано и опровергнуто.
\end{definition}

\begin{theorem}[Теорема полноты ИР.]
  Если $S$ -- несовместное множество универсальных дизъюнктов, то из $S$ можно вывести пустой дизъюнкт с помощью правил резолюций.
  \begin{proof}
    Пусть $S'$ -- множество, полученное из второго правила подставлением всех возможных термов. То есть множество $S'$ -- множество обычных дизъюнктов. Так как $S$ несовместно, то $S'$ тоже несовместно. Допустим, это не так и можно выбрать значения атомарных формул из $S'$ так, что все дизъюнкты были истинны. Тогда строим модель для $S$. $M$ -- множество термов. Пусть в сигнатуре есть функциональный символ $f$, будем интерпретировать его $\hat{f} = f$ и предикатный символ $P$, его интерпретировать будем так, чтобы он был истинным (предположили, что так можно сделать). Тогда все дизъюнкты $S'$ будут истинными, а значит будут истинными все дизъюнкты $S$. Следовательно, $S$ совместно, противоречие. Значит, мы свели теорему к случаю для дизъюнктов из пропозициональных переменных. А теорема полноты для нее уже была доказана.
  \end{proof}
\end{theorem}