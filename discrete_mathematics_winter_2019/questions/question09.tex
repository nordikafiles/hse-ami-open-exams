\section{Теорема Райса-Успенского.}

\begin{theorem}[Теорема Райса-Успенского]
  Пусть нумерация $U$ главная и множество $\mathcal{F}$ вычислимых функций одного аргумента нетривиально (т.е. найдется $f \in F$ и найдется вычислимая $g \notin F$). Тогда множество индексов
  \[
    F = \{n \> | \> U_n \in \mathcal{F}\}
  \]
  неразрешимо.
  \begin{proof}
    Пусть $g \in \mathcal{F}$ и $h \notin \mathcal{F}$ и множество $F$ разрешимо. Пусть $g = U_k$ и $h = U_m$. Определим функцию $f$
    \[
      f(n) = \begin{cases}
        m, & \textit{если } \> n \in F \\
        k, & \textit{если } \> n \notin F
      \end{cases}
    \]
    для всех $n \in \mathbb{N}$. Так как $F$ разрешимо, то тотальная функция $f$ вычислима. Тогда, по теореме Клини о неподвижной точке, найдется число $n \in \mathbb{N}$, т.ч. $U_{f(n)} = U_n$. Если $n \in F$, то $U_n \in \mathcal{F}$, но $U_n = U_{f(n)} = U_m = h \notin \mathcal{F}$. Если же $n \notin F$, то $U_n \notin \mathcal{F}$, следовательно, $U_n = U_{f(n)} = U_k = g \in \mathcal{F}$. Противоречие. Значит, $F$ неразрешимо.
  \end{proof}
\end{theorem}