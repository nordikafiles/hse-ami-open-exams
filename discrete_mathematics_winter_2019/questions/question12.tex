\section{Исчисление высказываний (аксиомы и правила вывода), понятие вывода. Теорема корректности исчисления высказываний.}

\subsection{Исчисление высказываний (аксиомы и правила вывода), понятие вывода.}

\begin{definition}
  Высказыванием называется утверждение, которое либо истинно, либо ложно. При этом если $A,B$ являются высказываниями, то $\neg A, A \vee B, A \wedge B, A \to B$ -- тоже высказывания.
\end{definition}

\begin{definition}
  Выводом называется конечная последовательность формул, каждая из которой либо является аксиомой, либо получается из ранее встретившихся по правилам вывода.
\end{definition}

Имеется 11 аксиом:
\begin{enumerate}
  \item $A \to (B \to A)$: истинна следует из чего угодно.
  \item $(A \to (B \to C)) \to ((A \to B) \to (A \to C))$: левая дистрибутивность импликации относительно самой себя
  \item $(A \wedge B) \to A$;
  \item $(A \wedge B) \to B$: из конъюнкции двух формул следует каждая из формул.
  \item $A \to (B \to (A \wedge B))$: если выполнены лбе формулы, то выполнена из конъюнкция.
  \item $A \to (A \vee B)$;
  \item $B \to (A \vee B)$: дизъюнкция двух формул следует из каждой из них.
  \item $(A \to C) \to ((B \to C) \to ((A \vee B) \to C))$: если формула $C$ следует из каждой из формул $A$ и $B$, то то она следует и из их дизъюнкции.
  \item $\neg A \to (A \to B)$: из лжи следует все, что угодно.
  \item $(A \to B) \to ((A \to \neg B) \to \neg A)$: правило рассуждения от противного, если из $A$ следует $B$ и $\neg B$, то само $A$ обязано быть неверным.
  \item $A \vee \neg A$: закон исключенного третьего.
\end{enumerate}

В качестве единственного правила вывода выступает \textit{modus ponens}:
\[
  \frac{A \squad A \to B}{B}
\]
Понимается оно так: если ранее уже выведены формулы $A$ и $A \to B$, то в вывод можно приписать $B$. \newline

Буквами $A,B,C$ могут быть обозначены любые формулы. \newline

Утверждение о том, что формула $\varphi$ выводима в исчислении высказываний (ИВ), записывается так: $\vdash \varphi$.

\subsection{Теорема корректности исчисления высказываний.}

\begin{definition}
  Формула называется тавтологией, если она как булева формула верна при всех значениях входящих в нее переменных.
\end{definition}

\begin{theorem}[Теорема корректности исчисления высказываний]
  Если $\vdash \varphi$, то $\varphi$ -- тавтология.
  \begin{proof}
    Любая аксиома является тавтологией. Это можно проверить непосредственно по таблице истинности. Если $A$ и $A \to B$ являются тавтологиями, то $B$ также является тавтологией: только при $A = B = 1$ верны и формула $A$ и импликация $A \to B$. Индукцией по номеры формулы в выводе доказывается, что все формулы в выводе тавтологичны, что и требовалось.
  \end{proof}
\end{theorem}