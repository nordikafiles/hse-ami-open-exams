\section{Теорема Поста. Теорема о графике.}

\subsection{Теорема Поста}

\begin{theorem}[Теорема Поста]
  Множества $A$ и $\overline{A}$ перечислимы тогда и только тогда, когда $A$ разрешимо.
  \begin{proof}
    $\Rightarrow$ \newline
    Построим алгоритм, вычисляющий $\bigchi_A(x)$: будем по очереди делать по одному шагу для $w_A(x)$ и $w_{\overline{A}}(x)$, т.к. $x \in A \vee x \in \overline{A}$, то какой-то один из алгоритмов вернет $1$ на каком-то шаге. Если это будет $w_A$, то вернем $1$, если же $w_{\overline{A}}$, то вернем $0$. \newline
    $\Leftarrow$ \newline
    Очевидно, из разрешимости следует перечислимость. Если $A$ разрешимо, то и $\overline{A}$ разрешимо.
  \end{proof}
\end{theorem}

\subsection{Теорема о графике}

\begin{definition}
  Пусть задана функция $f : \mathbb{N} \overset{p}{\to} \mathbb{N}$. Графиком функции $f$ называется множество $\Gamma_f = \{(x, f(x)) \> | \> x \in \operatorname{Dom} f\}$.
\end{definition}

\begin{theorem}[Теорема о графике]
  Функция $f : \mathbb{N} \overset{p}{\to} \mathbb{N}$ вычислима тогда и только тогда, когда ее график $\Gamma_f$ перечислим.
  \begin{proof}
    $\Rightarrow$ \newline
    Пусть $f$ вычислима. Тогда $\operatorname{Dom} f$ вычислима и, следовательно, есть вычислимая функция $g$, т.ч. $\operatorname{rng} g = \operatorname{Dom} f$. Рассмотрим функцию $h : \mathbb{N} \to \mathbb{N}^2$, т.ч. $h(x) \simeq (g(x), f(g(x)))$ для всех $x \in \mathbb{N}$. Она вычислима, причем
    \[
      (x,y) \in \operatorname{rng} h \Leftrightarrow
      (x \in \operatorname{rng} g \wedge y = f(x)) \Leftrightarrow
      (x \in \operatorname{Dom} f \wedge y = f(x)) \Leftrightarrow
      (x,y) \in \Gamma_f
    \]
    для всех $x,y \in \mathbb{N}$. Значит, $\Gamma_f = \operatorname{rng} h$, следовательно, $\Gamma_f$ перечислим (т.к. $h$ вычислимая). \newline
    $\Leftarrow$ \newline
    Пусть $\Gamma_f$ перечислим. Тогда, чтобы вычислить $f(x)$, достаточно выписывать элементы $\Gamma_f$ и проверять, совпадает ли первая координата пары с $x$. Если совпадает, выдавать вторую координату. Этот процесс завершается тогда и только тогда, когда $x \in \operatorname{Dom} f$.
  \end{proof}
\end{theorem}