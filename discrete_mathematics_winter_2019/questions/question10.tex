\section{Определение машин Тьюринга и вычислимых на машинах Тьюринга функций. Тезис Черча-Тьюринга. Неразрешимость проблемы остановки машины Тьюринга.}

\subsection{Определение машин Тьюринга и вычислимых на машинах Тьюринга функций.}

\begin{definition}
  Машина Тьюринга задается
  \begin{itemize}
    \item непустым конечным алфавитом $\Sigma$, среди которого выделен пробельный символ $\textvisiblespace$ и не содержащее пробельного символа множества $\Gamma$ -- входной алфавит;
    \item непустым конечным множеством состояний $Q$, среди которого выделено начальное состояние $s_0$ и множество терминальных состояний $F \subseteq Q$;
    \item функций переходов $\delta : (Q \setminus F) \times \Sigma \to Q \times \Sigma \times \{-1,0,1\}$.
  \end{itemize}
\end{definition}

Машина Тьюринга состоит из бесконечной ленты, разбитой на ячейки, головки, в любой момент времени указывающей на одну ячейку и одной ячейки памяти, в которой хранится текущее состояние. В начальный момент времени на ленте записано некоторое слово, составленной из букв входного алфавита, головка смотрит на первый символ этого слова, во всех остальных ячейках пробелы. Затем в каждый момент времени вычисляется $\delta(q,c) = (q', c', \Delta)$, где $q$ -- текущее состояние, $c$ -- символ записанный в ячейке, на которую сейчас смотрит головка. Состояние меняется на $q'$, символ в текущей ячейках на $c'$, головка остается на месте или передвигается на один влево или вправо в соответсвии со значением $\Delta$. Если $q' \in F$, то работа машины заканчивается, иначе этот процесс продолжается. \newline
Машины Тьюринга естественным образом отождествляются с частичными функциями $f : \Gamma* \to \Gamma*$ -- аргументом функции является входное слово, а возвращает функция слово, записанное на ленте после завершения работы машины. Функции, которые можно таким образом получить по некоторой машине Тьюринга, называются \textit{вычислимыми на машине Тьюринга}.

\subsection{Тезис Черча-Тьюринга.}

\begin{statement}[Тезис Черча-Тьюринга]
  Любая вычислимая функция вычислима на машине Тьюринга.
\end{statement}
 Здесь понятие <<вычислимая функция>> используется в неформальном смысле, под ним понимается функция, вычислимая в любой разумной модели, которая может прийти нам в голову. Тезис не является формальным утверждением, он никак не доказывается и принимается нами на веру.

\subsection{Неразрешимость проблемы остановки машин Тьюринга.}

\begin{statement}
  Не существует вычислимой функции, определяющей по машине Тьюринга и входному слову, остановится ли эта машина.
  \begin{proof}
    Следует из утверждения о существовании полного перечислимого множества.
  \end{proof}
\end{statement}