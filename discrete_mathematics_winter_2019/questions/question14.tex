\section{Теорема полноты исчисления высказываний.}

\subsection{Теорема полноты исчисления высказываний.}

\begin{lemma}[Базовая]
  Имеют место следующие выводимости:
  \begin{center}
    \begin{tabular}{cccc}
      $A,B \vdash A \wedge B$ & $A,B \vdash A \vee B$ & $A,B \vdash A \to B$ & \\
      $A,\neg B \vdash \neg (A \wedge B)$ & $A,\neg B \vdash A \vee B$ & $A,\neg B \vdash \neg (A \to B)$ & $A \vdash \neg (\neg A)$ \\
      $\neg A,B \vdash \neg (A \wedge B)$ & $\neg A,B \vdash A \vee B$ & $\neg A,B \vdash A \to B$ & $\neg A \vdash \neg A$ \\
      $\neg A,\neg B \vdash \neg (A \wedge B)$ & $\neg A,\neg B \vdash \neg (A \vee B)$ & $\neg A,\neg B \vdash A \to B$ & \\
    \end{tabular}
  \end{center}
  \begin{proof}
    Слева от знака выводимости стоят либо формулы, либо отрицания формул $A$ и $B$. Справа стоит более сложная формула или ее отрицание в зависимости от того, что из двух верно, если верны обе посылки. \newline
    Все 14 утверждений очень простые и выводятся из аксиом за несколького шагов.
  \end{proof}
\end{lemma}

Следующая лемма обобщает предыдущую.
\begin{lemma}[Основная]
  Пусть $A$ -- формула, $x \in \{0,1\}$. Через $A^x$ обозначим формулу $A$, если $x=1$, и формулу $\neg A$, если $x=0$. Далее, пусть $\Phi$ -- формула от $n$ переменных, выражающая функцию $\varphi$. Тогда для всех $x_1, \dots, x_n$ выполнено
  \[
    A_1^{x_1}, \dots, A_n^{x_n} \vdash \Phi(A_1, \dots, A_n)^{\varphi(x_1, \dots, x_n)}
  \]
  \begin{proof}
    Будем доказывать индукцией по построению формулы. В качестве базы возьмем $\Phi(A_1,$ $\dots, A_n) = A_i$. Тогда $\varphi(x_1, \dots, x_n) = x_i$ и в правой части стоит просто одна из посылок. \newline
    Теперь докажем переход. Для разных связок он делается одинаково, возьмем для примера конъюнкцию. Таким образом, формула $\Phi$ есть конъюнкция $\Psi \wedge \Omega$, а $\varphi$ есть также конъюнкция функций $\psi$ и $\omega$, но уже в смысле булевой функции, а не синтаксической связки. По предположению индукции имеем $A_1^{x_1}, \dots, A_n^{x_n} \vdash \Psi(A_1, \dots, A_n)^{\psi(x_1, \dots, x_n)}$ и $A_1^{x_1}, \dots, A_n^{x_n} \vdash \Omega(A_1, \dots, A_n)^{\omega(x_1, \dots, x_n)}$. Далее по базовой лемме имеем
    \[
      \Psi(A_1, \dots, A_n)^{\psi(x_1, \dots, x_n)}, \Omega(A_1, \dots, A_n)^{\omega(x_1, \dots, x_n)} \vdash (\Psi \wedge \Omega)(A_1, \dots, A_n)^{(\psi \wedge \omega)(x_1, \dots, x_n)}.
    \]
    Последняя формула есть $\Phi(A_1, \dots, A_n)^{\varphi(x_1, \dots, x_n)}$. Далее получаем, что $A_1^{x_1}, \dots, A_n^{x_n} \vdash \Phi(A_1, \dots, A_n)^{\varphi(x_1, \dots, x_n)}$, что и требовалось. Переходы для остальных связок доказываются аналогично. Таким образом, индуктивное утверждение доказано.
  \end{proof}
\end{lemma}

\begin{theorem}[Теорема полноты исчисления высказываний]
  Пусть формула $\Phi$, зависящая от переменных $p_1, \dots, p_n$, является тавтологией. Тогда $\vdash \Phi$.
  \begin{proof}
    Поскольку формула $\Phi$ является тавтологией, она выражает функцию $\varphi$, тождественно равную единице. По основной лемме для любого набора $(x_1, \dots, x_n) \in \{0,1\}^n$ выполнено $p_1^{x_1}, \dots, p_n^{x_n} \vdash \Phi$. При помощи обратной индукции докажем, что для каждого $k$ от $0$ до $n$ выполнено $p_1^{x_1}, \dots, p_k^{x_k} \vdash \Phi$. Начальное утверждение (для $k=n$) у нас уже есть. Конечное (для $k=0$) утверждает $\vdash \Phi$, что и требуется. Осталось доказать переход. \newline
    Пусть для некоторого $k$ утверждение уже доказано. Докажем для $k-1$. Из полученного имеем $p_1^{x_1}, \dots, p_k \vdash \Phi$ и $p_1^{x_1}, \dots, \neg p_k \vdash \Phi$. По правилу исчерпывающего разбора случаев получаем $p_1^{x_1}, \dots, p_{k-1}^{x_{k-1}} \vdash \Phi$, что и требовалось. Таким образом, теорема о полноте доказана.
  \end{proof}
\end{theorem}