\section{Критерий Коши сходимости функциональных последовательностей и рядов.}

\begin{theorem}[Критерий Коши равномерной сходимости функциональной последовательности]\label{koshfun}
    \[
        \{f_n(x)\} \rightrightarrows \text{ на } \mathbb{E}
        \Leftrightarrow
        \forall \varepsilon > 0 \>
        \exists N(\varepsilon) \>
        \forall n \geqslant N \>
        \forall p \in \mathbb{N} \>
        \forall x \in \mathbb{E} \>
        |f_n(x) - f_{n+p}(x)| < \varepsilon
    \]
    \begin{proof}
        \text{} \newline
        \begin{itemize}
            \item \textbf{Необходимость}
                Пусть $\{f_n(x)\} \rightrightarrows f(x)$ на $\mathbb{E}$. Тогда
                \[
                    \forall \varepsilon > 0
                    \exists N(\varepsilon)
                    \forall n \geqslant N
                    \forall x \in \mathbb{E}
                    |f_n(x) - f(x)| < \frac{\varepsilon}{2}.
                \]
                Тогда и подавно $\forall p \in \mathbb{N} |f_{n+p}(x) - f(x)| < \frac{\varepsilon}{2}$. Но
                \[
                    |f_{n+p}(x) - f_n(x)| = |f_{n+p}(x) - f(x) + f(x) - f_n(x)|
                    \leqslant
                    |f_{n+p}(x) - f(x)| + |f_n(x) - f(x)| < \varepsilon.
                \]
            \item \textbf{Достаточность}
            Зафиксируем произвольное $x \in \mathbb{E}$. Теперь, используя признак Коши сходимости числовой последовательности, получаем сходимость $\{f_n(x)\} \forall x \in \mathbb{E}$. А это значит, что существует предельная функция $f(x)$.
            \newline
            Снова зафиксируем произвольные $x \in \mathbb{E}$ и $\varepsilon > 0$. Делая предельный перезод в неравенстве $|f_{n+p}(x) - f_n(x)| < \varepsilon$ при $p \to \infty$, получаем $|f_n(x) - f(x)| \leqslant \varepsilon < 2 \varepsilon = \varepsilon'$.
        \end{itemize}
    \end{proof}
\end{theorem}

\begin{theorem}
    Функциональный ряд сходится равномерно, тогда и только тогда, когда последовательность его частичных сумм сходится равномерно.
    \begin{proof}
        Прямое следствие из теоремы \ref{koshfun}.
    \end{proof}
\end{theorem}