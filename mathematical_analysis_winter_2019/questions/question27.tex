\section{Дайте определение квадрируемости плоской фигуры по Жордану. Докажите критейрий квадрируемости плоской фигуры. В чем состоит свойство конечной аддитивности меры Жордана?}

\subsection{Дайте определение квадрируемости плоской фигуры по Жордану.}
\begin{definition}
    Множество $M \subset \mathbb{R}^2$ называется элементарным, если его можно представить в виде объединения конечного числа непересекающихся прямоугольников с вычислимой площадью.
\end{definition}
\begin{definition}
    Пусть $E \subset \mathbb{R}^2$ -- ограниченное множество. Числа
    \[
        S_*(E) = \sup_{A \subset E} S(A),
        \quad
        S^*(E) = \inf_{E \subset B} S(B),
    \] где верхняя и нижняя грани берутся по всем желементарным множествам $A$ и $B$ ($A \subset E \subset B$), называются соответственно нижней и верхней мерой множества $E$.
\end{definition}
\begin{definition}
    Ограниченное множество $E \subset \mathbb{R}^2$ называется квадрируемым по Жордану, если его нижняя и верхняя меры совпадают (т.е. $S_*(E) = S^*(E)$).
\end{definition}

\subsection{Докажите критейрий квадрируемости плоской фигуры.}
\begin{theorem}
    Плоская фигура $E$ квадрируема тогда и только тогда, когда
    \[
        \forall \varepsilon > 0
        \exists Q, P (P \subset e \subset Q)
        \quad
        S(Q) - S(P) < \varepsilon
    \]
    \begin{proof}
        \text{} \newline
        \begin{enumerate}
            \item \textbf{Необходимость} \newline
            Пусть $E$ квадрируема, т.е. $S_*(E) = S^*(E)$. По определению верхней и нижней меры для любого фиксированного $\varepsilon > 0$ найдутся такие $P$ и $Q (P \subset E \subset Q)$, что $S_* - \frac{\varepsilon}{2} < S(P) \leqslant S_*, S^* < S(Q) \leqslant S^* + \frac{\varepsilon}{2}$.
            Получается, что $S(Q) - S(P) < \varepsilon$
            \item \textbf{Достаточность} \newline
            Пусть 
            \[
                \forall \varepsilon > 0
                \exists Q, P (P \subset E \subset Q)
                S(Q) - S(P) < \varepsilon
            \]
            \[
                S(P) \leqslant S_* \leqslant S^* \leqslant S(Q)
                \Rightarrow
                0 \leqslant S^* - S_* \leqslant S(Q) - S(P) < \varepsilon
            \]
            Так как $\varepsilon$ -- произвольное положительное число, то получаем, что $S_* = S^*$.
        \end{enumerate}
    \end{proof}
\end{theorem}

\subsection{В чем состоит свойство конечной аддитивности меры Жордана?}
\begin{definition}
    Измеримость по Жордану обладает свойством конечной аддитивности, т.е. если
    \[
        F = \bigcup_{i=1}^n F_i,
    \]
    а для любых $i \neq j$ выполняется $F_i \cap F_j = \varnothing$, причем все $F_i$ измеримы, то и $F$ измерима, причем
    \[
        S(F) = \sum S(F_i)
    \]
\end{definition}