\section{Условно сходящийся числовой ряд. Признак Лейбница сходимости знакопеременного ряда вместе с оценкой на его остаток.}

\begin{definition}
    Будем говорить, что числовой ряд $\sum u_k$ сходится \textbf{условно}, если ряд $\sum u_k$ сходится, а ряд $\sum |u_k|$ расходится.
\end{definition}

\begin{theorem}
    Пусть для любого $k \in \mathbb{N}$ выполняется $a_k \geqslant a_{k+1}$, причем $a_k \to 0$. Тогда числовой ряд (называемый рядом Лейбница) $\sum (-1)^{k+1}a_k$ сходится, причем
    \[
        |r_k| = \left|
            \sum_{l=k+1}^\infty (-1)^{l+1} a_l
        \right| \leqslant a_{k+1}
    \]
    \begin{proof}
        Рассмотрим частичную сумму ряда Лейбница $S_{2n}$:
        \[
            0 \leqslant S_{2n} = (a_1 - a_2) + (a_3 - a_4) + ... + (a_{2n-1} - a_{2n}) =
            a_1 - (a_2 - a_3) - (a_4 - a_5) - ...- (a_{2n-2} - a_{2n-1}) - a_{2n} \leqslant a_1
        \]
        Из этого можно сделать вывод, что последовательность $\{S_{2n}\}$ -- ограниченная и монотонно неубывающая. Тогда $\exists \lim_{n \to \infty} S_{2n} = S$. С другой стороны, видно, что $S_{2n-1} = S_{2n} + a_{2n}$. Тогда $\exists \lim_{n \to \infty} S_{2n-1} = S + 0 = S$, т.е. $\lim_{n \to \infty} S_n = S$.
        \newline
        Итак, мы доказали, что ряд сходится. Теперь докажем вторую часть теоремы. Для этого заметим, что поскольку $\{S_{2n}\}$ не убывает, а $\{S_{2n - 1}\}$ не возрастает (т.к. $S_{2n + 1} = S_{2n-1} - (a_{2n} - a_{2n+1})$), то $S_{2n} \leqslant S \leqslant S_{2n - 1}$, а также $S \leqslant S_{2n + 1}$. По определению остаточного члена $r_{2n} = S - S_{2n}$. Пользуясь этими замечаниями, можно записать
        \[
            r_{wn} = S - S_{2n} \leqslant S_{2n + 1} - S_{2n} = a_{2n + 1},
        \]
        \[
            S_{2n-1} - S \leqslant S_{2n-1} - S_{2n} = a_{2n} \Rightarrow |r_{2n-1}| \leqslant a_{2n}.
        \]
        Но тогда $|r_n| \leqslant a_{n+1}$, что и требовалось доказать.
    \end{proof}
\end{theorem}