\section{Приведите 3 примера степенных рядов: (1) сходится везде на границе круга сходимости, (2) не сходится на границе круга сходимости, (3) в некоторых точках границы круга сходимости ряд сходится, а в некоторых -- нет. Дайте определение функции, аналитической в точке $x_0$.}

\subsection{Приведите 3 примера степенных рядов: (1) сходится везде на границе круга сходимости, (2) не сходится на границе круга сходимости, (3) в некоторых точках границы круга сходимости ряд сходится, а в некоторых -- нет.}
\begin{enumerate}
    \item $\sum \frac{z^n}{n^2}$ -- сходится при $|z| = 1$
    \item $\sum z^n$ -- не сходится на границе круга сходимости 
    \item $\sum \frac{z^n}{n}, R = 1$ -- расходится при $z = 1$, сходится при $z = -1$ (по признаку Лейбница)
\end{enumerate}

\subsection{Дайте определение функции, аналитической в точке $x_0$.}