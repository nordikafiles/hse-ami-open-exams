\section{Дайте определение кратного интеграла от функции двух переменных по компактному квадрируемому множеству, со всеми необходимыми определениями (разбиение, диаметр разбиения, размеченное разбиение, измельчение, интегральная сумма).}
Пусть дана функция $z = f(x,y), G$ -- область изменения переменных $x$ и $y$ ($G$ -- компактно и квадрируемо).
\begin{definition}
    Разбиение $\sigma$ множества $G$ -- набор попарно непересекающихся подмножеств $\sigma = \{G_i \subset G\}$, которые в объединении дают все $G$.
\end{definition}
\begin{definition}
    Диаметр разбиения $d$ -- наибольший диаметр множеств $G_i$.
    \[
        в = \max_{i} ( \sup_{M_1, M_2 \in G_i} \rho(M_1, M_2))
    \]
\end{definition}
\begin{definition}
    Размеченное разбиение -- разбиение множества $G$ вместе с конечной последовательностью $M_1, ..., M_n$, с условием, что $M_i \in G_i$
\end{definition}
\begin{definition}[Измельчение разбиения]
    Возьмем более мелкое разбиение по $x,y$, т.е. каждая клетка мелкого разбиения будет содержаться в более крупной. Тогда получим разбиение мельче исходного.
\end{definition}
\begin{definition}[Интегральная сумма]
    Сумма $S_{f, (\sigma, M)} = \sum_{i=1}^n f(M_i) S(G_i)$ называется интегральной суммой для функции $f$, соответствующей разбиению $\sigma$ и заданному выбору точек $M_i$.
\end{definition}
\begin{definition}
    Кратным интегралом функции $f$ на множестве $G$ называется число $I$, такое что
    \[
        I = \int \limits_{G} f(x,y) dx dy = \lim_{|\sigma| \to 0} S_{f, (\sigma, M)}.
    \]
    Обозначение:
    \[
        \iint \limits_{G} f(x,y) dx dy = \int \limits_G f(M) dS
    \]
\end{definition}