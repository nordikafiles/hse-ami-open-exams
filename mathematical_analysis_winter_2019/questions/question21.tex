\section{Определение радиуса и круга сходимости степенного ряда. Докажите, что степенной ряд сходится равномерно на любом замкнутом круге, лежащем строго внутри круга сходимости.}
\begin{definition}[Формула Коши-Адамара]
    Радиус сходимости ряда -- это число $$R = \frac{1}{\lim_{n \to \infty} \sqrt[n]{|a_n|}}$$ (число или $+ \infty$)
\end{definition}
\begin{definition}
    Круг сходимости ряда -- это $\{z \in \mathbb{C} \> | \> |z - z_0| < R\}$.
    Нас интересует, при каких $z$ сходится $\sum a_n (z - z_0)^n$. Сделав замену $z := z - z_0$, сведем вопрос к $\sum a_n z^n$.
\end{definition}

\begin{theorem}[о равномерной сходимости степенного ряда]
    Пусть $R$ -- радиус сходимости ряда $\sum a_n z^n$ и $0 < r < R$. Тогда в замкнутом круге $\{ z \in \mathbb{C} | |z| \leqslant r\}$ ряд сходится равномерно.
    \begin{proof}
        При $|z| \leqslant r$ имеем $|a_n z^n| \leqslant |a_n|r^n$, а ряд $\sum |a_n| \cdot r^n$ сходится по теореме \ref{sumconvr} (т.к. $r < R$). Значит, по признаку Вейерштрасса $\sum a_n z^n$ сходится равномерно.
    \end{proof}
\end{theorem}