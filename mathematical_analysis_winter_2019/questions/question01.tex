\section{Понятие числового ряда, его частичной суммы. Сходимость и расходимость числовых рядов. Примеры сходящихся и расходящихся числовых рядов. Необходимый признак сходимости числового ряда.}

\subsection{Понятие числового ряда, его частичной суммы.}
\begin{definition}
    Числовая последовательность $a_k$, рассматриваемая вкупе с последовательностью
    \[
        S_n = \sum_{k=1}^{n} a_k
    \]
    ее частичных сумм, называется \textbf{числовым рядом}.
\end{definition}

\subsection{Сходимость и расходимость числовых рядов.}
\begin{definition}
    Числовой ряд называется \textbf{сходящимся}, если
    \[
        \exists \lim_{n \to \infty} S_n = S < \infty
    \]
    и \textbf{расходящимся} иначе. Число $S$ называется \textbf{суммой ряда}.
\end{definition}

\subsection{Примеры сходящихся и расходящихся числовых рядов.}
\begin{enumerate}
    \item $\sum_{n=1}^{\infty} \frac{1}{n}$ -- расходится (гармонический ряд)
    \item $\sum_{n=1}^{\infty} \frac{1}{n^2}$ -- сходится
    \item $\sum_{n=1}^{\infty} \frac{1}{e^n}$ -- сходится
    \item $\sum_{n=1}^{\infty} n$ -- расходится
\end{enumerate}

\subsection{Необходимый признак сходимости числового ряда.}
\begin{theorem}
    Необходимым условием сходимости числового ряда является стремление к 0 его $n$-го члена $a_n$.
    \begin{proof}
        Действительно, в противном случае не выполняется критерий Коши для числовой последовательности $S_n$.
    \end{proof}
\end{theorem}