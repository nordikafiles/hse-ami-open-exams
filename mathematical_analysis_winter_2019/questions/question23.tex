\section{Лемма о сохранении радиуса сходимости при почленном дифференцировании степенного ряда. Теорема о почленном дифференцировании и интегрировании степенного ряда.}

\subsection{Лемма о сохранении радиуса сходимости при почленном дифференцировании степенного ряда.}
\begin{lemma}
    Радиусы сходимости рядов $\sum a_k (x - x_0)^k$ и $\sum k a_k (x - x_0)^{k-1}$ совпадают.
    \begin{proof}
        Очевидно, т.к. $\lim_{n \to \infty} \sqrt[k]{k} = 1$.
    \end{proof}
\end{lemma}

\subsection{Теорема о почленном дифференцировании и интегрировании степенного ряда.}
\begin{theorem}
    Пусть $R > 0$ -- радиус сходимости ряда $f(x) = \sum a_k (x - x_0)^k$. Тогда при $|x - x_0| < R$
    \begin{enumerate}
        \item $f$ имеет производную всех порядков, которые можно вычислить почленным дифференцированием
        \item $\int \limits_{x_0}^k f(t) dt = \sum a_k \frac{(x - x_0)^{k+1}}{k+1}$
        \item Ряды, полученные почленным дифференцированием и интегрированием имеют радиус сходимости $R$.
    \end{enumerate}
\end{theorem}