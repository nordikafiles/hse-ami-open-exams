\section{Теорема о сведении двойного интеграла к повторному (доказательство для прямоугольной области).}

\begin{theorem}
    Пусть $R = [a, b] \times [c, d]$.
    Если $f$ интегрируема на $R$ и для любого $\widetilde{x} \in [a, b]$ существует $I(\widetilde{x}) = \int \limits_c^d f(\widetilde{x}, y) dy$, тогда существует интеграл
    \[
        \int \limits_a^b \left(
            \int \limits_c^d f(x,y) dy
        \right) dx =
        \iint \limits_R f(x,y) dx dy
    \]
    \begin{proof}
        Разобъем прямоугольник $R$ точками
        \[
            a = x_0 < x_1 < ... < x_n = b,
            \quad
            \Delta x_k = x_k - x_{k-1};
            \quad
            c = y_0 < y_1 < ... < y_n = d,
            \quad
            \Delta y_l = y_l - y_{l-1};
        \] на $n \cdot m$ прямоугольников $R_{kl} = [x_{k-1}, x_k] \times [y_{l-1}, y_l], \Delta R_{kl} = \Delta x_k \cdot \Delta y_l$.
        Пусть $m_{kl} = \inf_{R_kl} f(x,y), M_{kl} = \sup_{R_kl} f(x,y)$. Тогда
        \[
            m_{kl} \leqslant f(x,y) \leqslant M_{kl}
            \quad
            \forall (x,y) \in R_kl.
        \]
        Зафиксируем $x = \xi_k \in [x_{k-1}, x_k]$ и проинтегрируем по y на $[y_{l-1}, y_l]$:
        \[
            m_{kl} \Delta y_l
            \leqslant
            \int \limits_{y_{l-1}}^{y_l} f(\xi_k, y) dy
            \leqslant
            M_{kl} \Delta y_l
        \]
        Домножим далее на $\Delta x_k$ и просуммируем полученные неравенства по $l$ от 1 до $m$, а затем по $k$ от 1 до $n$. Имеем:
        \[
            \underline{S}_\tau (f) =
            \sum_{k=1}^n \sum_{l=1}^m m_{kl} \Delta x_k \Delta y_l
            \leqslant
            \sum_{k=1}^n I(\xi_k) \Delta x_k
            \leqslant
            \overline{S}_\tau (f) =
            \sum_{k=1}^n \sum_{l=1}^m M_{kl} \Delta x_k \Delta y_l.
        \]
        Устремим диаметр разбиения к 0, получаем, в силу интегрируемости функции $f$, что суммы Дарбу стремятся к двойному интегралу. Значит, что предел среднего члена в данном выше неравенстве равен как двойному, так и повторному интегралу. 
    \end{proof}
\end{theorem}