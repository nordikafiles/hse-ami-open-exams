\section{Признак Даламбера в простой и предельной формах. Примеры.}

\begin{theorem}[признак Даламбера в допредельной форме]
    Если $\forall k \in \mathbb{N}$ выполнено
    \[
        \frac{p_{k+1}}{p_{k}} \leqslant q < 1
        \left(
            \frac{p_{k+1}}{p_{k}} \geqslant 1
        \right)
        ,
    \] то ряд $\sum p_k$ сходится (расходится).
    \begin{proof}
        Положим $p'_k = q^k$. Тогда
        \[
            \frac{p'_{k+1}}{p'_k} = q < 1
            \left(
                \frac{p'_{k+1}}{p'_k} = 1
            \right)
        \]
        \[
            \frac{p_{k+1}}{p_{k}} \leqslant
            \frac{p'_{k+1}}{p'_{k}}
            \left(
                \frac{p_{k+1}}{p_{k}} \geqslant
                \frac{p'_{k+1}}{p'_{k}}
            \right)
        \]
        Но теперь, учитывая тот факт, что ряд $\sum_{k=1}^{\infty} p'_k$ сходится (расходится) и, пользуясь первым признаком сравнения (теорема \ref{first_comp}), делаем вывод, что ряд $\sum_{k=1}^{\infty} p_k$ сходится (расходится).
    \end{proof}
\end{theorem}

\begin{theorem}[признак Даламбера в предельной форме]
    Пусть существует
    \[
        \lim_{k \to \infty} \frac{p_{k+1}}{p_{k}} = L
    \]
    Тогда при $L < 1$ ряд $\sum p_k$ сходится, при $L > 1$ расходится, а при $L = 1$ может как сходиться, так и расходиться.
    \begin{proof}
        Как мы знаем,
        \[
            \lim_{k \to \infty} \frac{p_{k+1}}{p_{k}} = L
        \]
        Это означает, что $\forall \varepsilon > 0 \exists N(\varepsilon) \forall k \geqslant N$ выполняется
        \[
            L - \varepsilon < \frac{p_{k+1}}{p_{k}} < L + \varepsilon
        \]
        Теперь если $L > 1$, то мы можем выбрать такое $\epsilon$, что $L + 2 \varepsilon = 1 \Leftrightarrow L + \varepsilon = 1 - \varepsilon$. Но тогда
        \[
            \frac{p_{k+1}}{p_k} < L + \varepsilon < 1
        \]
        Тем самым получили допредельный вариант теоремы, из которого следует, что ряд $\sum p_k$ сходится.
        \newline
        Пусть теперь $L > 1$. Выберем такое $\varepsilon$, что $L - \varepsilon = 1$. Получаем
        \[
            \frac{p_{k+1}}{p_k} > L - \epsilon = 1
        \]
        Снова получили допредельный вариант теоремы, из которого следует, что ряд $\sum p_k$ расходится.
        \newline
        Наконец, рассмотрим ряды $\sum \frac{1}{k}$ и $\sum \frac{1}{k^2}$. В обоих случаях $L=1$, но ряд $\sum \frac{1}{k}$ расходится, а ряд $\sum \frac{1}{k^2}$ сходится.
    \end{proof}
\end{theorem}

\subsection{Примеры.}
\begin{enumerate}
    \item $\sum \frac{1}{n!}$ -- сходится
    \item $\sum n!$ -- расходится
\end{enumerate}