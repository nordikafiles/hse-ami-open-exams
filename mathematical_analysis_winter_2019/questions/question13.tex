\section{Теорема Римана о перестановке членов условно сходящегося ряда, идея доказательства.}

\begin{lemma}\label{enc}
    Если $\sum a_k$ сходится условно, то $\sum a^{+}$ и $\sum a^{-}$ расходятся.
    \begin{proof}
        Пусть $a_k = a_k^+ + a_k^-$. Допустим, что один из $\sum a^{+}$ или $\sum a^{-}$ сходится. Тогда сходится и второй (т.к. сходится сумма). Тогда
        \[
            \sum |a_k| = \sum a^{+} - \sum a^{-}
        \] 
        тоже сходится. Противоречие с условной сходимостью.
    \end{proof}
\end{lemma}

\begin{theorem}
    Какого бы ни было число $L \in \mathbb{R}$, члены условно сходящегося ряда $\sum u_n$ можно переставить так, чтобы его сумма стала равной $L$.
    \begin{proof}
        Пусть для определенности $L > 0$. Приведем следующий алгоритм:
        \begin{enumerate}
            \item Будем добавлять неиспользованные положительные члены ряда до тех пор пока сумма не станет больше $L$. Это всегда возможно по лемме \ref{enc}.
            \item Будем добавлять неиспользованные отрицательные члены ряда до тех пор пока сумма не станет меньше $L$. Это всегда возможно по лемме \ref{enc}.
            \item Вернемся к первому шагу.
        \end{enumerate}
        Таким образом, полученный ряд сходится к $L$.
    \end{proof}
\end{theorem}