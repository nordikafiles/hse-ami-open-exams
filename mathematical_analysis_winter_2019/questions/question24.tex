\section{Единственность разложения в ряд для аналитической функции. Ряд Тейлора.}

\subsection{Единственность разложения в ряд для аналитической функции.}
\begin{theorem}
    Пусть $f$ -- аналитическая функция. Тогда ее предствление в виде $$f(x) = \sum_{k=0}^{\infty} a_k (x - x_0)^k$$ единственно. Более того, 
    \[
        a_k = \frac{f^{(k)}(x_0)}{k!}
        \quad
        \forall k = 0,1,2,...
    \]
    \begin{proof}
        Из леммы ? и теоремы о почленном дифференцировании ряда имеем
        \[
            f^{(n)}(x) = \sum_{k=0}^{\infty} (a_k (x - x_0)^k)^{(n)}
        \]
        \[
            f^{(n)}(x_0) = n! a_n \Rightarrow a_n = \frac{f^{(n)}(x_0)}{n!}
        \]
    \end{proof}
\end{theorem}

\subsection{Ряд Тейлора.}
\begin{definition}
    Ряд
    \[
        \sum_{k=0}^{\infty} a_k (x - x_0)^k, \text{ где }
        a_k = \frac{f^{(k)}(x_0)}{k!}
    \] называется рядом Тейлора функции $f$ в точке $x_0$.
\end{definition}