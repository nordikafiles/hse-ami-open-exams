\section{Определение степенного ряда, его радиуса и круга сходимости (формула Коши-Адамара). Докажите, что степенной ряд поточечно сходится строго внутри круга сходимости, и расходится строго вне круга сходимости.}


\subsection{Определение степенного ряда, его радиуса и круга сходимости (формула Коши-Адамара).}
\begin{definition}
    Функциональный ряд $\sum a_n (z - z_0)^n$ (где $a_n, z_0 \in \mathbb{C}$) называется степенным рядом.
\end{definition}
\begin{definition}[Формула Коши-Адамара]
    Радиус сходимости ряда -- это число $$R = \frac{1}{\lim_{n \to \infty} \sqrt[n]{|a_n|}}$$ (число или $+ \infty$)
\end{definition}
\begin{definition}
    Круг сходимости ряда -- это $\{z \in \mathbb{C} \> | \> |z - z_0| < R\}$.
    Нас интересует, при каких $z$ сходится $\sum a_n (z - z_0)^n$. Сделав замену $z := z - z_0$, сведем вопрос к $\sum a_n z^n$.
\end{definition}

\subsection{Докажите, что степенной ряд поточечно сходится строго внутри круга сходимости, и расходится строго вне круга сходимости.}
\begin{theorem}\label{sumconvr}
    Пусть $R$ -- радиус сходимости $\sum a_n z^n$. Тогда
    \begin{enumerate}
        \item При $|z| < R$ ряд сходится, причем абсолютно.
        \item При $|z| > R$ ряд расходится и даже его общий член не стремится к 0
        \item При $|z| = R$ всякое бывает
    \end{enumerate}
    \begin{proof}
        Применим признак Коши к ряду $\sum |a_n z^n| = \sum |a_n| \cdot |z^n|$.
        \[
            l = \lim_{n \to \infty} \sqrt[n]{|a_n| \cdot |z|^n} = |z| \cdot \lim_{n \to \infty} \sqrt[n]{|a_n|} = \frac{|z|}{R}
        \]
    \end{proof}
\end{theorem}