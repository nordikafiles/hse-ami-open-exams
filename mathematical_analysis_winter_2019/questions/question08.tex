\section{Определение перестановки членов ряда. Теорема о перестановке членов абсолютно сходящегося ряда (без доказательства). Теорема о произведении двух абсолютно сходящихся рядов.}

\subsection{Определение перестановки членов ряда.}
\begin{definition}
    Говорят, что два ряда $\sum a_n$ и $\sum b_n$ получаются друг из друга перестановкой членов, если существует такое взаимо-однозначное отображение $\varphi$ множества $\mathbb{N}$ натуральных чисел на себя, что $b_n = a_{\varphi(n)}$.
\end{definition}

\subsection{Теорема о перестановке членов абсолютно сходящегося ряда (без доказательства).} \label{transp_theorem}
\begin{theorem}
    Если числовой ряд $\sum u_k$ сходится абсолютно, то любая его перестановка членов сходится к той же самой сумме.
\end{theorem}

\subsection{Теорема о произведении двух абсолютно сходящихся рядов.}
\begin{theorem}
    Если $\sum u_k$ и $\sum v_k$ сходятся абсолютно к $u$ и $v$ соответственно, то ряд $\sum w_k$, составленный из всевозможных произведений $u_i \cdot v_j$ сходится абсолютно к $u \cdot v$.
    \begin{proof}
        Докажем сначала, что ряд $\sum w_k$ сходится абсолютно. Возьмем произвольное $n_0$ и рассмотрим $\sum_{k=1}^{n_0} |w_k|$. Эта сумма состоит из членов вида $|u_iv_j|$. Найдем среди этих индексов $i$ и $j$ наибольший индекс $m$, входящий в исследуемую сумму. Тогда
        \[
            \sum_{k=1}^{n_0} |w_k| \leqslant (|u_1| + ... + |u_m|) \cdot (|v_1| + ... + |v_m|)
            \leqslant M_1 M_2
        \]
        \[
            \begin{rcases}
                (|u_1| + ... + |u_m|) \leqslant M_1 \\
                (|v_1| + ... + |v_m|) \leqslant M_2 \\
            \end{rcases}
            \Rightarrow
            \sum_{k=1}^{n_0} |w_k| \leqslant M_1 M_2
        \]
        Ограничения $M_1$ и $M_2$ следуют из абсолютной сходимости рядов $\sum u_k$ и $\sum v_k$. Мы ограничили $n_0$-ую частичную сумму исследуемого ряда $\sum |w_k|$, значит этот ряд сходится. Осталось лишь доказать, что он сходится к $uv$.
        \newline
        Пусть данный ряд сходится к $S$. Заметим, что в силу теоремы \ref{transp_theorem} мы можем как угодно переставлять члены ряда $w_i$, не влияя на сходимость. Иными словами, любая последовательность или подпоследовательность частичный сумм будет сходиться к $S$. Тогда рассмотрим последовательность частичных сумм $\{S_{m^2}\}$, где $S_{m^2} = (u_1 + ... + u_m) \cdot (v_1 + ... + v_m)$. Но
        \[
            \begin{rcases}
                \lim_{m \to \infty} (u_1 + ... + u_m) = u \\
                \lim_{m \to \infty} (v_1 + ... + v_m) = v \\
            \end{rcases}
            \Rightarrow
            S_{m^2} \to uv
        \]
    \end{proof}
\end{theorem}