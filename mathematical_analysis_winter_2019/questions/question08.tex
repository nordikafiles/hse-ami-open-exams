\section{Определение перестановки членов ряда. Теорема о перестановке членов абсолютно сходящегося ряда.}

\subsection{Определение перестановки членов ряда.}
\begin{definition}
    Говорят, что два ряда $\sum a_n$ и $\sum b_n$ получаются друг из друга перестановкой членов, если существует такое взаимо-однозначное отображение $\varphi$ множества $\mathbb{N}$ натуральных чисел на себя, что $b_n = a_{\varphi(n)}$.
\end{definition}

\subsection{Теорема о перестановке членов абсолютно сходящегося ряда.} \label{transp_theorem}
\begin{theorem}
    Если числовой ряд $\sum u_k$ сходится абсолютно, то любая его перестановка членов сходится к той же самой сумме.
    \begin{proof}
        Пусть $\sum u_k$ абсолютно сходится к $S$, а $\sum u'_k$ -- некоторая перестановка членов исходного ряда. Требуется доказать, что $\sum u'_k = S$ и $\sum u'_k$ сходится абсолютно. Докажем сначала первое утверждение. Для этого достаточно доказать, что
        \[
            \forall \varepsilon > 0
            \exists N(\varepsilon)
            \forall n \geqslant N
            \left| 
                \sum_{k=1}^n u'_k - S 
            \right| < \varepsilon.
        \]
        Зафиксируем произвольное $\varepsilon$. Поскольку ряд $\sum u_k$ сходится абсолютно, то по признаку Коши
        \[
            \exists N'_0
            \forall p \in \mathbb{N}
            \sum_{k=N'_0 + 1}^{N'_0 + p}
            |u_k| < \frac{\varepsilon}{2},
        \]
        а по определению сходимости ряда
        \[
            \exists N''_0
            \left|
                \sum_{k=1}^{N''_0} u_k - S
            \right| \leqslant \frac{\varepsilon}{2}.
        \]
        Напоминаем, что данные неравенства по определениям выполняются и для $n \geqslant N'_0, N''_0$. Примем $N_0 = \max \{ N'_0, N''_0 \}$, чтобы для этого номера выполнялись оба неравенства. Теперь возьмем такое $N$, чтобы любая частичная сумма $S'_n$ ряда $\sum u'_k$ при $n \geqslant N$ содержала все первые $N_0$ членов ряда $\sum u_k$. Заметим, что такое $N$ всегда можно выбрать, поскольку мы просто переставили некоторые члены исходного ряда.
        \newline
        Оценим теперь разность
        \[
            \left| 
                \sum_{k=1}^n u'_k - S 
            \right| < \varepsilon.
        \]
        Пусть $n \geqslant N$. Указанную разность можно перезаписать в виде
        \[
            \sum u'_k - S
            =
            \left(
                \sum_{k=1}^n u'_k
                -
                \sum_{k=1}^{N_0} u_k
            \right)
            +
            \left(
                \sum_{k=1}^{N_0} u_k
                -
                S
            \right).
        \]
        Переходя к модулям, получаем
        \[
            \left|
                \sum u'_k - S
            \right|
            \leqslant
            \left|
                \sum_{k=1}^n u'_k
                -
                \sum_{k=1}^{N_0} u_k
            \right|
            +
            \left|
                \sum_{k=1}^{N_0} u_k
                -
                S
            \right| .
        \]
        Если воспользоваться неравенством $\left|
            \sum_{k=1}^{N''_0} u_k - S
        \right| \leqslant \frac{\varepsilon}{2}$, то достаточно доказать, что
        \[
            \left|
                \sum_{k=1}^n u'_k
                -
                \sum_{k=1}^{N_0} u_k
            \right| < \frac{\varepsilon}{2}.
        \]
        Вспомним теперь, что мы таким образом выбрали $N$, что при $n \geqslant N$ первая из сумм содержит все $N_0$ членов второй суммы. Поэтому указанная выше разность представляет собой сумму $n - N_0$ членов ряда $\sum u_k$ с номерами, каждый из которых превосходит $N_0$.
        \newline
        Тогда выберем такое $p$, чтобы номер $N_0 + p$ превосходил номера всех $n - N_0$ членов только что указанной суммы. Тогда справедливо
        \[
            \left|
                \sum_{k=1}^{n} u'_k
                -
                \sum_{k=1}^{n} u_k
            \right|
            \leqslant
            \sum_{k = N_0 + 1}^{N_0 + p} |u_k|
        \]
        Но теперь, пользуясь неравенством
        \[
            \left|
                \sum_{k=1}^{N''_0} u_k - S
            \right| \leqslant \frac{\varepsilon}{2},
        \]
        получаем то, что и требовалось доказать. Таким образом, мы доказали, что ряд $\sum u'_k$ сходится к $S$. Осталось лишь доказать, что он сходится абсолютно. Для этого достаточно применить приведенное выше доказательство для рядов $\sum |u_k|$ и $\sum |u'_k|$.

    \end{proof}
\end{theorem}