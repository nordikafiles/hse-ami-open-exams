\section{Интегральный признак сходимости числового ряда. Сходимость ряда $\sum_{k=1}^{\infty} \frac{1}{k^\alpha \ln^{\beta} k}$ в зависимости от значений $\alpha$ и $\beta$.}

\subsection{Интегральный признак сходимости числового ряда.}
\begin{theorem}
    Пусть при любом $k \in [1, +\infty)$ выполняется $f(k) \geqslant 0$, причем $f(k) \searrow 0$. Тогда сходимость ряда $\sum_{k=1}^{\infty} f(k)$ эквивалентна сходимости несобственного интеграла $\int \limits_{1}^{\infty} f(x) dx$.
    \begin{proof}
        При $x \in [k, k + 1]$, в силу $f(x) \searrow$, имеем $f(k + 1) \leqslant f(x) \leqslant f(k)$. Возьмем определенный интеграл от всех частей неравенства:
        \[
            \int \limits_k^{k+1} f(k+1) dx
            \leqslant
            \int \limits_{k}^{k+1} f(x) dx
            \leqslant
            \int \limits_{k}^{k+1} f(k) dx
        \]
        \[
            f(k+1)
            \leqslant
            \int \limits_{k}^{k+1} f(x) dx
            \leqslant
            f(k)
        \]
        Просуммируем теперь это неравенство по всем $k$ от $1$ до $n$. Получаем
        \[
            \sum_{k=2}^{n + 1} f(k)
            \leqslant
            \int \limits_1^{n+1} f(x) dx
            \leqslant
            \sum_{k=1}^{n} f(k)
        \]
        Теперь, если ряд $\sum_{k=1}^{\infty}$ сходится, то из правой части неравенства следует, что сходится интеграл. Если же сходится интеграл, то из левой части неравенства вытекает, что сходится ряд. Аналогично с расходимостью.
    \end{proof}
\end{theorem}

\subsection{Сходимость ряда $\sum_{k=1}^{\infty} \frac{1}{k^\alpha \ln^{\beta} k}$ в зависимости от значений $\alpha$ и $\beta$. (TODO)}