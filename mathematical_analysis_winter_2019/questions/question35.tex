\section{Теорема о среднем для двойного интеграла (формулировка и доказательство).}

\begin{theorem}
    \text{} \newline
    \begin{enumerate}
        \item Если $f$ интегрируема на $A \subseteq R^2$ и если $\forall x \in A \> m \leqslant f(x,y) \leqslant M$, то $m \cdot S(A) \leqslant \iint \limits_{A} f(x,y) dx dy \leqslant M \cdot S(A)$
        \item Если $f$ -- непрерывна, множество $A$ связно, то
        \[
            \exists (x_0, y_0) \in A \> f(x_0, y_0) =
            \frac{
                \iint \limits_A f(x,y) dx dy
            }{ S(A) }
        \]
    \end{enumerate}
    \begin{proof}
        \text{} \newline
        \begin{enumerate}
            \item Просто навесить интеграл на данное неравенство.
            \item Из первого пункта следует, что
            \[
                m = \min_A f = f(x_1, y_1) \leqslant R = \frac{
                    \iint \limits_A f(x,y) dx dy
                }{ S(A) } \leqslant M = \max_A f = f(x_2, y_2).
            \]
            Так как множество связно, то существует непрерывная кривая $(\varphi(t), \psi(t))$, такая что $(\varphi(0), \psi(0)) = (x_1, y_1), (\varphi(1), \psi(1)) = (x_1, y_1)$. Рассмотрим функцию $g(t) = f(\varphi(t), \psi(t))$. Она непрерывна на отрезке $[0,1]$ и достигает минимума и максимума на концах. Значит, существует некоторая точка $c \in [0, 1]$ такая, что $f(\varphi(c), \psi(c)) = g(c) = R$ (по теореме о промежуточном значении).
        \end{enumerate}
    \end{proof}
\end{theorem}