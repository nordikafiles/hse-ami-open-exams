\documentclass{article}
\usepackage[T2A, T1]{fontenc}
\usepackage[utf8]{inputenc}
\usepackage[russian]{babel}
\usepackage{titling}
\usepackage{amsmath}
\usepackage{mathtools}
\usepackage{amsthm}
\usepackage{python}
% \usepackage{minted}
\usepackage{amssymb}
\usepackage{amsthm}
\usepackage{amsthm}
\usepackage{hyperref}
\usepackage{listings}
\usepackage{xcolor}

\usepackage{graphicx}
\graphicspath{ {./} }

\hypersetup{
    colorlinks,
    citecolor=black,
    filecolor=black,
    linkcolor=black,
    urlcolor=black
}
\usepackage{titlesec}
\usepackage[rightcaption]{sidecap}
\usepackage{wrapfig}
\titleformat*{\subsubsection}{\normalfont}

\makeatletter
\renewcommand*\env@matrix[1][*\c@MaxMatrixCols c]{%
  \hskip -\arraycolsep
  \let\@ifnextchar\new@ifnextchar
  \array{#1}}
\makeatother


\setlength{\droptitle}{-3.5cm}
\setlength{\parindent}{0cm}
\newcommand{\squad}{\hspace{0.5em}}
\renewcommand{\arraystretch}{1.5}

\newtheorem{theorem}{Теорема}
\newtheorem{lemma}{Лемма}
\newtheorem{definition}{Определение}
% \renewcommand{\sectionbreak}{}

\author{hse-ami-open-exams}
\title{Летний экзамен по алгебре}
\date{}
\usepackage
[
        a4paper,
        left=1.5cm,
        right=1.5cm,
        top=3cm,
        bottom=3cm
]
{geometry}

\begin{document}

\maketitle
\tableofcontents{}
\newpage

\section{Дискретное вероятностное пространство. Задача о разделе ставки. Вероятностный алгоритм проверки на простоту. Универсальная хеш-функция.}

\subsection{Дискретное вероятностное пространство.}
Пусть $\Omega$ -- непустое конечное \textit{множество элементарных исходов}.
\begin{definition}
    Всякое подмножество $A \subseteq \Omega$ называют \textbf{событием}.
\end{definition}
\begin{definition}
    Функцию $P : 2^{\Omega} \to [0, 1]$, удовлетворяющую следующим свойствам:
    \begin{itemize}
        \item $P(\Omega) = 1$
        \item $A \cap B = \varnothing \Rightarrow P(A \cup B) = P(A) + P(B)$
    \end{itemize}
    называют \textbf{вероятностной мерой}, а значение $P(A)$ \textbf{вероятностью события $A$}.
    Вероятностная мера полностью определяется значениями $P(\{\omega\}) = p_{\omega}$, т.е.
    \[
        P(A) = \sum_{\omega \in A} p_{\omega}
    \]
\end{definition}
Если все элементарные исходы равновозможны, то полагаем $p_{\omega_1} = ... = p_{\omega_n} = 1/n$.

\subsection{Задача о разделе ставки.}
Два человека играют в некоторую игру, причем у обоих шансы победить одинаковые. Они договорились, что тот, кто первым выиграет 6 партий, получит весь приз. Однако игра остановилась раньше, когда первый выиграл пять партий, а второй выиграл три партии. Как справедливо разделить приз?
\newline
Предлагается разделить приз в отношении, в котором относятся вероятности выиграть для каждого из игроков в случае продолжения игры. Ясно, что еще надо сыграть не более трех партий. Пространство исходов этих трех партий состоит из восьми элементов, причем только один из этих исходов означает выигрыш второго игрока. Значит приз надо разделить в отношении 7 к 1.

\subsection{Вероятностный алгоритм проверки на простоту.}
Пусть дано некоторое натуральное число $N > 1$. Если $N$ простое число, то по малой теореме Ферма для всякого натурального числа, такого, что $(b, N) = 1$, число $b^{N -1} - 1$ делится на $N$. Следовательно, если для некоторого $b$, удовлетворяющего условию $(b, N) = 1$, число $b^{N-1} - 1$ не делится на $N$, то $N$ не является простым. Это наблюдение используют для построения простейшего теста на простоту. Если $b^{N-1} - 1$  не делится на $N$, то говорим, что $N$ не проходит тест для основания $b$.
\newline
Пусть основание мы выбираем случайно из множества $\mathbb{Z}_N^*$. Предположим, что существует такое основание, для которого $N$ не проходит тест. Какова вероятность выбрать такое основание?
\newline
Предположим, что для $a \in \mathbb{Z}_N^*$ число $N$ не проходит тест. Если $N$ проходит тест для основания $b$, то для основания $ab$ число $N$ уже тест не проходит. В противном случае $(ab)^{N-1} \equiv_N 1$ и $(b^{-1})^{N-1} \equiv_N 1$. Следовательно, $a^{N-1} \equiv (b^{-1})^{N-1}(ab)^{N-1} \equiv 1$, что противоречит предположению. Таким образом, каждому основанию $b$, для которого $N$ проходит тест, можно сопоставить основания $ab$, для котрого результат теста отрицательный. Значит, оснований, для которых $N$ не проходит тест, не м еньше оснований, для которых $N$ проходит тест на простоту. Искомая вероятность не меньше $1/2$. Если независимым образом повторять набор основания $k$ раз, то вероятность выбрать основание, для которого данное число не проходит тест, меньше $1/2^k$.
\newpage

\subsection{Универсальная хеш-функция.}
Пусть $K = \{0,1,2,...,n-1\}$ -- множество <<ключей>>. Отображение
\[
h: K \to \{0,1,2,...,m-1\}
\]
называется хеш-функцией. Предполагается, что $m < n$. Одним из важнейших свойств функции $h$ является равномерность, когда доля ключей $k$ с фиксированным значением $h(k)$ должно быть примерно $n/m$.  Это означает, что вероятность коллизии $h(k_1) = h(k_2)$ при $k_1 \neq k_2$ не больше $1/m$. Ясно, что не всегда можно предполагать, что ключи равномерно распределены по таблице. Предположим, что $h(k) = k \mod{m}$ и на вход сначала подаются ключи вида $m, 2m, 3m, ...$ Ясно, что всем таким ключам присваивается хеш-код 0 и реально никакого равномерного распределения значений не происходит. Оказывается, с этой проблемой можно справиться, если перед началом хеширования случайным образом выбирать функцию $h$ из некоторого набора таких функций.
\newline
Зафиксируем простое число $p > n$. Пусть $a \in \{1,2,...,o-1\}$ и $b \in \{0,1,2,...,o-1\}$. Положим $$h_{a,b} = ak + b \mod{p} \mod{m}.$$
Пара параметров $(a,b)$ выбирается случайным образом из множества $$\{1,2,...,p-1\} \times \{0,1,2,...,p-1\},$$ причем все элементы этого множества считаем равновероятными. Докажем, что для любых $$k_1, k_2 \in \{0,2,...,n-1\}, \quad k_1 \neq k_2,$$ вероятность коллизии $h_{a,b}(k_1) = h_{a,b}(k_2)$ не превосходит $1/m$.
\newline
Заметим, что $ak_1 + b = ak_2 + b \mod{p}$ тогда и только тогда, когда $k_1 = k2$. Кроме того, для различных $k_1$ и $k_2$ по значениям $ak_1 + b \mod{p}$ и $ak_2 + b \mod{p}$ однозначно находятся числа $a$ и $b$. Пусть $k_1 \neq k_2$. Отображение
$$(a,b) \to (ak_1 + b \mod{p}, ak_2 + b \mod{p})$$ является биекцией множества
$$\{1,2,...,p-1\} \times \{0,1,2,...,p-1\}$$ на множество 
$$(\{0, 1,2,...,p-1\} \times \{0,1,2,...,p-1\}) \setminus \{(i, i) \> | \> 0 \leqslant i \leqslant p - 1 \}.$$ Остается отметить, что для всякого $t \in \{0,1,2,...,p-1\}$ количетсво чисел $s \in \{0,1,2,...,p-1\}$ таких, что $s \neq t$ и $s = t \mod{m}$, не превосходит $(p-1)/m$, т.е. вероятность выбора такой пары $(t,s)$ или, что эквивалентно, выбора пары $(a,b)$, у которой $h_{a,b}(k_1) = h_{a,b}(k_2)$, не превосходит $1/m$.
\section{Первообразная и неопределенный интеграл. Определение и основные правила интегрирования: метод замены переменной и метод интегрирования по частям.}
--
\newline
\section{Условная вероятность. Независимые события. Отличие попарной независимости и независимости в совокупности.}

\subsection{Условная вероятность.}

\subsection{Независимые события.}

\subsection{Отличие попарной независимости и независимости в совокупности.}

\section{Билет 4.}

\begin{theorem}
    Замкнутая ОНС полная.
    \begin{proof}
        Пусть $\forall k \> f \perp w_k$. Имеем $f_k  = (f, w_k) = 0$. Тогда из тождества Парсеваля следует
        \[
            ||f|| = \sum^{\infty} f_k = 0 \Rightarrow f = 0
        \]
        \end{proof}
\end{theorem}

\subsection{Бывают ли замкнутые ортонормированные системы, но не полные?}
Нет, так как из замкнутости следует полнота.
\section{Общее определение математического ожидания и его корректность. Математическое ожидание случайной величины, распределение которое задано плотностью.}

\subsection{Общее определение математического ожидания и его корректность.}

\begin{theorem}
    Для любой случайной величины $\xi$ существует последовательность случайных величин $\{\xi_n\}$ такая, что $ \xi_n \rightrightarrows \xi$ на $\Omega$.
    \begin{proof}
        Пусть $\xi_n = 10^{-n} \cdot \lfloor 10^n \cdot \xi \rfloor$. Тогда $\sup |\xi_n - \xi| \leqslant 10^{-n} \to 0$.
    \end{proof}
\end{theorem}

\begin{definition}
    Пусть множество значений $\{x_1, x_2, x_3, ...\}$ дискретной случайной величины $\xi$ бесконечно. Положим $A_i = \xi^{-1}(\{x_i\})$. Будем говорить, что у $\xi$ существует конечное математическое ожидание если ряд $E \xi = \sum_{k=1}^{\infty} x_k P(A_k)$ сходится абсолютно.
    \begin{proof}[Доказательство корректности.]
        Так как перестановка членов абсолютно сходящегося ряда не влияет на сходимость и сумму ряда, а произведение абсолютно сходящихся рядов сходится к произведению их сумм, то все свойства математического ожидания будут выполняться и для суммы этого ряда.
    \end{proof}
\end{definition}

\subsection{Математическое ожидание случайной величины, распределение которое задано плотностью.}
\begin{theorem}
    Пусть $\varphi$ -- кусочно-непрерывная функция на $\mathbb{R}$ и $\xi : \Omega \to \mathbb{R}$ -- случайная величина, распределение которой задано плотностью $\rho_\xi$, тогда 
    \[
        \exists E(\varphi(\xi))
        \Leftrightarrow
        \int \limits_{-\infty}^{\infty} |\varphi(x) \rho_\xi(x) | dx
        \text{ сходится. }
    \]
    В случае сходимости
    \[
        E(\varphi(\xi)) = \int \limits_{-\infty}^{\infty} \varphi(x) \rho_\xi(x) dx.
    \]
    \begin{proof}
        Докажем для кусочно-постоянных функций. Пусть $f$ -- кусочно-постоянная функция, а это значит, что $f(\xi)$ -- дискретная величина, тогда
        \[
            E(f(\xi)) =
            \sum_n C_n P(A_n) =
            \sum_n C_n \int \limits_{\Delta n} \rho_\xi(x) dx =
            \sum_n \int \limits_{\Delta n} f(x) \rho_\xi(x) dx =
            \int \limits_{-\infty}^{+\infty} f(x) \rho_\xi(x) dx.
        \]
        В общем случае мы можем разбивать числовую прямую на счетное число промежутков таким образом, чтобы каждое такое разбиение задавало кусочно-постоянную функцию $f_n(x)$, и при этом $f_n(\xi) \rightrightarrows \varphi(\xi)$, тогда получим то же утверждение для кусочно-непрерывной функции $\varphi(x)$.
    \end{proof}
\end{theorem}
\section{Случайное блуждание: принцип отражения, задача о баллотировке и задача о возвращении в начало координат. Броуновское движение.}

\subsection{Случайное блуждание}
Схема Бернулли имеет красивую геометрическую интерпретацию.
По числовой прямой двигается частица, которая каждую секунду перемешается на единицу вправо или на единицу влево, причем выбор обоих направлений равновозможен и не зависит от соответствующего выбора на других шагах. Мы считаем, что в начальный момент времени частица находится в точке $x = 0$. Ясно, что траекторию движения частицы за $N$ перемещений можно закодировать последовательностью из 1 или -1 длины $N$. Набор таких последовательностей -- пространство элементарных исходов. Вероятность каждой траектории равна $2^{-N}$. Таким образом, с точностью до обозначений мы получили схему Бернулли, описывающую бросание правильной монеты.
\newline
При исследовании случайного блуждания, нас интересует вероятность того, что траектория обладает некоторым свойством. Какова вероятность того, что частица не возвращается в начало координат? Какова вероятность, что $N$-м шаге частица первый раз вернулась в начало координат?
\newline
Траектории частицы изображаем на координатной плоскости переменных $(t, x)$ в виде ломанных, соединяющих точки с целочисленными координатами $t$ и $x$. Здесь $x$ -- положение частицы, а $t$ -- время.
\subsection{Принцип отражения}
\begin{theorem}
    Пусть $x_0 > 0, x_1 > 0$ и $t_0 < t_1$. Число путей из $(t_0, x_0)$ в $(t_1, x_1)$, которые касаются или пересекают ось времени, равно числу путей из $(t_0, -x_0)$ в $(t_1, x_1)$.
    \begin{proof}
        Очевидно, что существует биекция.
    \end{proof}
\end{theorem}

\subsection{Задача о баллотировке}
Какова вероятность того, что частица, которая вышла из нуля и пришла в точку $k > 0$ за $N$ шагов, все время пути находилась в точках с положительными координатами? Обратим внимание, что требуется вычислить условную вероятность, где условием является то, что частица за $N$ шагов пришла в точку $k$. Следовательно, надо среди таких путей найти долю тех, которые проходят только через точки с положительными координатами.
\newline
Рассматриваемая задача иеет интересную интерпретацию и называется <<теоремой о баллотировке>>. Если на выборах один кандидат набрал $q$ голосов, а другой $r$ голосов и $r > q$, то какова вероятность того, что победивший кандидат все время выборов был впереди? Предполагается, что голосовавшие не имели предпочтений и отдавали свой голос случайно, а подсчет голосов происходил последовательно.
\newline
В первый момент времени частица с вероятностью $1/2$ перемешается в точку $x = 1$ или $x = -1$. Нас устраивает только первый вариант. Затем, нужная нам траектория частицы соединяет точки $(1,1)$ и $(N, k)$ и не касается и не пересекает ось времени. По принципу отражения мы умеем считать число остальных траекторий, соединяющих $(1,1)$ и $(N, k)$. Таких траекторий ровно столько, сколько всего траекторий из $(1, -1)$ в $(N, k)$. Несложно посчитать, что таких траекторий $C_{N-1}^{\frac{N+k}{2}}$. Всего траекторий из $(1,1)$ в $(N, k)$ равно $C_{N-1}^{\frac{N+k}{2} - 1}$. Следовательно, число нужных нам траекторий частицы
\[
    C_{N-1}^{\frac{N+k}{2} - 1} - C_{N-1}^{\frac{N+k}{2}} = \frac{k}{N} \cdot C_{N}^{\frac{N+k}{2}}.
\]
Здесь $C_{N}^{\frac{N+k}{2}}$ -- количество путей, соединяющих начало координат и точку $(N, k)$. Значит вероятность искомого события равна $\frac{k}{N}$. В условиях задачи о баллотировке соответствующая вероятность равна $\frac{r-q}{r+q}$.
\newpage

\subsection{Задача о возвращении в начало координат}
Пусть чатсица вышла из начала координат. Обозначим через $u_{2n}$ вероятность того, что в момент времени $t = 2n$ частица вернулась в точку $x = 0$, а через $f_{2n}$ вероятность того, чтро это произошло первый раз.
\newline
Легко посчитать, что $u_{2n} = C_{2n}^n \cdot 2^{-2n}$. Сложнее найти $f_{2n}$. Частица приходит в точку $x = 0$ в момент времени $2n$ из точек $x = 1$ или $x = -1$ в момент времени $t = 2n - 1$. Число путей в точку $(2n-1, 1)$ из начала координат таких, что все координаты точек, через которые проходит путь, положительные, равно $\frac{1}{2n -1} \cdot C_{2n-1}^n$. Столько же путей в точку $(2n-1, -1)$ из начала координат таких, что все координаты точек, через которые проходит путь, отрицательные. Следовательно, всего нужных нам путей $\frac{2}{2n-1} \cdot C_{2n-1}^n$ и
\[
    f_{2n} = \frac{2}{2n - 1} \cdot C_{2n-1}^n \cdot 2^{-2n}.
\]
Несложно проверить, что $f_{2n} = u_{2n-2} - u_{2n}$ и $f_{2n} = (2n)^{-1} \cdot u_{2n-2}$. Формула Стирлинга позволяет найти асимптотику этих вероятостей:
\[
    u_{2n} \sim \frac{1}{\sqrt{\pi n}},
    \quad
    f_{2n} \sim \frac{1}{2 \sqrt{\pi} \cdot n^{3/2}}.
\]

\subsection{Броуновское движение.}
Будем считать, что частица за время $\Delta t$ перемещается вправо или влево на $\Delta x$, где уже не предполагается, что $\Delta t$ и $\Delta x$ равны единице. Пусть в момент времени $t$ частица находится в точке $X(t)$. Хотим узнать распределение значений $X(t)$. Предположим, что $t = N \Delta t$. Если за эти $N$ перемещений частица $k$ раз перемещалась вправо, то
\[
    X(t) = k \Delta x + (N - k)(- \Delta x) = (2k - N) \Delta x.
\]
Из наблюдений известно, что $|\Delta x|^2 = \sigma \Delta t$ для некоторого числа $\sigma > 0$. Тогда
\[
    X(t) = \frac{k - \frac{N}{2}}{\sqrt{\frac{N}{4}}} \sqrt{t \sigma}.
\]
Для моделирования непрерывного движения частицы устремим $\Delta t$ к нулю. Это равносильно тому, что $N \to \infty$. По теореме Муавра-Лаплаа
\[
    \lim_{N \to \infty} P \left(
        a \leqslant X(t) \leqslant b
    \right) =
    \frac{1}{\sqrt{2\pi}} \int_{\frac{a}{\sqrt{t \sigma}}}^{\frac{b}{\sqrt{t \sigma}}} e^{-x^2 / 2} dx = 
    \int_a^b \frac{1}{\sqrt{2 \pi \sigma t}} e^{-\frac{x^2}{2 \sigma t}} dx.
\]
Таким образом, вероятность того, что частица в момент времени $t$ находится в $[a,b]$ вычисляется с помощью плотности $$\frac{1}{\sqrt{2\pi \sigma t}} e^{-\frac{x^2}{2 \sigma t}}.$$
\section{Абсолютно сходящиеся ряды. Докажите, что абсолютно сходящийся ряд сходится.}

\begin{definition}
    Будем говорить, что ряд $\sum u_k$ сходится абсолютно, если $\sum |u_k|$ сходится.
\end{definition}

\begin{theorem}
    Абсолютно сходящийся ряд сходится.
    \begin{proof}
        По критерию Коши имеем
        \[
            \forall \varepsilon > 0
            \exists N(\varepsilon)
            \forall n \geqslant N
            \forall p \in N
            \sum_{k=n+1}^{n+p} |u_k| < \varepsilon.
        \]
        Осталось лишь воспользоваться неравенством
        \[
            \left|
                \sum_{k=n+1}^{n+p} u_k
            \right|
            \leqslant
            \sum_{k=n+1}^{n+p} |u_k| < \varepsilon.
        \]
    \end{proof}
\end{theorem}
\section{Закон больших чисел в слабой форме. Метод Монте-Карло.}
\section{Прямое произведение групп. Разложение конечной циклической группы.}

\subsection{Прямое произведение групп.}
\begin{definition}
    \textbf{Прямым произведением} групп $G_1, ..., G_m$ называется множество
    \[
        G_1 \times ... \times G_m = \{ (g_1, ..., g_m) \> | \> g_1 \in G_1, ..., g_m \in G_m \}
    \]
    с операцией $(g_1, ..., g_m)(g'_1, ..., g'_m) = (g_1g'_1, ..., g_mg'_m)$.
    Ясно, что эта операция ассоциативна, обладает нейтральным элементом $(e_{G_1}, ..., e_{G_m})$ и для каждого элемента $(g_1, ..., g_m)$ есть обратный элемент $(g_1^{-1}, ..., g_m^{-1})$.
\end{definition}
\large \faYoutube \normalsize $\>$ \url{https://youtu.be/1oceAPu3b8o}

\subsection{Разложение конечной циклической группы.}
\begin{definition}
    Группа $G$ раскладывается в прямое произведение своих подгрупп $H_1, ..., H_m$, если отображение $H_1 \times ... \times H_m \to G, (h_1, ..., h_m) \mapsto h_1 \cdot ... \cdot h_m$ является изоморфизмом.
\end{definition}
\large \faYoutube \normalsize $\>$ \url{https://youtu.be/1oceAPu3b8o?t=293}
\begin{theorem}
    Пусть $n = ml$ -- разложение натурального числа $n$ на два взаимно простых множителя. Тогда имеет место изоморфизм групп
    \[
        \mathbb{Z}_n \simeq \mathbb{Z}_m \times \mathbb{Z}_l.
    \]
    \begin{proof}
        Рассмотрим отображение
        \[
            \varphi : \mathbb{Z}_n \to \mathbb{Z}_m \times \mathbb{Z}_l,
            \quad
            (k \mod n) \mapsto (k \mod m, k \mod l).
        \]
        Поскольку $m$ и $l$ делят $n$, отображение $\varphi$ определено корректно. Ясно, что $\varphi$ -- гомоморфизм.
        \newline
        Далее, $a \mod n \in \operatorname{Ker}(\varphi) \Rightarrow a \mod m = 0, a \mod l = 0 \Rightarrow a $ делится на $ m, a $ делится на $ k$.
        Так как НОД$(m, l) = 1$, то $a$ делится на $n = ml \Rightarrow a \mod n = 0 \Rightarrow \operatorname{Ker}(\varphi) = \{0\}$. Следовательно, гомоморфизм $\varphi$ инъективен. Поскольку множества $\mathbb{Z}_n$ и $\mathbb{Z}_m \times \mathbb{Z}_l$ содержат одинаковое число элементов, отображение $\varphi$ биективно.
    \end{proof}
\end{theorem}
\large \faYoutube \normalsize $\>$ \url{https://youtu.be/1oceAPu3b8o?t=585}
\begin{consequence} \label{rkzg}
    Пусть $n \geqslant 2$ -- натуральное число и $n = p_1^{k_1} ... p_s^{k_s}$ -- его разложение в произвежение простых множителей (где $p_i \neq p_j$ при $i \neq j$). Тогда имеет место изоморфизм групп
    \[
        \mathbb{Z}_n \simeq \mathbb{Z}_{p_1^{k_1}} \times ... \times \mathbb{Z}_{p_s^{k_s}}.
    \]
\end{consequence}
\large \faYoutube \normalsize $\>$ \url{https://youtu.be/1oceAPu3b8o?t=1061}
\section{Преобразование Абеля. Объясните, почему это преобразование является дискретным аналогом формулы интегрирования по частям.}

Пусть $B_n = \sum_{k=1}^n b_k$ и $B_0 = 0$. Тогда
\[
    \sum_{k=1}^n = a_n B_n - \sum_{k=1}^{n-1} B_k (a_{k+1} - a_k).
\]

Преобразование Абеля является дискретным аналогом интегрирования по частям. Для наглядности рассмотрим следующюю таблицу:
\begin{center}
    \begin{tabular}{ |c|c| }
        \hline
        $f$ & $\{a_n\}_{n=1}^{\infty}$ \\
        \hline
        $f'$ & $\{a_n - a_{n-1}\}_{n=2}^{\infty}$ \\
        \hline
        $\int \limits_a^b f(x) \> dx$ & $\sum_{k=1}^{\infty} a_k$ \\
        \hline
        $\left( \int \limits_a^x f(x) \> dx \right)'_x = f(x)$ & $\sum_{k=1}^{n} a_k - \sum_{k=1}^{n-1} a_k = a_n$ \\
        \hline
        $f,g,G = \int \limits_a^x g(t) \> dt + C$ & $\{a_k\}, \{b_k\}, \{B_k = \sum_{j=1}^k b_j + B_0 \}$ \\
        \hline
        $\int \limits_a^b f g dx = \int \limits_a^b f dG = \left. f \cdot G \right|_a^b - \int \limits_a^b G f' dx$ & $\sum_{k=1}^n a_k b_k = a_n B_n - a_1 B_0 - \sum_{k=1}^{n-1} (a_k+1 - a_k) B_k$ \\
        \hline
    \end{tabular}
\end{center}
\section{Билет 11.}

\subsection{Дайте определение интеграла Фурье и обратного преобразования Фурье.}
\begin{definition}
    Интегралом Фурье от функции $f : \mathbb{R} \to \mathbb{C}$ в точке $x$ называют
    \[
        \frac{1}{2\pi} v. p. \int_{-\infty}^{\infty} \hat{f}(y) e^{-ixy} dy
    \]
\end{definition}
\begin{definition}
    $g(y) \mapsto \tilde{g}(x) = \frac{1}{2\pi} v. p. \int_{-\infty}^{\infty} g(y) e^{-ixy} dy$ называют обратным преобразованием Фурье.
\end{definition}

\subsection{Сформулируйте и докажите теорему о свертке.}
\begin{theorem}
    Если $f, g \in L_1(\mathbb{R})$, то $F[f * g] = F[f] \cdot F[g]$, где $F$ -- преобразование Фурье.
    \begin{proof}
        \[
            F[f * g](y) = \int \limits_{-\infty}^{\infty} (f * g)(x) e^{ixy} dx =
            \int \limits_{-\infty}^{\infty} \left(
                \int \limits_{-\infty}^{\infty} f(t)g(x-t) dt
            \right)(x) e^{ixy} dx =
            \iint \limits_{\mathbb{R}^2} f(t) g(x-t) e^{ixy} dx dt =
        \]
        \[
            = \left[
                \begin{matrix}
                    \tilde{t} =t \\
                    \tilde{x} = x - t \\
                    dx dt = d \tilde{x} d \tilde{t} \\
                \end{matrix}
            \right]
            =
            \iint \limits_{\mathbb{R}^2} f(\tilde{t}) g(\tilde{x}) e^{i(\tilde{x} + t)y} d \tilde{x} d \tilde{t} =
            \int \limits_{-\infty}^{\infty} f(\tilde{t}) e^{i\tilde{t}y} d \tilde{t}
            \cdot
            \int \limits_{-\infty}^{\infty} g(\tilde{x}) e^{i\tilde{x}y} d \tilde{x}
            =
            F[f] \cdot F[g]
        \]
    \end{proof}
\end{theorem}

\subsection{Чему равно преобразование Фурье от произведения двух функций?}
\begin{statement}
    $F[f \cdot g] = \frac{1}{2\pi} F[f] * F[g]$.
    \begin{proof}
        \[
            F^{-1}[f * g] = 2 \pi F^{-1}[f] \cdot F^{-1}[g]
            \Rightarrow
            F^{-1}[F[f] * F[g]] = 2 \pi f \cdot g
            \Rightarrow
            F[f] * F[g] = 2 \pi F[f \cdot g]
            \Rightarrow
            F[f \cdot g] = \frac{1}{2\pi} F[f] * F[g]
        \]
    \end{proof}
\end{statement}
\section{Экспонента конечной абелевы группы и критерий цикличности.}
\begin{definition}
    Пусть $A$ -- конечная абелева группа. \textbf{Экспонента} группы $A$ -- это чиcло
    \[
        \exp(A) = \text{НОК}\{ \operatorname{ord}(a) \> | \> a \in A \} = min\{ n \in \mathbb{N} \> | \> na = 0 \> \forall \> a \in A \}
    \]
\end{definition}
\large \faYoutube \normalsize $\>$ \url{https://youtu.be/1oceAPu3b8o?t=3792}
\begin{statement}
    Пусть $A$ -- конечная абелева группа. Тогда $\exp(A) = |A| \Leftrightarrow A$ -- циклическая группа.
    \begin{proof}
        \begin{enumerate}
            \item[$\Leftarrow$] $A$ -- циклическая $\Rightarrow A \simeq \mathbb{Z}_n \Rightarrow \operatorname{ord}(a) = n = |A| \Rightarrow \exp(A) = |A|$ 
            \item[$\Rightarrow$] $\exp(A) = |A|$ Знаем, что $A \simeq T_{p_1}(A) \times ... \times T_{p_s}(A)$, где $|A| = p_1^{k_1} \cdot ... \cdot p_s^{k_s}$. Пусть $b_i \in T_{p_i}(A)$ -- элемент наибольшего порядка $\Rightarrow \operatorname{ord}(b_i) = p_i^{m_i}$. Тогда для любого $a_i \in T_{p_i}(A), ..., a_s \in T_{p_s}(A)$ получаем $\operatorname{ord}(a_i) = p_i^{l_i}$, где $l_i \leqslant m_i$. $\operatorname{ord}(a_1 + ... + a_s) = \operatorname{ord}(a_1) \cdot ... \cdot \operatorname{ord}(a_s)$ делит $\operatorname{ord}(b_1) \cdot ... \cdot \operatorname{ord}(b_s) = \operatorname{ord}(b_1 + ... + b_s)$. Следовательно, $\exp(A) = \operatorname{ord}(b_1 + ... + b_s) \Rightarrow |A| = \exp(A) = |\langle b_1 + ... + b_s \rangle| \Rightarrow \langle b_1 + ... + b_s \rangle = A \Rightarrow A$ -- циклическая группа.
        \end{enumerate}
    \end{proof}
\end{statement}
\large \faYoutube \normalsize $\>$ \url{https://youtu.be/1oceAPu3b8o?t=4028}
\section{Вывод из гипотез. Лемма о дедукции. Полезные производные правила.}

\subsection{Вывод из гипотез.}

\begin{definition}
  Пусть $\Gamma$ -- некоторое множество формул. Тогда выводом из множества посылок $\Gamma$ называется последовательность формул, каждая из которых является либо аксиомой, либо элементом $\Gamma$, либо выводится из более ранних формул по правилу modus ponens. Если формула $A$ встречается в некотором выводе из $\Gamma$, то она называется выводимой из $\Gamma$. Обозначение: $\Gamma \vdash A$.
\end{definition}

\subsection{Лемма о дедукции.}

\begin{lemma}
  $\vdash A \to A$.
  \begin{proof}
    Вывод состоит из 5 формул:
    \begin{enumerate}
      \item $A \to ((A \to A) \to A)$ (аксиома 1)
      \item $A \to (A \to A)$ (аксиома 1)
      \item $(A \to ((A \to A) \to A)) \to ((A \to (A \to A)) \to (A \to A))$ (аксиома 2)
      \item $(A \to (A \to A)) \to (A \to A)$ (modus ponens 1, 3)
      \item $A \to A$ (modus ponens 2, 4)
    \end{enumerate}
  \end{proof}
\end{lemma}

\begin{theorem}[Лемма о дедукции]
  Из множества посылок выводится импликация $A \to B$ тогда и только тогда, когда при добавлении $A$ к списку посылок выводится $B$. Иначе говоря, $\Gamma \vdash A \to B \Leftrightarrow \Gamma \cup \{A\} \vdash B$.
  \begin{proof}
    $\Rightarrow$ \newline
    Действительно, если к выводу импликации $A \to B$ из $\Gamma$ добавить формулы $A$ и $B$, то получится вывод $B$ из $\Gamma \cup \{A\}$: формулу $A$ можно написать как посылку, а $B$ -- по modus ponens. \newline
    $\Leftarrow$ \newline
    Будем доказывать по индукции такое утверждение: если $C_1, \dots, C_n$ есть вывод из $\Gamma \cup \{A\}$, то для всех $i$ импликация $A \to C_i$ выводима из $\Gamma$. Рассуждение по индукции будет опираться на такой принцип, объединяющий в себе и базу, и переход: если импликация $A \to C_1, \dots, A \to C_{i-1}$ выводимы, то и $A \to C_i$ тоже выводима. Разберем несколько случаев:
    \begin{itemize}
      \item $C_i$ является аксиомой. В таком случае импликация выводится в три шага: $C_i$; $C_i \to (A \to C_i)$ (аксиома 1); $A \to C_i$ (modus ponens предыдущих двух)
      \item $C_i$ является посылкой, т.е. элементом $\Gamma \cup \{A\}$.
        \begin{itemize}
          \item $C_i \in \Gamma$. Годится тот же вывод, что и для аксиомы.
          \item $C_i = A$. Импликация $A \to A$ выводится по лемме.
        \end{itemize}
      \item $C_i$ выводится по правилу modus ponens из формул $C_j$ и $C_k$ для некоторых $j,k < i$. В этом случае $C_k$ обязательно имеет вид $C_k \to C_i$, иначе modus ponens не применить. По предположению индукции $\Gamma \vdash C_j$ и $\Gamma \vdash C_k$, т.е. $\Gamma \vdash A \to (C_j \to C_i)$. Далее добавим в вывод вторую аксиому: $(A \to (C_j \to C_i)) \to ((A \to C_j) \to (A \to C_i))$ -- и двумя применениями modus ponens получим $A \to C_i$, что и требовалось.
    \end{itemize}
    Все случаи разобраны, поэтому индукционный принцин установлен и теорема доказана.
  \end{proof}
\end{theorem}

\subsection{Полезные производные правила.}

\begin{statement}
  Справедливы следующие правила вывода (каждый раз сверху от горизонтальной черты записаны условия теоремы, а снизу -- утверждение) -- следствия леммы о дедукции:
\end{statement}

\begin{enumerate}
  \item Правило вывода конъюнкции: $\frac{\Gamma \> \vdash \> A \quad \Gamma \> \vdash \> B}{\Gamma \> \vdash \> A \wedge B}$;
  \item Правило рассуждения от противного: $\frac{\Gamma, A \> \vdash \> B \quad \Gamma, A \> \vdash \> \neg B}{\Gamma \> \vdash \> \neg A}$;
  \item Правило разбора случаев: $\frac{\Gamma, A \> \vdash \> C \quad \Gamma, B \> \vdash \> C}{\Gamma, A \vee B \> \vdash \> C}$.
\end{enumerate}
\section{Определенный интеграл с переменным верхним (или нижним) концом интегрирования. Теорема о производной интеграла с переменным верхним концом интегрирования.}
--
\newline
\section{Исчисление резолюций для опровержения пропозициональных формул в КНФ: дизъюнкты, правило резолюций, опровержение КНФ в исчислении резолюций. Теорема корректности исчисления резолюций (для пропозициональных формул в КНФ).}
\section{Основные геометрические приложения определенного интеграла: площадь криволинейной трапеции (без доказательства), площадь криволинейного сектора (с доказательством), объем тела вращения (без доказательства).}
--
\newline
\section{Деление с остатком в кольце многочленов от одной переменной над полем. Теорема о том, что это кольцо является кольцом главных идеалов. \textcolor{orange}{(todo)}}

\subsection{Деление с остатком в кольце многочленов от одной переменной над полем.}
--
\subsection{Теорема о том, что это кольцо является кольцом главных идеалов.}
--
\section{Билет 18.}

\subsection{Допустим, что все точки множества $X \in \mathbb{R}^n$ удовлетворяют уравнению $f(x) = 0$. Докажите, что в любой точке $x^{(0)} \in X$ любой касательный вектор к $X$ перпендикулярен градиенту $\operatorname{grad} f(x^{(0)})$.}
\begin{proof}
    Пусть
    \[
        \forall x^{(0)} \subset \mathbb{R}^n
        \>
        f(x^{(0)}) = 0
        \Rightarrow
        \forall \text{ кривой }
        \{
            x_i = \varphi_i(s)
        \} \subset X
        \>
        f(\varphi_1(s), .., \varphi_n(s)) = 0
    \]
    Продифференцируем по $S$:
    \[
        \frac{\partial f}{ \partial x_1} \cdot \frac{\partial \varphi_1}{\partial S} +
        ... +
        \frac{\partial f}{ \partial x_n} \cdot \frac{\partial \varphi_n}{\partial S}
        = 0,
        \text{ то есть }
        \left<
            \operatorname{grad} f(x^{(0)}),
            \text{ касательный вектор к $X$ в точке $x^{(0)}$ }
        \right> = 0
        \Leftrightarrow
    \]
    \[
        \Leftrightarrow
        \text{ касательный вектор к $X$ в точке $x^{(0)}$ перпендикулярен }
        \operatorname{grad} f(x^{(0)})
    \]
\end{proof}

\subsection{Опишите касательное пространство к $k$-мерному подмногообразию в $\mathbb{R}^n$, заданному системой неявных уранвнений (б.д.).}
Пусть $m \leqslant n$ и $X$ задано
\[
    \begin{cases}
        f_1(x) = 0 \\
        \vdots \\
        f_m(x) = 0 \\
    \end{cases},
\]
и в $x^{(0)} \in X$
\[
    \operatorname{grad} f_1,
    ...,
    \operatorname{grad} f_m,
    \text{ линейно независимы.}
\]
Тогда
\[
    T_{x^{(0)}} X = \left<
        \operatorname{grad} f_1,
        ...,
        \operatorname{grad} f_m,
    \right>^{\perp}
    \text{ (ортогональное дополнение)}.
\]
\section{Лексикографический порядок на множестве одночленов от нескольких переменных. Лемма о конечности убывающих цепочек одночленов. \textcolor{orange}{(todo)}}

\subsection{Лексикографический порядок на множестве одночленов от нескольких переменных.}
--

\subsection{Лемма о конечности убывающих цепочек одночленов.}
--
\section{Дизъюнкты, универсальные дизъюнкты. Исчисление резолюций (ИР) для доказательства несовместности множеств универсальных дизъюнктов. Теорема корректности ИР.}
\section{Остаток многочлена относительно заданной системы многочленов. Системы Гребнера. Характеризация систем Гребнера в терминах цепочек элементарных редукций. \textcolor{orange}{(todo)}}

\subsection{Остаток многочлена относительно заданной системы многочленов.}
--

\subsection{Системы Гребнера.}
--

\subsection{Характеризация систем Гребнера в терминах цепочек элементарных редукций.}
--
\section{S-многочлены. Критерий Бухбергера. \textcolor{orange}{(todo)}}

\subsection{S-многочлены.}
--

\subsection{Критерий Бухбергера.}
--

\section{Лемма о сохранении радиуса сходимости при почленном дифференцировании степенного ряда. Теорема о почленном дифференцировании и интегрировании степенного ряда.}

\subsection{Лемма о сохранении радиуса сходимости при почленном дифференцировании степенного ряда.}

\subsection{Теорема о почленном дифференцировании и интегрировании степенного ряда.}
\section{Лемма о конечности цепочек одночленов, в которых каждый следующий одночлен не делится ни на один из предыдущих. Алгоритм Бухбергера построения базиса Гребнера идеала. \textcolor{orange}{(todo)}}

\subsection{Лемма о конечности цепочек одночленов, в которых каждый следующий одночлен не делится ни на один из предыдущих.}
--

\subsection{Алгоритм Бухбергера построения базиса Гребнера идеала.}
--

\section{Вычислите ряды Маклорена для функций $\frac{1}{1-x}$ и $\frac{1}{(1-x)^2}$ и докажите, что эти функции аналитичны в точке 0. Приведите пример неаналитической функции (б.д.).}

\subsection{Вычислите ряды Маклорена для функций $\frac{1}{1-x}$ и $\frac{1}{(1-x)^2}$ и докажите, что эти функции аналитичны в точке 0.}

\subsection{Приведите пример неаналитической функции (б.д.).}
\section{Игра Эренфойхта для данной пары моделей данной сигнатуры. Теорема об элементарной эквивалентности моделей, для которых в игре Эренфойхта Консерватор имеет выигрышную стратегию.}
\section{Семантически полные теории. Критерий семантической полноты теории в терминах элементарной эквивалентности моделей. Аксиоматизация элементарной теории упорядоченного множества рациональных чисел.}
\section{Семантически полные теории. Критерий семантической полноты теории в терминах элементарной эквивалентности моделей. Аксиоматизация элементарной теории упорядоченного множества целых чисел.}
\section{Докажите, что если ФМП интегрируема на множестве, то она ограничена на этом множестве.}
\begin{theorem}
    Если ФМП интегрируема на множестве, то она ограничена на этом множестве.
    \begin{proof}
        От противного.
        Интеграл 
        $I = \lim_{\delta \to 0} I(M_i, G_i)$.
        Пусть для определенности функция неограничена в области
        $G$,
        тогда она неограничена в какой-то области
        $G_j$.
        \[
            I(M_i, G_i) =
            \sum_i f(M_i) \delta S_i = f(M_j) \delta S_j + \sum_{i \neq j} f(M_i) \delta S_i
        \]
        Так как $f(M_j)$ можно делать сколь угожно большим, то не будет существовать предела. Следовательно, функция $f$ неинтегрируема на $G$.
    \end{proof}
\end{theorem}
\section{Подполе в расширении полей, порожденное алгебраическим элементом. \textcolor{orange}{(todo)}}

--
\section{Билет 31.}

\subsection{Объясните, что такое дифференциальная форма ранга $k$, и как вычисляется интеграл (2-го рода) от $k$-формы $\omega$ по $k$-мерному многообразию $\Omega \subset \mathbb{R}^n$.}
\begin{definition}
    \textit{Дифференциальной формой} ранга $k$ (или дифференциальной $k$-формой) на $M \subseteq \mathbb{R}^n$ называется выражение вида $\sum\limits_{\{i_1, \ldots, i_k\} \subseteq \{1, \ldots, n\}} f_{i_1 \ldots i_k}(x)\diff x_{i_1} \land \ldots \land \diff x_{i_k}$, где $f_{i_1\ldots i_k}$ ---~ некоторые дифференцируемые\footnote{Часто ограничиваются гладкими функциями.} функции $f_{i_1\ldots i_k}:M \to \mathbb{R}$.
\end{definition}

Если вам очень понравился предыдущий билет, можно сказать, что это сумма грассмановых мономов степени $k$ от переменных $\diff x_1, \ldots \diff x_n$ с дифференцируемыми функциями в качестве коэффициентов. Можно считать, что среди чисел $i_1, \ldots, i_k$ нет повторений, так как мономы с повторениями всё равно зануляются.

Пусть имеются $k$-мерное многообразие $\Omega \subseteq \mathbb{R}^n$ с параметризацией $\varphi: M \to \Omega, M \subseteq \mathbb{R}^k$ и дифференциальная $k$-форма $\omega = \sum\limits_{\{i_1, \ldots, i_k\} \subseteq \{1, \ldots, n\}} f_{i_1 \ldots i_k}(x)\diff x_{i_1} \land \ldots \land \diff x_{i_k}$ на $\Omega$. Определим интеграл (2-го рода) $\omega$ по $\Omega$.

\[ \int\limits_\Omega \omega := \int\limits_M \sum_{\{i_1, \ldots, i_k\} \subseteq \{1, \ldots, n\}} f_{i_1 \ldots i_k} (\varphi(t)) \diff \varphi_{i_1} \land \ldots \land \diff \varphi_{i_k}; \]

Поясним, что творится в этой формуле. Во-первых, $\varphi_i : M \to \mathbb{R}$ ---~ это функция, соответствующая $i$-й координате $\varphi$. Во-вторых, $\diff \varphi_i$ ---~ это привычный дифференциал функции нескольких переменных, но теперь мы говорим, что это линейная комбинация грассмановых переменных $\diff t_1, \ldots, \diff t_k$. Когда мы грассманово перемножим эти дифференциалы, у нас останется выражение вида $f(t) \diff t_1 \land \ldots  \land \diff t_k$. Это так, ведь в любом слагаемом результата будут перемножаться $k$ переменных, одинаковые занулятся, останутся только слагаемые с различными, возможно, не в том порядке. Но мы можем привести порядок к правильному. После этих преобразований мы считаем интеграл как обычный кратный интеграл.
\[\int\limits_M f \diff t_1 \land \ldots \land \diff t_k = \int\limits_M f \diff t_1 \ldots \diff t_k;  \]  

\subsection{Запишите вычислительную формулу для поверхностного интеграла 2-го рода.}
Для случая $k = 2$ это всё можно записать в следующую формулу.

\[\iint\limits_\Omega P\diff y \land \diff z + Q \diff z \land \diff x + R \diff x \land \diff y = \iint\limits_M \begin{vmatrix} P & Q & R \\ \frac{\partial \varphi_1}{\partial u} & \frac{\partial \varphi_2}{\partial u} & \frac{\partial \varphi_3}{\partial u} \\ \frac{\partial \varphi_1}{\partial v} & \frac{\partial \varphi_2}{\partial v} & \frac{\partial \varphi_3}{\partial v} \end{vmatrix} \diff u \diff v;\]

Обратите внимание, при $Q$ стоит $\diff z \land \diff x$, а не $\diff x \land \diff z$. 
\section{Билет 32.}

\subsection{Что такое ориентация $k$-мерного многообразия? Как изменится интеграл 2-го рода от дифференциальной формы при смене ориентации многообразия (б. д.)?}

Пусть $\Omega \subseteq \mathbb{R}^n$ ---~ $k$-мерное связное\footnote{Напомним, многообразие называется гладким, если любые две его точки можно соединить проходяще по нему непрерывной кривой.} многообразие, и у него имеются две параметризации $\varphi: M \to \Omega, \psi: N \to \Omega; M, N \subseteq \mathbb{R}^k$. Предположим, что функция замены координат $c = \varphi^{-1} \circ \psi$ биективна и непрерывно дифференцируема.

\[ \begin{tikzcd}
    M \arrow[swap]{rd}{\varphi} &  &\arrow[swap]{ll}{c} N \arrow{ld}{\psi} \\
    & \Omega
\end{tikzcd} \]

Посмотрим на якобиан $J(c)$. Если бы где-то он был равен нулю, в окрестности этой точки $c$ была бы необратима. Значит он не равен нулю нигде. Поскольку $J(c)$ непрерывен и $\Omega$ связно, из этого следует, что он имеет постоянный знак. Тогда если он положителен, будем говорить, что $\varphi$ и $\psi$ задают одну и ту же ориентацию, а если отрицателен ---~ то разные. Таким образом мы определяем ориентацию как отношение эквивалентности с двумя классами на параметризациях многообразия.

Ориентация задаёт ориентацию на любом касательном пространстве $T_{x}\Omega$ как на векторном пространстве. Если мы назвали ориентацию некоторой параметризации положительной, то назовём положительным базис $T_x \Omega$, полученный из её производных.

При смене параметризации на имеющую противоположную ориентацию интеграл 2-го рода меняет знак. 
\section{Докажите, что функция, непрерывная на компакте, интегрируема на нем (теорему Кантора нужно сформулировать, но не обязательно доказывать).}
\section{Билет 34.}

\subsection{Выведите из общей формулы Стокса частные случаи: формулу Ньютона-Лейбница, формулу Грина, формулу Гаусса-Остроградского.}

\begin{enumerate}

	\item \textbf{Формула Ньютона-Лейбница, $n = k = 1$} \newline

	Пусть наше многообразие это отрезок на прямой $[a; b]$. Его границей будет множество из двух точек $\{a, b\}$. Заметим, что точка ---~ это нульмерное многообразие и по нему можно интегрировать 0-формы (то есть просто функции). При чём этот интеграл будет с точностью до знака (знак как всегда определяется ориентацией) равен значению функции в точке. Если на отрезке мы берём стандартную ориентацию <<слева направо>>, то для границы это будет означать взятие $b$ с плюсом и $a$ с минусом. Итак, формула Стокса принимает следующий вид:
	
	\[F(b) - F(a) = \int\limits_a^b \diff F; \]
	
	Перепишем в более привычную запись.

	\[\int\limits_a^b F'(x) \diff x = F(b) - F(a); \]

	\item \textbf{Формула Грина, $n = k = 2$} \newline

	Дифференциальная 1-форма в $\mathbb{R}^2$ имеет вид $P\diff x + Q \diff y$. Посчитаем её дифференциал

	\[\diff (P \diff x + Q \diff y) = \left(\frac{\partial P}{\partial x}\diff x + \frac{\partial P}{\partial y}\diff y \right) \land \diff x +\left(\frac{\partial Q}{\partial x}\diff x + \frac{\partial Q}{\partial y}\diff y \right) \land \diff y = \frac{\partial P}{\partial y} \diff y \land \diff x + \frac{\partial Q}{\partial x} \diff x \land \diff y = \] \[ = \left(\frac{\partial Q}{\partial x} - \frac{\partial P}{\partial y}\right) \diff x \land \diff y;  \]

	Формула Стокса принимает следующий вид:

	\[\int\limits_{\partial U} P\diff x + Q\diff y = \iint\limits_{U} \left(\frac{\partial Q}{\partial x} - \frac{\partial P}{\partial y}\right)\diff x \diff y; \]

	Здесь условие согласованности ориентации можно сформулировать как <<при обходе $\partial U$ по заданной параметризации $U$ всегда находится слева>>.

	\item \textbf{Формула Гаусса-Остроградского, $n = k = 3$} \newline

	Дифференциальная 2-форма в $\mathbb{R}^3$ имеет вид $P\diff y \land \diff x + Q \diff z \land \diff x + R \diff x \land \diff y$. Посчитаем её дифференциал.

	\[\diff (P\diff y \land \diff z + Q \diff z \land \diff x + R \diff x \land \diff y) = \frac{\partial P}{\partial x} \diff x \land \diff y \land \diff z + \frac{\partial Q}{\partial y}\diff y \land \diff z \land \diff x + \frac{\partial R}{\partial z} \diff z \land \diff x \land \diff y  = \] \[ =\left( \frac{\partial P}{\partial x} + \frac{\partial Q}{\partial y} + \frac{\partial R}{\partial z} \right) \diff x \land \diff y \land \diff z; \]

	Формула Стокса принимает следующий вид:

	\[\iint\limits_{\partial V} P\diff y \land \diff x + Q \diff z \land \diff x + R \diff x \land \diff y = \iiint\limits_{V}\left( \frac{\partial P}{\partial x} + \frac{\partial Q}{\partial y} + \frac{\partial R}{\partial z} \right) \diff x\diff y\diff z; \]
\end{enumerate}

\section{Теорема о среднем для двойного интеграла (формулировка и доказательство).}
\section{Линейные коды. Проверочная матрица. Связь минимального расстояния линейного кода с его проверочной матрицей. Бинарный код Хэмминга, его минимальное расстояние и число ошибок, которое он может исправлять. \textcolor{orange}{(todo)}}

\subsection{Линейные коды.}
--

\subsection{Проверочная матрица.}
--

\subsection{Связь минимального расстояния линейного кода с его проверочной матрицей.}
--

\subsection{Бинарный код Хэмминга, его минимальное расстояние и число ошибок, которое он может исправлять.}
--

\section{Теорема о сведении двойного интеграла к повторному (доказательство для произвольной области, можно пользоваться соответствующей теоремой для прямоугольной области и теоремой Лебега).}

\begin{theorem}
    Пусть $\Omega$ -- элементарное относительно оси $O_x$ множество, функция $f$ интегрируема на $\Omega$ и при $\forall x \in [a, b]$ существует интеграл
    \[
        \int \limits_{\varphi(x)}^{\psi(x)} f(x,y) dy.
    \]
    Тогда существует повторный интеграл
    \[
        \int \limits_a^b
        \left(
            \int \limits_{\varphi(x)}^{\psi(x)} f(x,y) dy.
        \right) dx,
    \] причем
    \[
        \int \limits_a^b
        \left(
            \int \limits_{\varphi(x)}^{\psi(x)} f(x,y) dy.
        \right) dx
        =
        \iint \limits_{\Omega} f(x,y) dx dy.
    \]
    \begin{proof}
        Пусть $R$ прямоугольник со сторонами, параллельными осям координат, содержащий область $\Omega$ и
        \[
            F(x,y) = f(x,y) \cdot \chi_\Omega (x,y).
        \]
        Применяя предыдущую теорему к функции $F$, получаем искомую формулу.
    \end{proof}
\end{theorem}

\end{document}

