\section{Бинарные операции. Полугруппы, моноиды и группы. Коммутативные группы. Примеры групп. Порядок группы. Описание всех подгрупп в группе $(\mathbb{Z}, +)$.}



\subsection{Бинарные операции.}
\begin{definition}
    Множество с бинарной операцией -- это множество $M$ с заданным отображением
    \[
        M \times M \to M,
        \quad
        (a, b) \mapsto a \circ b.
    \]
    Множество с бинарной операцией обычно обозначают $(M, \circ)$.
\end{definition}

\subsection{Полугруппы, моноиды и группы.}
\begin{definition}
    Множество с бинарной операцией $(M, \circ)$ называется \textbf{полугруппой}, если данная бинарная операция \textbf{ассоциативна}, т.е.
    \[
        a \circ (b \circ c) = (a \circ b) \circ c
        \quad
        \text{ для всех $a,b,c \in M$. }
    \]
\end{definition}
\begin{definition}
    Полугруппа $(S, \circ)$ называется \textbf{моноидом}, если в ней есть нейтральный элемент, т.е. такой элемент $e \in S$, что $e \circ a = a \circ e = a$ для любого $a \in S$.
\end{definition}
\begin{definition}
    Моноид $(S, \circ)$ называется \textbf{группой}, если для каждого элемента $a \in S$ найдется обратный элемент, т.е. такой $b \in S$, что $a \circ b = b \circ a = e$.
\end{definition}

\subsection{Коммутативные группы.}
\begin{definition}
    Группа $(G, \circ)$ называется \textbf{коммутативной} или \textbf{абелевой}, если групповая операция коммутативна, т.е. $a \circ b = b \circ a$ для любых $a, b \in G$.
\end{definition}

\subsection{Примеры групп.}
\begin{enumerate}
    \item Числовые аддитивные группы: $(\mathbb{Z}, +), (\mathbb{Q}, +), (\mathbb{R}, +), (\mathbb{C}, +), (\mathbb{Z}_n, +)$.
    \item Числовые мультипликативные группы: $(\mathbb{Q} \setminus \{0\}, \times), (\mathbb{R} \setminus \{0\}, \times), (\mathbb{C} \setminus \{0\}, \times), (\mathbb{Z}_p \setminus \{0\}, \times)$, $p$ -- простое.
    \item Группы матриц: $GL_n(\mathbb{R}) = \{A \in Mat(n \times n, \mathbb{R}) \> | \> \det(A) \neq 0 \}; SL_n(\mathbb{R}) = \{A \in Mat(n \times n, \mathbb{R}) \> | \> \det(A) = 1 \}$.
    \item Группы подстановок: симметрическая группа $S_n$ -- все подстановки длины $n$, $|S_n| = n!$; знакопеременная группа $A_n$ -- четные подстановки длины $n$, $|A_n| = n!/2$.
\end{enumerate}

\subsection{Порядок группы.}
\begin{definition}
    \textbf{Порядок} группы $G$ -- это число элементов в $G$. Группа называется \textbf{конечной}, если ее порядок конечен, и \textbf{бесконечной} иначе.
\end{definition}


\subsection{Описание всех подгрупп в группе $(\mathbb{Z}, +)$.}
\begin{definition}
    Подмножество $H$ группы $G$ называется \textbf{подгруппой}, если выполнены следующий три условия:
    \begin{enumerate}
        \item $e \in H$
        \item $ab \in H$ для любых $a,b \in H$
        \item $a^{-1} \in H$ для любого $a \in H$
    \end{enumerate}
\end{definition}
\begin{statement}
    Всякая подгруппа в $(\mathbb{Z}, +)$ имеет вид $k\mathbb{Z} = \{ ka \> | \> a \in \mathbb{Z} \}$ для некоторого целого неотрицательного $k$.
    \begin{proof}
        Пусть $H$ -- подгруппа в $\mathbb{Z}$. Если $H = \{0\}$, положим $k = 0$. Иначе пусть $k = \min(H \cap \mathbb{N})$ -- наименьшее натуральное число, лежащее в $H$. Тогда $k\mathbb{Z} \subseteq H$. С другой стороны, если $a \in H$ и $a = qk + r$ -- результат деления $a$ на $k$ с остатком, то $0 \leqslant r \leqslant k - 1$ и $r = a - qk \in H$. Отсюда $r = 0$ и $H = k\mathbb{Z}$.
    \end{proof}
\end{statement}