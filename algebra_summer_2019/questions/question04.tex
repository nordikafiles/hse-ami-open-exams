\section{Смежные классы. Индекс подгруппы. Теорема Лагранжа.}

\subsection{Смежные классы.}
\begin{definition}
    Пусть $G$ -- группа, $H \subseteq G$ -- подгруппа и $g \in G$. \textbf{Левым смежным классом} элемента $g$ группы $G$ по подгруппе $H$ называется подмножество
    \[
        gH = \{gh \> | \> h \in H \}.
    \]
\end{definition}

\subsection{Индекс подгруппы.}
\begin{definition}
    Пусть $G$ -- группа и $H \subseteq G$ -- подгруппа. \textbf{Индексом подгруппы} $H$ в группе $G$ называется число левых смежных классов $G$ по $H$.
    Индекс группы $G$ по подгруппе $H$ обозначается $[G : H]$.
\end{definition}

\subsection{Теорема Лагранжа.}
\begin{lemma} \label{lemma1}
    Пусть $G$ -- группа, $H \subseteq G$ -- ее подгруппа и $g_1, g_2 \in G$. Тогда либо $g_1H = g_2H$, либо $g_1H \cap g_2H = \varnothing$.
    \begin{proof}
        Предположим, что $g_1G \cap g_2H \neq \varnothing$, т.е. $g_1h_1 = g_2h_2$ для некоторых $h_1, h_2 \in H$. Нужно доказать, что $g_1H = g_2H$. Заметим, что $g_1H = g_2h_2h_1^{-1}H \subseteq g_2H$. Обратное включение доказывается аналогично.
    \end{proof}
\end{lemma}
\begin{lemma} \label{lemma2}
    Пусть $G$ -- группа и $H \subseteq G$ -- конечная подгруппа. Тогда $|gH| = |H|$ для любого $g \in G$.
    \begin{proof}
        Поскольку $gH = \{gh \> | \>  h \in H\}$, в $gH$ элементов не больше, чем в $H$. Если $gh_1 = gh_2$, то домножаем слева на $g^{-1}$ и получаем $h_1 = h_2$. Значит, все элементы вида $gh$, где $h \in H$, попарно различны, откуда $|gH| = |H|$.
    \end{proof}
\end{lemma}
\begin{theorem}
    Пусть $G$ -- конечная группа и $H \subseteq G$ -- подгруппа. Тогда
    \[
        |G| = |H| \cdot [G : H].
    \]
    \begin{proof}
        Каждый элемент группы $G$ лежит в (своем) левом смежном классе по подгруппе $H$, разные смежные классы не пересекаются (лемма \ref{lemma1}) и каждый из них содержит по $|H|$ элементов (лемма \ref{lemma2}).
    \end{proof}
\end{theorem}