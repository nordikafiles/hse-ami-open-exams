\section{Пять следствий из теоремы Лагранжа.}

\subsection{Следствие 1.}
\begin{consequence} \label{consequence1}
    Пусть $G$ -- конечная группа и $H \subseteq G$ -- подгруппа. Тогда $|H|$ делит $|G|$.
\end{consequence}

\subsection{Следствие 2.}
\begin{consequence} \label{consequence2}
    Пусть $G$ -- конечная группа и $g \in G$. Тогда $ord(g)$ делит $|G|$.
    \begin{proof}
        Это вытекает из следствия \ref{consequence1} и утверждения \ref{statement1}.
    \end{proof}
\end{consequence}

\subsection{Следствие 3.}
\begin{consequence} \label{consequece3}
    Пусть $G$ -- конечная группа и $g \in G$. Тогда $g^{|G|} = e$.
    \begin{proof}
        Согласно следствию \ref{consequence2} мы имеем $|G| = ord(g) \cdot s$, откуда $g^|G| = (g^{ord(g)})^s = e^s = e$.
    \end{proof}
\end{consequence}

\subsection{Следствие 4.} \label{consequece4}
\begin{consequence}
    Пусть $G$ -- группа. Предположим, что $|G|$ -- простое число. Тогда  $G$ -- циклическая группа, порождаемая любым своим неединичным элементом.
    \begin{proof}
        Пусть $g \in G$ -- произвольный неединичный элемент. Тогда циклическая подгруппа $\langle g \rangle$ содержит более одного элемента и $|\langle g \rangle|$ делит $|G|$ по следствию \ref{consequence1}. Значит, $|\langle g \rangle| = |G|$, откуда $G = \langle g \rangle$.
    \end{proof}
\end{consequence}

\subsection{Следствие 5.} \label{consequece5}
\begin{consequence}[малая теорема Ферма]
    Пусть $p$ -- простое число и $\text{НОД} (a, p) = 1$. Тогда $a^{p-1} \equiv 1 \mod p$.
    \begin{proof}
        Применим следствие \ref{consequece3} к группе $(\mathbb{Z}_p \setminus \{0\}, \times)$.
    \end{proof}
\end{consequence}