\section{Криптография с открытым ключом. Задача дискретного логарифмирования. Система Диффи-Хеллмана обмена ключами. Криптосистема Эль-Гамаля.}

Пусть у нас есть $G$ -- конечная абелева группа. И также есть элемент $g \in G$, для которого $ord(g)$ будет достаточно большим значением.

\subsection{Задача дискретного логарифмирования.}
Дано: $h \in \langle g \rangle$. Найти такое $\alpha$, что $g^{\alpha} = h$.
Возведение в степень -- задача более простая с технической стороны реализации. Существует алгоритм бинарного возведения в степень: $g^{16} = ((((g)^2)^2)^2)^2$. Задача нахождения степени решается только перебором или близким к перебору способом.
\newline
\newline
\large \faYoutube \normalsize $\>$ \url{https://youtu.be/1oceAPu3b8o?t=4480}

\subsection{Система Диффи-Хеллмана обмена ключами.}
Группа $G$ и  некоторый ее элемент $g$ известны всем, причем $g$ имеет достаточно большой порядок.
Пусть есть два пользователя системы -- $A$ и $B$. $A$ фиксирует свое секретное $\alpha \in \mathbb{N}$ и сообщает всем пользователям $g^{\alpha}$. $B$ совершает аналогичные действия: фиксирует $\beta \in \mathbb{N}$ и сообщает всем пользователям $g^{\beta}$. Теперь $A$ и $B$ опять совершают аналогичные действия -- каждый из них возводит элемент другого в свою секретную степерь, они оба получают элемент $g^{\alpha \beta}$, который извествен только им двоим. Теперь по этому ключу можно устроить шифрованный канал связи, к которому никто не имеет доступа. В силу сложности задачи дискретного логарифмирования по $g^{\alpha}$ и $g^{\beta}$ нельзя быстро получить $g^{\alpha \beta}$.
\newline
\newline
\large \faYoutube \normalsize $\>$ \url{https://youtu.be/mNd30oeCugc?t=78}

\subsection{Криптосистема Эль-Гамаля.}
Группа $G$ и  некоторый ее элемент $g$ известны всем, причем $g$ имеет достаточно большой порядок. Пусть есть два пользователя системы -- $A$ и $B$. $A$ фиксирует свое секретное $\alpha \in \mathbb{N}$ и сообщает всем пользователям $g^{\alpha}$. $B$ хочет передать для $A$ элемент $h \in G$. Для этого $B$ фиксирует какое-то $k \in \mathbb{N}$ и объявляет пару $\{ g^k, h \cdot (g^{\alpha})^k \}$. 
\newline
\newline
\large \faYoutube \normalsize $\>$ \url{https://youtu.be/mNd30oeCugc?t=360}