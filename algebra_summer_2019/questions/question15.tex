\section{Идеалы колец. Факторкольцо кольца по идеалу. Гомоморфизмы и изоморфизмы колец. Ядро и образ гомоморфизма колец. Теорема о гомоморфизме колец.}

\subsection{Идеалы колец.}
\begin{definition}
    Пусть $R$ -- кольцо. Подмножество $I \subseteq R$ называется идеалом, если
    \begin{enumerate}
        \item $I$ -- подгруппа в $R$ по сложению
        \item $\forall \> r \in R, a \in I : ar, ra \in I$
    \end{enumerate}
    Обозначение: $I \triangleleft R$.
\end{definition}
В каждом кольце $R$ есть \textbf{несобственные идеалы} $I = \{0\}$ и $I = R$. Все остальные идеалы называются \textbf{собственными}.
\begin{definition}
    Идеал $I$ называется главным, если существует такой элемент $a \in R$, что $I = (a)$. (В этой ситуации говорят, что $I$ порожден элементом $a$.)
\end{definition}
\large \faYoutube \normalsize $\>$ \url{https://youtu.be/mNd30oeCugc?t=3367}

\subsection{Факторкольцо кольца по идеалу.}
Пусть $R$ -- кольцо, $I \triangleleft R$, $R / I$ -- факторгруппа по сложению. Введем на $R / I$ операцию умножения по формуле $(a + I)(b + I) = ab + I$.
\begin{statement}
    Указанная выше операция корректна.
    \begin{proof}
        $a + I = a' + I, b + I = b' + I \Rightarrow a' = a + x, b' = b + y$, где $x,y \in I$.
        \[
            (a' + I)(b' + I) = a'b' + I = (a + x)(b + y) + I = ab + ay + xb + xy + I = ab + I \> (\text{т.к. } ay, xb, xy \in I).
        \]
        Ясно, что $R / I$ -- кольцо.
    \end{proof}
\end{statement}
\begin{definition}
    $R / I$ называется \textbf{факторкольцом} кольца $R$ по идеалу $I$.
\end{definition}
\large \faYoutube \normalsize $\>$ \url{https://youtu.be/mNd30oeCugc?t=3984}

\subsection{Гомоморфизмы и изоморфизмы колец.}
\begin{definition}
    Пусть $R, S$ -- кольца. Отображение $\varphi : R \to S$ называется \textbf{гомоморфизмом}, если $\varphi(a + b) = \varphi(a) + \varphi(b)$ и $\varphi(ab) = \varphi(a)\varphi(b)$ для любых $a, b \in R$.
\end{definition}
\begin{definition}
    Пусть $R, S$ -- кольца. Гомоморфизм $\varphi : R \to S$ называется \textbf{изоморфизмом}, если $\varphi$ -- биекция.
\end{definition}
\large \faYoutube \normalsize $\>$ \url{https://youtu.be/mNd30oeCugc?t=3152}

\subsection{Ядро и образ гомоморфизма колец.}
Пусть $\varphi : R \to S$ -- гомоморфизм колец.
\begin{definition}
    Множество $\operatorname{Ker}(\varphi) = \{ a \in R \> | \> \varphi(a) = 0 \}$ называется \textbf{ядром} гомоморфизма $\varphi$.
\end{definition}
\begin{definition}
    Множество $\operatorname{Im}(\varphi) = \varphi(R)$ называется \textbf{образом} гомоморфизма $\varphi$.
\end{definition}
\large \faYoutube \normalsize $\>$ \url{https://youtu.be/mNd30oeCugc?t=4420}

\subsection{Теорема о гомоморфизме колец.}
\begin{theorem}
    Если $\varphi : R \to S$ -- гомоморфизм колец, то $R / \operatorname{Ker}(\varphi) \simeq \operatorname{Im}(\varphi)$.
    \begin{proof}
        Пусть $I = \operatorname{Ker}(\varphi)$.
        \[
            \psi : R / I \to \operatorname{Im}(\varphi), \> r + I \mapsto \varphi(r)
        \]
        Из теоремы о гомоморфизме групп известно, что $\psi$ -- изоморфизм групп $(R / I, +)$ и $(\operatorname{Im}(\varphi), +)$. Осталось проверить, что $\psi$ -- гомоморфизм колец:
        \[
            \psi((a + I)(b + I)) = \psi(ab + I) = \varphi(ab) = \varphi(a)\varphi(b) = \psi(a + I)\psi(b + I). 
        \]
    \end{proof}
\end{theorem}
\large \faYoutube \normalsize $\>$ \url{https://youtu.be/YdjrTEepVpg}