\section{Простые элементы. Факториальные кольца. Факториальность кольца многочленов от одной переменной над полем. \textcolor{orange}{(todo)}}

\subsection{Простые элементы.}
\begin{definition}
    Элемент $p \in R$ называется \textbf{простым}, если $p$ необратим и его нельзя представить в виде $p = ab$, где $a, b \in R$ необратимы.
\end{definition}
\large \faYoutube \normalsize $\>$ \url{https://youtu.be/YdjrTEepVpg?t=1859}

\subsection{Факториальные кольца.}
\begin{definition}
    Пусть $R$ -- коммутативное кольцо без делителей 0. $R$ называется \textbf{факториальным кольцом}, если любой ненулевой необратимый элемент $a \in R$ представим в виде $a = p_1 \cdot ... \cdot p_n$, где $p_i$ -- простое, причем такое разложение определено однозначно с точностью до перестановки множителей и ассоциированности.
\end{definition}
\large \faYoutube \normalsize $\>$ \url{https://youtu.be/YdjrTEepVpg?t=2446}

\subsection{Факториальность кольца многочленов от одной переменной над полем.}
--
\large \faYoutube \normalsize $\>$ \url{}