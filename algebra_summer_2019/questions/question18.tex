\section{Простые элементы. Факториальные кольца. Факториальность кольца многочленов от одной переменной над полем.}

\subsection{Простые элементы.}
\begin{definition}
    Элемент $p \in R$ называется \textbf{простым}, если $p$ необратим и его нельзя представить в виде $p = ab$, где $a, b \in R$ необратимы.
\end{definition}
\large \faYoutube \normalsize $\>$ \url{https://youtu.be/YdjrTEepVpg?t=1859}

\subsection{Факториальные кольца.}
\begin{definition}
    Пусть $R$ -- коммутативное кольцо без делителей 0. $R$ называется \textbf{факториальным кольцом}, если любой ненулевой необратимый элемент $a \in R$ представим в виде $a = p_1 \cdot ... \cdot p_n$, где $p_i$ -- простое, причем такое разложение определено однозначно с точностью до перестановки множителей и ассоциированности.
\end{definition}
\large \faYoutube \normalsize $\>$ \url{https://youtu.be/YdjrTEepVpg?t=2446}

\subsection{Факториальность кольца многочленов от одной переменной над полем.}
% \begin{theorem}
%     Если $R$ -- кольцо главных идеалов, то оно факториально.
%     \begin{proof}
        
%     \end{proof}
% \end{theorem}
% \large \faYoutube \normalsize $\>$ \url{https://youtu.be/YdjrTEepVpg?t=2815}
\begin{definition}
    Простые элементы в $K[x]$ называются \textbf{неприводимыми многочленами}.
\end{definition}
\begin{theorem}
    $K[x]$ факториально.
    \begin{proof}
        Докажем индукцией по $\deg(f) = n$. Если $n = 1$, то $f = ax + b$ -- неприводимый многочлен. Пусть $n > 1$. Докажем существование. Если $f$ неприводим, то разложение будет иметь вид $f = f$. Если $f$ приводим, то $f = g \cdot h$, где $\deg(g), \deg(h) < n$. По предположению индукции $g = p_1 \cdot ... \cdot p_s$ и $h = q_1 \cdot ... \cdot q_t$, где $p_i, q_i$ -- неприводимые многочлены. Следовательно, $f = p_1 \cdot ... \cdot p_s \cdot q_1 \cdot ... \cdot q_t$ -- разложение на неприводимые многочлены. Докажем единственность. Пусть $f = p_1 \cdot ... \cdot p_n = q_1 \cdot ... \cdot q_n$, где $p_i, q_i$ -- неприводимые многочлены. Тогда $p_1$ делит $q_i$ для некоторого $i$. Без ограничения общности считаем $i = 1$. $p_1$ делит $p_1 \Rightarrow q_1 = \epsilon \cdot p_1 \Rightarrow p_1 \cdot p_2 \cdot ... \cdot p_n = p_1 \cdot \epsilon \cdot q_2 \cdot ... \cdot q_m$. Так как в $K[x]$ нет делителей нуля, сократив на $p_1$, получаем $p_2 \cdot ... \cdot p_n = (\epsilon q_2) \cdot ... \cdot q_m$. Так как $\deg < n$, то можно воспользоваться предположением индукции.
    \end{proof}
\end{theorem}
\large \faYoutube \normalsize $\>$ \url{https://youtu.be/YdjrTEepVpg?t=4364}