\section{Гомоморфизмы групп. Простейшие свойства гомоморфизмов. Изоморфизмы групп. Ядро и образ гомоморфизма групп, их свойства.}

\subsection{Гомоморфизмы групп.}
\begin{definition}
    Пусть $G$ и $F$ -- группы. Отображение $\varphi : G \to F$ называется \textbf{гомоморфизмом}, если $\varphi(ab) = \varphi(a)\varphi(b)$ для любых $a, b \in G$.
\end{definition}

\subsection{Простейшие свойства гомоморфизмов.}
\begin{lemma}
    Пусть $\varphi : G \to F$ -- гомоморфизм групп и пусть $e_G$ и $e_F$ -- нейтральные элементы групп $G$ и $F$ соответственно. Тогда
    \begin{enumerate}
        \item[(а)] $\varphi(e_G) = e_F$
        \item[(б)] $\varphi(a^{-1}) = \varphi(a)^{-1}$ для любого $a \in G$.
    \end{enumerate}
    \begin{proof}
        (а) Имеем $\varphi(e_G) = \varphi(e_Ge_G) = \varphi(e_G)\varphi(e_G)$. Теперь умножая крайние части этого равенства на $\varphi(e_G)^{-1}$ (например, слева) получим $e_F = \varphi(e_G)$.
        \newline
        (б) Имеем $\varphi(a^{-1})\varphi(a) = \varphi(a^{-1}a) = \varphi(e_G) = e_F$, откуда $\varphi(a^{-1}) = \varphi(a)^{-1}$.
    \end{proof}
\end{lemma}

\subsection{Изоморфизмы групп.}
\begin{definition}
    Гомоморфизм групп $\varphi : G \to F$ называется \textbf{изоморфизмом}. если отображение $\varphi$ биективно.
\end{definition}

\subsection{Ядро и образ гомоморфизма групп, их свойства.}
\begin{definition}
    С каждым гомоморфизмом групп $\varphi : G \to F$ связаны его ядро
    \[
        \operatorname{Ker}(\varphi) = \{g \in G \> | \> \varphi(g) = e_F \}
    \]
    и образ
    \[
        \operatorname{Im}(\varphi) = \{a \in F \> | \> \exists \> g \in G : \varphi(g) = a \}.
    \]
    Ясно, что $\operatorname{Ker}(\varphi) \subseteq G$ и $\operatorname{Im}(\varphi) \subseteq F$ -- подгруппы.
\end{definition}
\begin{lemma}
    Гомоморфизм групп $\varphi : G \to F$ инъективен тогда и только тогда, когда $\operatorname{Ker}(\varphi) = \{e_G\}$.
    \begin{proof}
        Ясно, что если $\varphi$ инъективен, то $\operatorname{Ker}(\varphi) = \{e_G\}$. Обратно, пусть $g_1, g_2 \in G$ и $\varphi(g_1) = \varphi(g_2)$. Тогда $g_1^{-1}g_2 \in \operatorname{Ker}(\varphi)$, поскольку $\varphi(g_1^{-1}g_2) = \varphi(g_1^{-1})\varphi(g_2) = \varphi(g_1)^{-1}\varphi(g_2) = e_F$. Отсюда $g_1^{-1}g_2 = e_G$ и $g_1 = g_2$.
    \end{proof}
\end{lemma}
\begin{consequence}
    Гомоморфизм групп $\varphi : G \to F$ является изоморфизмом тогда и только тогда, когда $\operatorname{Ker}(\varphi) = \{e_G\}$ и $\operatorname{Im}(\varphi) = F$.
\end{consequence}
\begin{statement}
    Пусть $\varphi : G \to F$ -- гомоморфизм групп. Тогда подгруппа $\operatorname{Ker}(\varphi)$ нормальна в $G$.
    \begin{proof}
        Достаточно проверить, что $g^{-1}hg \in \operatorname{Ker}(\varphi)$ для любых $g \in G$ и $h \in \operatorname{Ker}(\varphi)$. Это следует из цепочки равенств
        \[
            \varphi(g^{-1}hg) = \varphi(g^{-1})\varphi(h)\varphi(g) = \varphi(g^{-1})e_F\varphi(g) = \varphi(g^{-1})\varphi(g) = e_F.
        \]
    \end{proof}
\end{statement}