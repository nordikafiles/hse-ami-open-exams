\section{Делимость и ассоциированные элементы в коммутативных кольцах без делителей нуля. Наибольший общий делитель. Кольца главных идеалов. Существование наибольшего общего делителя и его линейного выражения в кольце главных идеалов.}

\subsection{Делимость и ассоциированные элементы в коммутативных кольцах без делителей нуля.}
Пусть $R$ -- коммутативное кольцо без делителей нуля, $a, b \in R$.
\begin{definition}
    Говорят, что $b$ делит $a$ (<<$a$ делится на $b$>>, <<$b$ делитель $a$>>), если $a = bc$ для некоторого $c \in R$.
\end{definition}
\begin{definition}
    Говорят, что $a$ и $b$ ассоциированны, если $a = bc$, где $c \in R$ -- обратимый элемент.
\end{definition}
\large \faYoutube \normalsize $\>$ \url{https://youtu.be/YdjrTEepVpg?t=343}

\subsection{Наибольший общий делитель.}
\begin{definition}
    \textbf{Наибольший общий делитель} элементов $a$ и $b$ -- это их общий делитель, который делится на любой другой общий делитель. Обозначение: $(a, b)$.
\end{definition}
\large \faYoutube \normalsize $\>$ \url{https://youtu.be/YdjrTEepVpg?t=879}

\subsection{Кольца главных идеалов.}
\begin{definition}
    $R$ называется \textbf{кольцом главных идеалов}, если всякий идеал в $R$ является главным.
\end{definition}
\large \faYoutube \normalsize $\>$ \url{https://youtu.be/YdjrTEepVpg?t=1253}

\subsection{Существование наибольшего общего делителя и его линейного выражения в кольце главных идеалов.}
\begin{theorem}
    Пусть $R$ -- кольцо главных идеалов. Тогда для любых $a, b \in R$
    \begin{enumerate}
        \item Существует наибольший общий делитель $(a, b)$.
        \item $(a, b) = au + bv$ для некоторых $u, v \in R$.
    \end{enumerate}
    \begin{proof}
        Рассмотрим идеал $I = \{ ax + by \> | \> x, y \in R \}$. Заметим, что $a = a \cdot 1 + b \cdot 0 \in I$ и $b = a \cdot 0 + b \cdot 1 \in I$. Так как $R$ -- кольцо главных идеалов, то существует $d \in R$, такое что $I = (d)$. Следовательно, $a$ делится на $d$ и $b$ делится на $d$, т.е. $d$ общий делитель $a$ и $b$. $d \in I \Rightarrow d = au + bv$ для некоторых $u, v \in R$. Если $d'$ -- какой-то общий делитель $a$ и $b$, то $d'$ делит $d$. Таким образом, $d$ -- наибольший общий делитель $a$ и $b$.
    \end{proof}
\end{theorem}
\large \faYoutube \normalsize $\>$ \url{https://youtu.be/YdjrTEepVpg?t=1435}
