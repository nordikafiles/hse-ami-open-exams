\section{Нормальные подгруппы и факторгруппы.}

\subsection{Нормальные подгруппы.}
\begin{definition}
    Подгруппа $H$ группы $G$ называется \textbf{нормальной}, если $gH = Hg$ для любого $g \in G$.
\end{definition}
\subsubsection{Эквивалентность условий нормальности группы.}
\begin{statement}
    Для подгруппы $H \subseteq G$ следующие условия эквивалентны:
    \begin{enumerate}
        \item $H$ нормальна
        \item $gHg^{-1} \subseteq H$ для любого $g \in G$
        \item $gHg^{-1} = H$ для любого $g \in G$
    \end{enumerate}
    \begin{proof}
        (1) $\Rightarrow$ (2) Пусть $h \in H$ и $g \in G$. Поскольку $gH = Hg$, имеем $gh = h'g$ для некоторого $h' \in H$. Тогда $ghg^{-1} = h'gg^{-1} = h' \in H$.
        \newline
        (2) $\Rightarrow$ (3) Так как $gHg^{-1} \in H$, остается проверить обратное включение. Для $h \in H$ имеем $h = gg^{-1}hgg^{-1} = g(g^{-1}hg)g^{-1} \in gHg^{-1}$, поскольку $g^{-1}hg \in H$ в силу пункта (2), где вместо $g$ взято $g^{-1}$.
        \newline
        (3) $\Rightarrow$ (1) Для произвольного $g \in G$ в силу (3) имеем $gH = gHg^{-1}g \subseteq Hg$, так что $gH \subseteq Hg$. Аналогично проверяется обратное включение.
        \newline
    \end{proof}
\end{statement}

\subsection{Факторгруппы.}
\subsubsection{Корректность.}
Обозначим через $G / H$ множество смежных классов группы $G$ по нормальной подгруппе $H$. На $G / H$ можно определить бинарную операцию следующим образом:
\[
    (g_1H)(g_2H) := g_1g_2H.
\]
\begin{statement}
    Указанная выше операция корректна.
    \begin{proof}
        Заменим $g_1$ и $g_2$ другими представителями $g_1h_1$ и $g_2h_2$ тех же смежных классов. Нужно проверить, что $g_1g_2H = g_1h_1g_2h_2H$. Это следует из того, что $g_1h_1g_2h_2 = g_1g_2(g_2^{-1}h_1g_2)h_2$ и $g_2^{-1}h_1g_2$ лежит в $H$.
        Ясно, что указанная операция на множестве $G / H$ ассоциативна, обладает нейтральным элементом $eH$ и для каждого элемента $gH$ есть обратный элемент $g^{-1}H$.
    \end{proof}
\end{statement}
\begin{definition}
    Множество $G / H$ с указанной операцией называется \textbf{факторгруппой} группы $G$ по нормальной подгруппе $H$.
\end{definition}
\subsubsection{Примеры факторгрупп.}
\begin{enumerate}
    \item Если $G = (\mathbb{Z}, +)$ и $H = n\mathbb{Z}$, то $G/H$ -- это в точности группа вычетов $(\mathbb{Z}_n, +)$.
\end{enumerate}
