\section{Подгруппы. Циклические подгруппы. Циклические группы. Порядок элемента. Связь между порядком элемента и порядком порождаемой им циклической подгруппы.}

% \subsection{Подгруппы.}

\subsection{Циклические подгруппы.}
\begin{definition}
    Пусть $G$ -- группа и $g \in G$. \textbf{Циклической подгруппой}, порожденной элементом $g$, называется подмножество $\{g^n \> | \> n \in \mathbb{Z} \}$. Циклическая подгруппа, порожденная элементом $g$, обозначается $\langle g \rangle$. Элемент $g$ называется \textbf{порождающим} или \textbf{образующим} для подгруппы $\langle g \rangle$.
\end{definition}

\subsection{Циклические группы.}
\begin{definition}
    Группа $G$ называется \textbf{циклической}, если найдется такой элемент $g \in G$, что $G = \langle g \rangle$.
\end{definition}

\subsection{Порядок элемента.}
\begin{definition}
    Пусть $G$ -- группа и $g \in G$. \textbf{Порядком элемента} $g$ называется такое наименьшее натуральное число $m$, что $g^m = e$. Если такого натурального числа $m$ не существует, говорят, что порядок элемента $g$ равен бесконечности.
    Порядок элемента обозначается $ord(g)$.
\end{definition}

\subsection{Связь между порядком элемента и порядком порождаемой им циклической подгруппы.}
\begin{statement}
    Пусть $G$ -- группа и $g \in G$. Тогда $ord(g) = |\langle g \rangle|$.
    \begin{proof}
        Заметим, что если $g^k = g^s$, то $g^{k-s} = e$. Поэтому если элемент $g$ имеет бесконечный порядок, то все элементы $g^n, n \in \mathbb{Z}$, попарно различны и подгруппа $\langle g \rangle$ содержит бесконечно много элементов. Если же порядок элемента $g$ равен $m$, то из минимальности числа $m$ следует, что элеметы $e = g^0, g = g^1, g^2,..., g^{m-1}$ попарно различны. Далее, для всякого $n \in \mathbb{Z}$ мы имеем $n = mq + r$, где $0 \leqslant r \leqslant m-1$, и
        \[
            g^n = g^{mq + r} = (g^m)^q g^r = e^q g^r = g^r.
        \]
        Следовательно, $\langle g \rangle = \{e, g, ..., g^{m-1}\}$ и $|\langle g \rangle | = m$.
    \end{proof}
\end{statement}
