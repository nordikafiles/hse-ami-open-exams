\section{Случайная величина и ее распределение. Функция распределения случайной величины. Совместное распределение двух случайных величин. Свойства функции распределения.}

\subsection{Случайная величина и ее распределение.}
Пусть $(\Omega, A, P)$ -- вероятностное пространство.
\begin{definition}
    Функция $\xi : \Omega \to \mathbb{R}$ называется случайной величиной, если для любого промежутка $I$ выполнено:
    \[
        \xi^{-1}(I) = \{ w \> | \> \xi(w) \in I \} \in A.
    \]
\end{definition}
\begin{definition}
    Распределением случайной величины $\xi$ называется вероятностная мера $\mu_{\xi}$ на $B = \mathcal{B}(\mathbb{R})$, определяемая равенством
    \[
        \mu_{\xi} (B) = P(\{ w \> | \> \xi(w) \in B \}) = P(\xi^{-1}(B)).
    \]
\end{definition}

\subsection{Функция распределения случайной величины.}
\begin{definition}
    Функцией распределения $F_\xi$ вероятностной меры $\mu_\xi$ называется функцией распределения случайной величины $\xi$, то есть
    \[
        F_\xi(t) = \mu_\xi((-\infty, t]) = P(\{ w \> | \> \xi(w) \leqslant t \}),
    \]
    мера $\mu_\xi$ показывает с какой вероятностью $\xi$ принимает те или иные значения.
\end{definition}

\subsection{Совместное распределение двух случайных величин.}
Пусть $\xi$ и $\eta$ -- случайные велечины.
\begin{definition}
    Отображение  $w \mapsto (\xi(w), \eta(w))$ определяет вероятностную меру $\mu(B) = P(\{ w \> | \> (\xi(w), \eta(w)) \in B\})$ -- совместное распределение случайных величин $\xi$ и $\eta$:
    \[
        F(x,y) = \mu((-\infty, x] \times (-\infty, y]) = P(\{ w \> | \> \xi(w) \leqslant x \wedge \eta(w) \leqslant y \}).
    \]
\end{definition}

\subsection{Свойства функции распределения.}
\begin{theorem}
    Если $F$ -- функция распределения, то
    \begin{enumerate}
        \item $0 \leqslant F \leqslant 1$
        \item $F$ неубывает
        \item $F$ непрерывна справа, т.е. $\lim_{t \to s+} F(t) = F(s)$
        \item $\lim_{t \to -\infty} F(t) = 0$ и $\lim_{t \to \infty} F(t) = 1$
    \end{enumerate}
    \begin{proof}
        \text{}
        \begin{enumerate}
            \item Очевидно, т.к. $0 \leqslant P \leqslant 1$
            \item $b > a \Rightarrow F(b) - F(a) = P(a \leqslant \xi \leqslant b)$
            \item Найдем $\lim_{t \to s+} F(t)$. Пусть
            $
                A_n = \left( -\infty, s + \frac{1}{n} \right], A_{n+1} \subset A_n, \bigcap_n A_n  = (-\infty, s].
            $
            Из непрерывности меры $\mu$ следует, что
            \[
                \mu(A_n) \to \mu \left( \bigcap_n A_n \right)
                \Rightarrow
                F \left( s + \frac{1}{n} \right) \to F(s).
            \]
            \item Доказывается аналогично 3 свойству.
        \end{enumerate}
    \end{proof}
\end{theorem}