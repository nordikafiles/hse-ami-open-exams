\section{Общее определение математического ожидания и его корректность. Математическое ожидание случайной величины, распределение которое задано плотностью.}

\subsection{Общее определение математического ожидания и его корректность.}

\begin{theorem}
    Для любой случайной величины $\xi$ существует последовательность случайных величин $\{\xi_n\}$ такая, что $ \xi_n \rightrightarrows \xi$ на $\Omega$.
    \begin{proof}
        Пусть $\xi_n = 10^{-n} \cdot \lfloor 10^n \cdot \xi \rfloor$. Тогда $\sup |\xi_n - \xi| \leqslant 10^{-n} \to 0$.
    \end{proof}
\end{theorem}

\begin{definition}
    Пусть множество значений $\{x_1, x_2, x_3, ...\}$ дискретной случайной величины $\xi$ бесконечно. Положим $A_i = \xi^{-1}(\{x_i\})$. Будем говорить, что у $\xi$ существует конечное математическое ожидание если ряд $E \xi = \sum_{k=1}^{\infty} x_k P(A_k)$ сходится абсолютно.
    \begin{proof}[Доказательство корректности.]
        Так как перестановка членов абсолютно сходящегося ряда не влияет на сходимость и сумму ряда, а произведение абсолютно сходящихся рядов сходится к произведению их сумм, то все свойства математического ожидания будут выполняться и для суммы этого ряда.
    \end{proof}
\end{definition}

\subsection{Математическое ожидание случайной величины, распределение которое задано плотностью.}
\begin{theorem}
    Пусть $\varphi$ -- кусочно-непрерывная функция на $\mathbb{R}$ и $\xi : \Omega \to \mathbb{R}$ -- случайная величина, распределение которой задано плотностью $\rho_\xi$, тогда 
    \[
        \exists E(\varphi(\xi))
        \Leftrightarrow
        \int \limits_{-\infty}^{\infty} |\varphi(x) \rho_\xi(x) | dx
        \text{ сходится. }
    \]
    В случае сходимости
    \[
        E(\varphi(\xi)) = \int \limits_{-\infty}^{\infty} \varphi(x) \rho_\xi(x) dx.
    \]
    \begin{proof}
        Докажем для кусочно-постоянных функций. Пусть $f$ -- кусочно-постоянная функция, а это значит, что $f(\xi)$ -- дискретная величина, тогда
        \[
            E(f(\xi)) =
            \sum_n C_n P(A_n) =
            \sum_n C_n \int \limits_{\Delta n} \rho_\xi(x) dx =
            \sum_n \int \limits_{\Delta n} f(x) \rho_\xi(x) dx =
            \int \limits_{-\infty}^{+\infty} f(x) \rho_\xi(x) dx.
        \]
        В общем случае мы можем разбивать числовую прямую на счетное число промежутков таким образом, чтобы каждое такое разбиение задавало кусочно-постоянную функцию $f_n(x)$, и при этом $f_n(\xi) \rightrightarrows \varphi(\xi)$, тогда получим то же утверждение для кусочно-непрерывной функции $\varphi(x)$.
    \end{proof}
\end{theorem}