\section{Вероятностное пространство. Сигма алгебра событий. Борелевская сигма алгебра. Вероятностная мера. Непрерывность вероятностной меры.}

\subsection{Вероятностное пространство.}

\begin{definition}
    Класс множеств, который содержит $\varnothing$ и $\Omega$, замкнутый относительно операций $\cap$ и $\cup$, содержит вместе с каждым множеством его дополнение и называется \textbf{алгеброй множеств} или \textbf{алгеброй событий}.
\end{definition}

\subsection{Сигма алгебра событий.}
\begin{definition}
    Если алгебра событий замкнута относительно счетных объединений и пересечений, то ее называют $\sigma$-алгеброй. 
\end{definition}

\subsection{Борелевская сигма алгебра.}
\begin{definition}
    \text{}
    \begin{itemize}
        \item Борелевская $\sigma$-алгебра -- минимальная $\sigma$-алгебра, содержащая все открытые подмножества топологического пространства.
        \item Борелевская $\sigma$-алгебра $\mathcal{B}(\mathbb{R})$  -- $\sigma$-алгебра, порожденная отрезками, интервалами или полуинтервалами.
    \end{itemize}
\end{definition}

\subsection{Вероятностная мера.}
Пусть $A$ -- $\sigma$-алгебра.
\begin{definition}
    Функция $P : A \to [0, 1]$ называется \textbf{вероятностной мерой}, если
    \begin{itemize}
        \item $P(\Omega) = 1$
        \item Для любого набора попарно непересекащихся событий $\{A_n\} \in A$ выполняется $P(\bigcup_n A_n) = \sum_n P(A_n)$.
    \end{itemize}
\end{definition}

\subsection{Непрерывность вероятностной меры.}
\begin{theorem}
    Пусть $(\Omega, A, P)$ -- вероятностное пространство. Тогда
    \begin{enumerate}
        \item Если $\{A_n\} \in A, A_n \subset A_{n+1}$ и $A = \bigcup_n A_n$, то $\lim_{n \to \infty} P(A_n) = P(A)$.
        \item Если $\{A_n\} \in A, A_{n+1} \subset A_n$ и $A = \bigcap_n A_n$, то $\lim_{n \to \infty} P(A_n) = P(A)$.
    \end{enumerate}
    \begin{proof}
        \text{}
        \begin{enumerate}
            \item Пусть $C_{n+1} = A_{n+1} \setminus A_n, C_1 = A_1$. Тогда $A = \bigcup_n C_n$ и $A_{n+1} = \bigcup_{k=1}^n C_k$. По своству аддитивности вероятностной меры $P$ получаем:
            \[
                P(A) = \sum_n P(C_n) = \lim_{n \to \infty} \sum_{k=1}^n P(C_k) = \lim_{n \to \infty} P(A_{n+1}).
            \]
            \item Пусть $A'_n = \Omega \setminus A_n$. Тогда по закону де Моргана получаем первый пункт.
        \end{enumerate}
    \end{proof}
\end{theorem}