\section{Закон больших чисел в слабой форме. Метод Монте-Карло.}

\subsection{Закон больших чисел в слабой форме.}
\begin{theorem}
    Пусть $\{\xi_n\}$ -- последовательность независимых случайных величин таких, что $E \xi_n^2 < \infty$. Обозначим $E \xi_n = \mu_n$ и $D \xi_n = \sigma_n^2$. Тогда
    \[
        \frac{\sigma_1^2 + ... + \sigma_n^2}{n^2} \to 0
        \Rightarrow
        P \left(
            \left|
                \frac{\xi_1 + ... + \xi_n}{n}
                -
                \frac{\mu_1 + ... + \mu_n}{n}
            \right| \geqslant \varepsilon
        \right) \to 0
    \]
    \begin{proof}
        Из неравенства Чебышева:
        \[
            P \left(
                \left|
                    \frac{\xi_1 + ... + \xi_n}{n}
                    -
                    \frac{\mu_1 + ... + \mu_n}{n}
                \right| \geqslant \varepsilon
            \right) \leqslant
            \frac{\sigma_1^2 + ... + \sigma_n^2}{n^2 \varepsilon^2} \to 0
        \]
    \end{proof}
\end{theorem}

\subsection{Метод Монте-Карло.}
Предположим, требуется вычислить определенный интеграл $\int \limits_a^b f(x) dx$.
Рассмотрим случайную величину $\xi$, равномерно распределенную на отрезке $[a,b]$. Тогда $f(\xi)$ также будет случайной величиной, причем ее математическое ожидание выражается как
\[
    E f(\xi) = \int \limits_a^b f(x) \rho_\xi(x) dx = \int \limits_a^b f(x) \frac{1}{b - a} dx
\]
Таким образом, интеграл выражается как
\[
    \int \limits_a^b f(x) dx = (b - a) E f(\xi) \approx
    \frac{b - a}{N} \sum_{i=1}^N f(x_i).
\]