\section{Дисперсия и ее свойства. Ковариация и коэффициент корреляции двух случайных величин, геометрический смысл.}

\subsection{Дисперсия и ее свойства.}
\begin{definition}
    Дисперсией случайной величины $\xi$ называют число
    \[
        D \xi = E(\xi - E \xi)^2.
    \]
\end{definition}
\begin{statement}
    Дисперсия может быть вычислена по формуле
    \[
        D \xi = E \xi^2 - (E \xi)^2.
    \]
    \begin{proof}
        Следует из линейности математического ожидания.
    \end{proof}
\end{statement}
\begin{statement}
    При умножении случайной величины на константу $c$ дисперсия увеличивается в $c^2$ раз.
    \begin{proof}
        Следует из линейности математического ожидания.
    \end{proof}
\end{statement}
\begin{statement}
    Дисперсия всегда неотрицательна.
    \begin{proof}
        Очевидно.
    \end{proof}
\end{statement}
\begin{statement}
    $D \xi = 0 \Leftrightarrow \xi = const$ почти наверняка.
    \begin{proof}
        Очевидно, т.к. если $D \xi = 0$, то $\xi = E \xi$ почти наверняка.
    \end{proof}
\end{statement}
\begin{statement}
    Если случайные величины независимы, то дисперсия их суммы равна сумме их дисперсий.
    \begin{proof}
        Следует из того, что математическое ожидание произведения независимых величин равна произведению математических ожиданий этих величин.
    \end{proof}
\end{statement}
\begin{statement}
    Дисперсия не зависит от сдвига случайной величины на константу.
    \begin{proof}
        Следует из предыдущего утверждения.
    \end{proof}
\end{statement}
\begin{statement}
    Для дисперсии справедливо неравенство Чебышева.
    \begin{proof}
        Следует из того, что дисперсия является математическим ожиданием.
    \end{proof}
\end{statement}

\subsection{Ковариация и коэффициент корреляции двух случайных величин.}
\begin{definition}
    Ковариацией $\operatorname{cov}  (\xi, \eta)$ случайных величин $\xi$ и $\eta$ называется число
    \[
        \operatorname{cov} (\xi, \eta) = E((\xi - E \xi)(\eta - E \eta)).
    \]
    Она существует если существуют $D \xi $ и $D \eta$.
\end{definition}
\begin{definition}
    Коэффициентом корреляции $\rho(\xi, \eta)$ случайных величин $\xi$ и $\eta$, дисперсии которых отличны от 0, называется число
    \[
        \rho(\xi, \eta) = \frac{
            \operatorname{cov}  (\xi, \eta)
        }{
            \sqrt{D \xi} \sqrt{D \xi}
        }.
    \]
\end{definition}

\subsection{Геометрический смысл коэффициента корреляции двух случайных величин.}
Ковариация -- "скалярное произведение", а коэффициент корреляции -- "косинус угла".