\section{Математическое ожидание дискретной случайной величины и его свойства: линейность, монотонность, неравенство Чебышева. Математическое ожидание произведения независимых величин.}

\subsection{Математическое ожидание дискретной случайной величины и его свойства.}
Пусть $\xi$ -- случайная дискретная величина на $(\Omega, A, P)$, принимающая значения $\{x_1, ..., x_n\}$. Положим $A_i = \xi^{-1}(\{x_i\})$. Тогда
\[
    \xi = x_1 \mathbb{I}_{A_1} + ... + x_n \mathbb{I}_{A_n}.
\]
\begin{definition}
    Математическим ожиданием случайной величины $\xi$ называется число
    \[
        E \xi = x_1 P(A_1) + ... + x_n P(A_n).
    \]
\end{definition}

\begin{theorem}
    \text{}
    \begin{enumerate}
        \item $E( \alpha \xi + \beta \eta ) = \alpha E \xi + \beta E \eta$
        \item Если $\xi \geqslant \eta$ почти наверняка, то $E \xi \geqslant E \eta$.
        \item Если $\xi \geqslant 0$, то для любого $C > 0$ верно $P(\xi \geqslant C) \leqslant \frac{E \xi}{C}$
    \end{enumerate}
    \begin{proof}
        \text{}
        \begin{enumerate}
            \item Следует из определения математического ожидания дискретной случайной величины и разложения случайной величины в сумму произведений значений и индикаторов.
            \item $0 \leqslant E(\eta - \xi) = E\eta - E \xi \Rightarrow E\eta \geqslant E \xi$
            \item Пусть $A = \{ w \> | \> \xi \geqslant C \}$. Тогда
            \[
                \xi \geqslant C \cdot \mathbb{I}_A.
            \] Неравенство выполняется для любых значений $\mathbb{I}_A$. Применив свойства монотонности и линейности получим:
            \[
                E \xi \geqslant C E \mathbb{I}_A = C P(A) \Rightarrow \frac{E \xi}{C} \geqslant P(A).
            \]
        \end{enumerate}
    \end{proof}
\end{theorem}

\subsection{Математическое ожидание произведения независимых величин.}
\begin{theorem}
    Если случайные величины $\xi$ и $\eta$ независимы, то $E (\xi \cdot \eta) = E \xi \cdot E \eta$.
    \begin{proof}
        Разложим $\xi$ и $\eta$ в сумму произведений значений и индикаторов:
        \[
            \xi = \sum_i a_i \mathbb{I}_{A_i},
            \eta = \sum_j b_j \mathbb{I}_{B_j}
            \Rightarrow
            \xi \cdot \eta = \sum_{i,j} a_ib_j \mathbb{I}_{A_i \cap B_j}
            \Rightarrow
            E(\xi \cdot \eta) = \sum_{i,j} a_ib_j P(A_i \cap B_j) =
            \sum_{i,j} a_ib_j P(A_i)P(B_j) = E \xi \cdot E \eta
        \]
    \end{proof}
\end{theorem}