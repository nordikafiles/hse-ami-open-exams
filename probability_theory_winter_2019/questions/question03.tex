\section{Независимые случайные величины. Характеризация независимости в терминах функций распределения и плотностей. Плотность распределения суммы двух независимых случайных величин.}

\subsection{Независимые случайные величины.}
\begin{definition}
    Случайные величины $\xi$ и $\eta$ называются независимыми, если для всяких промежутков $U$ и $V$ выполняется равенство $$
        P(\{ w \> | \> \xi(w) \in U \wedge \eta(w) \in V \}) =
        P(\{ w \> | \> \xi(w) \in U \}) \cdot P(\{ w \> | \> \eta(w) \in V \}),
        \text{ то есть }
        \mu_\xi(U) = \mu_\eta(V).
    $$
\end{definition}

\subsection{Характеризация независимости в терминах функций распределения и плотностей.}
\begin{theorem}
    Случайные величины $\xi$ и $\eta$ независимы тогда и только тогда, когда
    $F(x,y) = F_\xi(x) \cdot F_\eta(y)$.
    \begin{proof}
        Совместное распределение одднозначно определяется функцией распределения $F$. Если $F$ совпадает с функцией распределения меры $\mu_\xi \times \mu_\eta$, то меры совпадают.
    \end{proof}
\end{theorem}
\begin{theorem}
    Пусть распределения $\xi$ и $\eta$ заданы плотностями. Тогда независимость $\xi$ и $\eta$ равносильна тому, что совместное распределение задано плотностью
    \[
        \rho(x, y) = \rho_\xi(x) \cdot \rho_\eta(y).
    \]
    \begin{proof}
        По теореме Фубини:
        \[
            \int \limits_a^b \rho_\xi(x) dx \cdot 
            \int \limits_c^d \rho_\eta(x) dy =
            \iint \limits_{[a, b] \times [c, d]} \rho_\xi(x) \rho_\eta(y) dx dy =
            \mu([a, b] \times [c, d]).
        \]
    \end{proof}
\end{theorem}

\subsection{Плотность распределения суммы двух независимых случайных величин.}
\begin{theorem}
    Пусть $\xi$ и $\eta$ независимы и их распределение задано плотностями. Тогда распределение суммы $\nu = \xi + \eta$ задано плотностью 
    \[
        \rho_\nu(x) = \int \limits_{-\infty}^{+\infty} \rho_\xi(t) \rho_\eta(x - t) dt.
    \]
    \begin{proof}
        \[
            F_\nu(t) =
            P(\{ w \> | \> \xi(w) + \eta(w) \leqslant t \})
            = \iint \limits_{x + y \leqslant t} \rho_\xi(x) \rho_\eta(y) dxdy =
            \int \limits_{-\infty}^{+\infty} \left(
                \int \limits_{-\infty}^{t - x} \rho_\xi(x) \rho_\eta(y) dy
            \right) dx =
            \int \limits_{-\infty}^{+\infty} \rho_\xi(x)  \left(
                \int \limits_{-\infty}^{t - x}  \rho_\eta(y) dy
            \right) dx =
        \]
        \[
            = \int \limits_{-\infty}^{+\infty} \rho_\xi(x)  \left(
                \int \limits_{-\infty}^{t}  \rho_\eta(v - x) dv
            \right) dx =
            \int \limits_{-\infty}^{t}   \left(
                \int \limits_{-\infty}^{+\infty} \rho_\xi(x) \rho_\eta(v - x) dv
            \right) dx
        \]
    \end{proof}
\end{theorem}